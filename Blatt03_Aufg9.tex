\subsection*{Aufgabe 9}
Gegeben:
\begin{align*} \label{eq8}
A=
\begin{split}
\begin{pmatrix}
1 & 2 & 1\\
2 & 5 & 2\\
1 & 2 & 10
\end{pmatrix}
\end{split}
\end{align*}
Da die Matrix $A$ symmetrisch und positiv definit ist besitzt sie eine Cholesly-Zerlegung $A = L\cdot L^*$\\
Wir berechnen also $L$ und $L^*$ mithilfe der Cholesky-Zerlegung\\
Dabei gelten folgende Formeln:\\
\begin{align}
i < k \Rightarrow l_{ik} &= 0\\
i = k \Rightarrow l_{kk} &= \sqrt{a_{kk}-\sum_{j=1}^{k-1}|l_{kj}|^2}\\
i > k \Rightarrow l_{ik} &= \frac{1}{l_{kk}}\cdot (a_{ik}-\sum_{j=1}^{k-1}l_{ij}\cdot \overline{l_{kj}})
\end{align}

\underline{Berechnung von $L$:}\\
Wir bestimmen mit Hilfe der Formeln (1) - (3)
\begin{align*} \label{eq8}
L=
\begin{split}
\begin{pmatrix}
l_{11} & 0 & 0\\
l_{21} & l_{22} & 0\\
l_{31} & l_{32} & l_{33}
\end{pmatrix}
\end{split}
\end{align*}
Alle Eintr�ge f�r die $i < k$ gilt wurden bereits auf die 0 gesetzt.\\

$l_{11}: i=k$ es gilt also Formel (2)\\
\begin{align*}
	l_{11} &= \sqrt{a_{kk}-\sum_{j=1}^{k-1}|l_{kj}|^2}
	= \sqrt{a_{11}}
	= \sqrt{1}
	= 1
\end{align*}

$l_{21}: i > k$ es gilt also Formel (3)\\
\begin{align*}
	l_{21} &= \frac{1}{l_{11}}(a_{21}-\sum_{j=1}^{1-1}l_{2j}\cdot \overline{l_{1j}})
	= \frac{1}{l_{11}}\cdot a_{21}
	= \frac{1}{1}\cdot 2
	=2
\end{align*}

$l_{31}: i > k$ es gilt also Formel (3)\\
\begin{align*}
	l_{31} &= \frac{1}{l_{11}}\cdot (a_{31}-\sum_{j=1}^{1-1}l_{3j}\cdot \overline{l_{1j}})
	= \frac{1}{l_{11}}\cdot a_{13}
	= \frac{1}{1}\cdot 1
	= 1
\end{align*}

$l_{22}: i=k$ es gilt also Formel (2)\\
\begin{align*}
	l_{22} &= \sqrt{a_{22}-\sum_{j=1}^{2-1}|l_{2j}|^2}
	= \sqrt{a_{kk}-l_{21}^2}
	= \sqrt{5-2^2}
	= 1
\end{align*}

$l_{32}: i > k$ es gilt also Formel (3)\\
\begin{align*}
	l_{32} &= \frac{1}{l_{22}}\cdot (a_{32}-\sum_{j=1}^{2-1}l_{3j}\cdot \overline{l_{2j}})
	= \frac{1}{l_{22}}\cdot a_{32} - l_{21} \cdot l_{31}
	= \frac{1}{1}\cdot 2 - 2 \cdot 1
	= 0
\end{align*}

$l_{33}: i=k$ es gilt also Formel (2)\\
\begin{align*}
	l_{33} &= \sqrt{a_{33}-\sum_{j=1}^{3-1}|l_{3j}|^2}
	= \sqrt{a_{33}-l_{31}^2-l_{32}^2}
	= \sqrt{10-1^2-0}
	= 3
\end{align*}

Daraus ergibt sich f�r $L$:
\begin{align*}
L=
\begin{split}
\begin{pmatrix}
 1 & 0 & 0\\
 2 & 1 & 0\\
 1 & 0 & 3
\end{pmatrix}
\end{split}
\end{align*}

Aus $L$ l�sst sich also auch $L^*$ ablesen mit:
\begin{align*}
L^*=
\begin{split}
\begin{pmatrix}
 1 & 2 & 1\\
 0 & 1 & 0\\
 0 & 0 & 3
\end{pmatrix}
\end{split}
\end{align*}

$A$ wurde somit mittels der Cholesky-Zerlegung faktorisiert und es gilt:
\begin{align*}
A = L \cdot L^* \Rightarrow
\begin{pmatrix}
1 & 2 & 1\\
2 & 5 & 2\\
1 & 2 & 10
\end{pmatrix}
=
\begin{pmatrix}
 1 & 0 & 0\\
 2 & 1 & 0\\
 1 & 0 & 3
\end{pmatrix}
\cdot
\begin{pmatrix}
 1 & 2 & 1\\
 0 & 1 & 0\\
 0 & 0 & 3
\end{pmatrix}
\end{align*}