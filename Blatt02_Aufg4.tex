\newpage
\subsection*{Aufgabe 4}
Gegeben:
\begin{align*} \label{eq8}
A=
\begin{split}
\begin{pmatrix}
\frac{1}{\sqrt2} & \frac{1}{\sqrt2} \\
-\frac{1}{\sqrt2} &\frac{1}{\sqrt2}
\end{pmatrix}
\end{split}
\end{align*}
Nach Vorlesung gilt: $cond_{||.||}(A)=||A||*||A^{-1}||$\\

Wir berechnen also $A^{-1}$ mithilfe des Gauss-Jordan-Verfahren und erhalten:
%hier weis ich nicht, wie ich den senkrechten Strich in der Matrix zwischen A und Einheitsmatrix hinbekomme
\begin{align*}
(A|I)=
\begin{split}
\begin{pmatrix}
\frac{1}{\sqrt2} & \frac{1}{\sqrt2} & |1 & 0 \\
-\frac{1}{\sqrt2} &\frac{1}{\sqrt2} & |0 & 1
\end{pmatrix}
\Rightarrow(I|A^{-1})=
\begin{pmatrix}
 1 & 0 & |\frac{\sqrt2}{2} & -\frac{\sqrt2}{2} \\
 0 & 1 & |\frac{\sqrt2}{2} & \frac{\sqrt2}{2}
\end{pmatrix}
\end{split}
\end{align*}
Daraus folgt:\\
Spaltensummennorm:
\begin{align*}
\begin{split}
&||A||_1=max\{|\frac{1}{\sqrt2}|+|-\frac{1}{\sqrt2}|, |\frac{1}{\sqrt2}|+|\frac{1}{\sqrt2}|\}=\frac{2}{\sqrt2}\\
&||A^{-1}||_1=max\{|\frac{\sqrt2}{2}|+|\frac{\sqrt2}{2}|, |-\frac{\sqrt2}{2}|+|\frac{\sqrt2}{2}|\}={\sqrt2}\\
& cond_{||.||_1}(A)=||A|_1*||A^{-1}||_1=\frac{2}{\sqrt2} * {\sqrt2} = 2\\
\end{split}
\end{align*}
euklidische Norm:
\begin{align*}
\begin{split}
&||A||_2=\sqrt{|\frac{1}{\sqrt2}|^2+|-\frac{1}{\sqrt2}|^2+|\frac{1}{\sqrt2}|^2+|\frac{1}{\sqrt2}|^2}=\sqrt{2}\\
&||A^{-1}||_2=\sqrt{|\frac{\sqrt2}{2}|^2+|\frac{\sqrt2}{2}|^2 + |-\frac{\sqrt2}{2}|^2 +|\frac{\sqrt2}{2}|^2}=\sqrt{2}\\
& cond_{||.||_2}(A)=||A|_2*||A^{-1}||_2=\sqrt{2}*\sqrt{2}=2\\
\end{split}
\end{align*}
Zeilensummennorm:
\begin{align*}
\begin{split}
&||A||_\infty =max\{|\frac{1}{\sqrt2}|+|\frac{1}{\sqrt2}|, |-\frac{1}{\sqrt2}|+|\frac{1}{\sqrt2}|\}=\frac{2}{\sqrt2}\\
&||A^{-1}||_\infty=max\{|\frac{\sqrt2}{2}|+|-\frac{\sqrt2}{2}|, |\frac{\sqrt2}{2}|+|\frac{\sqrt2}{2}|\}={\sqrt2}\\
& cond_{||.||_\infty}(A)=||A|_\infty*||A^{-1}||_\infty=\frac{2}{\sqrt2} * {\sqrt2} = 2\\
\end{split}
\end{align*}

%Berechnung für die Normen von B fehlen noch (arbeite ich gerade noch dran)
\begin{align*} 
B=
\begin{split}
\begin{pmatrix}
1 & 2 & 0 \\
0 & 1 & 0 \\
0 & 1 & 1
\end{pmatrix}
\end{split}
\end{align*}
