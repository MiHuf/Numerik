\newpage
\subsection*{Aufgabe 4}
Gegeben:
\begin{align*} \label{eq8}
A=
\begin{split}
\begin{pmatrix}
\frac{1}{\sqrt2} & \frac{1}{\sqrt2} \\
-\frac{1}{\sqrt2} &\frac{1}{\sqrt2}
\end{pmatrix}
\end{split}
\end{align*}
Nach Vorlesung gilt: \fbox{$\mathrm{\cond}(A)=\|A\|*\|A^{-1}\|$}\\

Wir berechnen also $A^{-1}$ mithilfe des Gauss-Jordan-Verfahren und erhalten:
%hier weis ich nicht, wie ich den senkrechten Strich in der Matrix zwischen A und Einheitsmatrix hinbekomme
\begin{align*}
(A|I)=
\begin{split}
\begin{array}({cc|cc})
\frac{1}{\sqrt2} & \frac{1}{\sqrt2} & 1 & 0 \\
-\frac{1}{\sqrt2} &\frac{1}{\sqrt2} & 0 & 1
\end{array}
\Rightarrow(I|A^{-1})=
\begin{array}({cc|cc})
 1 & 0 & \frac{\sqrt2}{2} & -\frac{\sqrt2}{2} \\
 0 & 1 & \frac{\sqrt2}{2} & \frac{\sqrt2}{2}
\end{array}
\end{split}
\end{align*}
Daraus folgt:\\
Spaltensummennorm:
\begin{align*}
\begin{split}
&\|A\|_1=\mathrm{max}\left\lbrace \left|\frac{1}{\sqrt2}\right|+\left|-\frac{1}{\sqrt2}\right|, \left|\frac{1}{\sqrt2}\right|+\left|\frac{1}{\sqrt2}\right|\right\rbrace=\frac{2}{\sqrt2}\\
&\|A^{-1}\|_1=\mathrm{max}\left\lbrace\left|\frac{\sqrt2}{2}\right|+\left|\frac{\sqrt2}{2}\right|, \left|-\frac{\sqrt2}{2}\right|+\left|\frac{\sqrt2}{2}\right|\right\rbrace={\sqrt2}\\
& \conda(A)=\|A|_1*\|A^{-1}\|_1=\frac{2}{\sqrt2} * {\sqrt2} = 2\\
\end{split}
\end{align*}
euklidische Norm:
\begin{align*}
\begin{split}
&\|A\|_2=\sqrt{ \left(\frac{1}{\sqrt2}\right)^2+\left(-\frac{1}{\sqrt2}\right)^2+\left(\frac{1}{\sqrt2}\right)^2+\left(\frac{1}{\sqrt2}\right)^2}=\sqrt{2}\\
&\|A^{-1}\|_2=\sqrt{\left(\frac{\sqrt2}{2}\right)^2+\left(\frac{\sqrt2}{2}\right)^2 + \left(-\frac{\sqrt2}{2}\right)^2 + \left(\frac{\sqrt2}{2}\right)^2}=\sqrt{2}\\
& \condb(A)=\|A|_2*\|A^{-1}\|_2=\sqrt{2}*\sqrt{2}=2\\
\end{split}
\end{align*}
Zeilensummennorm:
\begin{align*}
\begin{split}
&\|A\|_\infty =\mathrm{max}\left\lbrace\left|\frac{1}{\sqrt2}\right|+\left|\frac{1}{\sqrt2}\right|, \left|-\frac{1}{\sqrt2}\right|+\left|\frac{1}{\sqrt2}\right|\right\rbrace=\frac{2}{\sqrt2}\\
&\|A^{-1}\|_\infty=\mathrm{max}\left\lbrace\left|\frac{\sqrt2}{2}\right|+\left|-\frac{\sqrt2}{2}\right|, \left|\frac{\sqrt2}{2}\right|+\left|\frac{\sqrt2}{2}\right|\right\rbrace={\sqrt2}\\
& \condi(A)=\|A|_\infty*\|A^{-1}\|_\infty=\frac{2}{\sqrt2} * {\sqrt2} = 2\\
\end{split}
\end{align*}
\newline


Gegeben:\\
\begin{align*} 
B=
\begin{split}
\begin{pmatrix}
1 & 2 & 0 \\
0 & 1 & 0 \\
0 & 1 & 1
\end{pmatrix}
\end{split}
\end{align*}
Analog zu oben berechnen wir $B^{-1}$:\\
\begin{align*}
(B|I)=
\begin{split}
\begin{array}({ccc|ccc})
1 & 2 & 0 & 1 & 0 & 0\\
0 & 1 & 0 & 0 & 1 & 0\\
0 & 1 & 1 & 0 & 0 & 1
\end{array}
\Rightarrow(I|B^{-1})=
\begin{array}({ccc|ccc})
1 & 0 & 0 &1 & -2 & 0 \\
0 & 1 & 0 &0 & 1 & 0 \\
0 & 0 & 1 &0 & -1 & 1  
\end{array}
\end{split}
\end{align*}
Daraus folgt:\\
Spaltensummennorm:
\begin{align*}
\begin{split}
&\|B\|_1=\mathrm{max} \left\lbrace|1|+|0|+|0|, |2|+|1|+|1|, |0|+|0|+|1|\right\rbrace = 4\\
&\|B^{-1}\|_1=\mathrm{max} \left\lbrace|1|+|0|+|0|, |-2|+|1|+|-1|, |0|+|0|+|1|\right\rbrace = 4\\
&\conda(B)=\|B|_1*\|B^{-1}\|_1= 4 * 4 = 16\\
\end{split}
\end{align*}
euklidische Norm:
\begin{align*}
\begin{split}
&\|B\|_2=\sqrt{1^2+0^2+0^2+2^2+1^2+1^2+0^2+0^2+1^2}=\sqrt{8}\\
&\|B^{-1}\|_2=\sqrt{1^2+0^2+0^2+(-2)^2+1^2+(-1)^2+0^2+0^2+1^2}=\sqrt{8}\\
&\condb(B)=\|B|_2*\|B^{-1}\|_2=\sqrt{8}*\sqrt{8}=8\\
\end{split}
\end{align*}
Zeilensummennorm:
\begin{align*}
\begin{split}
&\|B\|_\infty = \mathrm{max} \left\lbrace|1|+|2|+|0|, |0|+|1|+|0|,|0|+|1|+|1|\right\rbrace = 3\\
&\|B^{-1}\|_\infty = \mathrm{max} \left\lbrace|1|+|-2|+|0|, |0|+|1|+|0|,|0|+|-1|+|1|\right\rbrace = 3\\
&\condi(B)=\|B|_\infty*\|B^{-1}\|_\infty = 3 * 3 = 9\\
\end{split}
\end{align*}