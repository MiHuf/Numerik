\newpage
\subsection*{Aufgabe 4}
Gegeben:
\begin{align*} \label{eq8}
A=
\begin{split}
\begin{pmatrix}
\frac{1}{\sqrt2} & \frac{1}{\sqrt2} \\
-\frac{1}{\sqrt2} &\frac{1}{\sqrt2}
\end{pmatrix}
\end{split}
\end{align*}
Nach Vorlesung gilt: \fbox{$\mathrm{\cond}(A)=\|A\|\cdot\|A^{-1}\|$}\\

Wir berechnen also $A^{-1}$ mithilfe des Gauss-Jordan-Verfahren und erhalten:
\begin{align*}
(A|I)=
\begin{split}
\begin{array}({cc|cc})
\frac{1}{\sqrt2} & \frac{1}{\sqrt2} & 1 & 0 \\
-\frac{1}{\sqrt2} &\frac{1}{\sqrt2} & 0 & 1
\end{array}
\Rightarrow(I|A^{-1})=
\begin{array}({cc|cc})
 1 & 0 & \frac{\sqrt2}{2} & -\frac{\sqrt2}{2} \\
 0 & 1 & \frac{\sqrt2}{2} & \frac{\sqrt2}{2}
\end{array}
\end{split}
\end{align*}
Daraus folgt:\\
\underline{Spaltensummennorm:}
\begin{align*}
\begin{split}
&\|A\|_1=\mathrm{max}\left\lbrace \left|\frac{1}{\sqrt2}\right|+\left|-\frac{1}{\sqrt2}\right|, \left|\frac{1}{\sqrt2}\right|+\left|\frac{1}{\sqrt2}\right|\right\rbrace=\frac{2}{\sqrt2}\\
&\|A^{-1}\|_1=\mathrm{max}\left\lbrace\left|\frac{\sqrt2}{2}\right|+\left|\frac{\sqrt2}{2}\right|, \left|-\frac{\sqrt2}{2}\right|+\left|\frac{\sqrt2}{2}\right|\right\rbrace={\sqrt2}\\
& \conda(A)=\|A\|_1\cdot\|A^{-1}\|_1=\frac{2}{\sqrt2} * {\sqrt2} = 2\\
\end{split}
\end{align*}
\underline{Spektralnorm:}\\
Wir bestimmen $A^T\cdot A$:\\
\begin{align*}
\begin{split}
A^T =
\begin{pmatrix}
\frac{1}{\sqrt{2}}&-\frac{1}{\sqrt{2}}\\
\frac{1}{\sqrt{2}}&\frac{1}{\sqrt{2}}
\end{pmatrix}
\Rightarrow A^T\cdot A=
\begin{pmatrix}
1&0\\
0&1
\end{pmatrix}\\
\end{split}
\end{align*}\\
Eigenwerte von $A^T\cdot A$ bestimmen:\\
\begin{align*}
&\mathrm{det}\left((A^T\cdot A)-\lambda I\right)=(1-\lambda)^2
\Rightarrow \lambda_{1,2}=1\\
&\|A\|_2=\sqrt{\lambda_{max}(A^T\cdot A)}=\sqrt{1}=1
\end{align*}
Weiterhin bestimmen wir $(A^{-1})^T\cdot(A^{-1})$:\\
\begin{align*}
\begin{split}
(A^{-1})^T =
\begin{pmatrix}
\frac{\sqrt{2}}{2}&\frac{\sqrt{2}}{2}\\
-\frac{\sqrt{2}}{2}&\frac{\sqrt{2}}{2}
\end{pmatrix}
\Rightarrow (A^{-1})^T\cdot(A^{-1})=
\begin{pmatrix}
1&0\\
0&1
\end{pmatrix}\\
\end{split}
\end{align*}\\
Eigenwerte von $(A^{-1})^T\cdot(A^{-1})$ bestimmen:\\
\begin{align*}
&\mathrm{det}\left((A^{-1})^T\cdot(A^{-1})-\lambda I\right)=(1-\lambda)^2
\Rightarrow \lambda_{1,2}=1\\
&\|A^{-1}\|_2=\sqrt{\lambda_{max}((A^{-1})^T\cdot A^{-1})}=\sqrt{1}=1\\
&\Rightarrow\condb(A)=\|A\|_2\cdot\|A^{-1}\|_2=1 \cdot 1=1\\
\end{align*}

\underline{Zeilensummennorm:}
\begin{align*}
\begin{split}
&\|A\|_\infty =\mathrm{max}\left\lbrace\left|\frac{1}{\sqrt2}\right|+\left|\frac{1}{\sqrt2}\right|, \left|-\frac{1}{\sqrt2}\right|+\left|\frac{1}{\sqrt2}\right|\right\rbrace=\frac{2}{\sqrt2}\\
&\|A^{-1}\|_\infty=\mathrm{max}\left\lbrace\left|\frac{\sqrt2}{2}\right|+\left|-\frac{\sqrt2}{2}\right|, \left|\frac{\sqrt2}{2}\right|+\left|\frac{\sqrt2}{2}\right|\right\rbrace={\sqrt2}\\
& \condi(A)=\|A\|_\infty*\|A^{-1}\|_\infty=\frac{2}{\sqrt2} \cdot {\sqrt2} = 2\\
\end{split}
\end{align*}
\newline
Gegeben:\\
\begin{align*} 
B=
\begin{split}
\begin{pmatrix}
1 & 2 & 0 \\
0 & 1 & 0 \\
0 & 1 & 1
\end{pmatrix}
\end{split}
\end{align*}
Analog zu oben berechnen wir $B^{-1}$:\\
\begin{align*}
(B|I)=
\begin{split}
\begin{array}({ccc|ccc})
1 & 2 & 0 & 1 & 0 & 0\\
0 & 1 & 0 & 0 & 1 & 0\\
0 & 1 & 1 & 0 & 0 & 1
\end{array}
\Rightarrow(I|B^{-1})=
\begin{array}({ccc|ccc})
1 & 0 & 0 &1 & -2 & 0 \\
0 & 1 & 0 &0 & 1 & 0 \\
0 & 0 & 1 &0 & -1 & 1  
\end{array}
\end{split}
\end{align*}
Daraus folgt:\\
\underline{Spaltensummennorm:}
\begin{align*}
\begin{split}
&\|B\|_1=\mathrm{max} \left\lbrace|1|+|0|+|0|, |2|+|1|+|1|, |0|+|0|+|1|\right\rbrace = 4\\
&\|B^{-1}\|_1=\mathrm{max} \left\lbrace|1|+|0|+|0|, |-2|+|1|+|-1|, |0|+|0|+|1|\right\rbrace = 4\\
&\conda(B)=\|B\|_1\cdot\|B^{-1}\|_1= 4 \cdot 4 = 16\\
\end{split}
\end{align*}
\underline{Spektralnorm:}\\
Wir bestimmen $B^T\cdot B$:\\
\begin{align*}
\begin{split}
B^T =
\begin{pmatrix}
1 & 0 & 0\\
2 & 1 & 1\\
0 & 0 & 1
\end{pmatrix}
\Rightarrow B^T\cdot B=
\begin{pmatrix}
1 & 2 & 0\\
2 & 6 & 1\\
0 & 1 & 1
\end{pmatrix}\\
\end{split}
\end{align*}\\
Eigenwerte von $B^T\cdot B$ bestimmen:\\
\begin{align*}
&\mathrm{det}\left((B^T\cdot B)-\lambda I\right)=-\lambda^3+8\cdot\lambda^2-8\cdot\lambda+1\\
&-\lambda^3+8\cdot\lambda^2-8\cdot\lambda+1=0\\
&\Rightarrow \lambda_{1}=\frac{1}{2}(7+3\sqrt{5})\\
&\Rightarrow\lambda_{2}=1\\
&\Rightarrow\lambda_{3}=\frac{1}{2}(7-3\sqrt{5})\\
&\|B\|_2=\sqrt{\lambda_{max}(B^T\cdot B)}=\sqrt{\frac{1}{2}(7+3\sqrt{5})}
\end{align*}
Weiterhin bestimmen wir $(B^{-1})^T\cdot(B^{-1})$:\\
\begin{align*}
\begin{split}
(B^{-1})^T =
\begin{pmatrix}
1 & 0 & 0\\
-2 & 1 & -1\\
0 & -1 & 1
\end{pmatrix}
\Rightarrow (B^{-1})^T\cdot(B^{-1})=
\begin{pmatrix}
1 & -2 & 0\\
-2 & 6 & -1\\
0 & -1 & 1
\end{pmatrix}\\
\end{split}
\end{align*}\\
Eigenwerte von $(B^{-1})^T\cdot(B^{-1})$ bestimmen:\\
\begin{align*}
\begin{split}
&\mathrm{det}\left((B^{-1})^T\cdot(B^{-1})-\lambda I \right)=-\lambda^3+8\cdot\lambda^2-8\cdot\lambda+1\\
&-\lambda^3+8\cdot\lambda^2-8\cdot\lambda+1=0\\
&\Rightarrow \lambda_{1}=\frac{1}{2}(7+3\sqrt{5})\\
&\Rightarrow\lambda_{2}=1\\
&\Rightarrow\lambda_{3}=\frac{1}{2}(7-3\sqrt{5})\\
&\|B^{-1}\|_2=\sqrt{\lambda_{max}((B^{-1})^T\cdot B^{-1})}=\sqrt{\frac{1}{2}(7+3\sqrt{5})}\\
&\condb(B)=\|B\|_2\cdot\|B^{-1}\|_2=\sqrt{\frac{1}{2}(7+3\sqrt{5})}\cdot\sqrt{\frac{1}{2}(7+3\sqrt{5})}=\frac{1}{2}(7+3\sqrt{5})
\end{split}
\end{align*}
\underline{Zeilensummennorm:}
\begin{align*}
\begin{split}
&\|B\|_\infty = \mathrm{max} \left\lbrace|1|+|2|+|0|, |0|+|1|+|0|,|0|+|1|+|1|\right\rbrace = 3\\
&\|B^{-1}\|_\infty = \mathrm{max} \left\lbrace|1|+|-2|+|0|, |0|+|1|+|0|,|0|+|-1|+|1|\right\rbrace = 3\\
&\condi(B)=\|B\|_\infty\cdot\|B^{-1}\|_\infty = 3 * 3 = 9\\
\end{split}
\end{align*}