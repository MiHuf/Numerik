\subsection*{Aufgabe 27}
Zu zeigen: Unter allen trigonometrischen Polynomen $q_s \in \mathcal{T}_s$ der Form
\[
	q_s= \sum_{k=0}^{s}d_ke^{ikx}
\]
gerade $p_s$ die Fehlerquadratsumme
\begin{align*}
	S(q):=\sum_{k=0}^{n-1}|f_k - q_s(x_k)|^2
\end{align*}
minimiert.\\ \newline
Beweis: Führt man auf dem n-dimentionalen komplexen Vektorraum $\CC^n$ aller komplexer Vektoren\\
 $u=[u_0 \quad u_1 \quad \cdots \quad u_{n-1}]$ der Länge $n$ das übliche Skalarprodukt
\[
	[u,v] := \sum_{j=0}^{n-1}u_j\bar{v}_j
\]
ein, so gilt, dass die speziellen Vektoren
\[
	\omega {(j)}:=[\omega_0^{(j)} \quad \omega_1^{(j)} \quad \cdots \quad \omega_{n-1}^{(j)}], \qquad j=0,1,\dots,n-1
\]
eine Orthogonalbasis des $\CC^n$ bilden,
\[
	[\omega^{(j)},\omega^{(h)}] = \left\{\begin{array}{ll} n, & j = h \\
	0, & j\neq h\end{array}\right.
\]

Hier fehlt noch was!!!\\ \newline

Mit den n-Vektoren
\begin{align*}
	p_s := [p_s(x_0)\quad \cdots \quad p_s(x_{n-1})], \qquad q:=[q(x_0) \quad \cdots \quad q(x_{n-1})]
\end{align*}
gilt
\begin{align*}
	S(q)=[f-q,f-q].
\end{align*}
Es gilt wegen $p(x_k)=f_k$ für $k=0, \dots ,n-1$, $[f,w^{(k)}]=nc_k$ für $k=0, \dots ,n-1$ und daher
\begin{align*}
	[f-p_s,w^{(k)}] = [f-\sum_{h=0}^{s}c_kw^{(h)}, w^{(j)}]=nc_j-nc_k=0,\qquad j=0, \dots ,s,
\end{align*}
sowie
\begin{align*}
	[f-p_s,p_s-q] = \sum_{j=0}^{s}[f-p_s,(c_j-d_j)w^{(j)}]=0.
\end{align*}
Damit gilt aber
\begin{align*}
	S(q) &= [f-q,f-q]\\
	&= [(f-p_s)+(p_s-q),(f-p_s)+(p_s-q)]\\
	&= [f-p_s,f-p_s]+[p_s-q,p_s-q]\\
	&\geq [f-p_s,f-p_s] = S(p_s)
\end{align*}
mit Gleichheit nur für $[p_s-q,p_s,q]=0$, d.h. $q(x_j)=p_s(x_j)$ für $0 \leq j \leq n-1$. Durch die Eindeutigkeit der trigonometrischen Polynome folgt damit $p_s=q$.