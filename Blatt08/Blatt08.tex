%  DOCUMENT CLASS
\documentclass[11pt]{article}

%PACKAGES
\usepackage[utf8]{inputenc}
\usepackage[ngerman]{babel}
\usepackage[reqno,fleqn]{amsmath}
\setlength\mathindent{10mm}
\usepackage{amssymb}
\usepackage{amsthm}
\usepackage{color}
\usepackage{delarray}
% \usepackage{fancyhdr}
\usepackage{units}
\usepackage{times, eurosym}
\usepackage{verbatim} %Für Verwendung von multiline Comments mittels \begin{comment}...\end{comment}
\usepackage{wasysym} % Für Smileys


% FORMATIERUNG
\usepackage[paper=a4paper,left=25mm,right=25mm,top=25mm,bottom=25mm]{geometry}
\usepackage{array}
\usepackage{fancybox} %zum Einrahmen von Formeln
\setlength{\parindent}{0cm}
\setlength{\parskip}{1mm plus1mm minus1mm}


% PAGESTYLE

%MATH SHORTCUTS
\newcommand{\NN}{\mathbb N}
\newcommand{\ZZ}{\mathbb Z}
\newcommand{\QQ}{\mathbb Q}
\newcommand{\RR}{\mathbb R}
\newcommand{\CC}{\mathbb C}
\newcommand{\KK}{\mathbb K}
\newcommand{\U}{\mathbb O}
\newcommand{\eqx}{\overset{!}{=}}
\newcommand{\Det}{\mathrm{Det}}
\newcommand{\Gl}{\mathrm{Gl}}
\newcommand{\diag}{\mathrm{diag}}
\newcommand{\sign}{\mathrm{sign}}
\newcommand{\rang}{\mathrm{rang}}
\newcommand{\cond}{\mathrm{cond}_{\| \cdot \|}}
\newcommand{\conda}{\mathrm{cond}_{\| \cdot \|_1}}
\newcommand{\condb}{\mathrm{cond}_{\| \cdot \|_2}}
\newcommand{\condi}{\mathrm{cond}_{\| \cdot \|_\infty}}
\newcommand{\eps}{\epsilon}

\setlength{\extrarowheight}{1ex}

\begin{document}

\begin{center}
\textbf{
Übungen zur Vorlesung Numerische Mathematik, WS 2014/15\\
Blatt 08 zum 17.12.2014\\
}

\begin{tabular}{lll}
& \\
von & Janina Geiser & Mat Nr. 6420269\\
& Michael Hufschmidt & Mat.Nr. 6436122\\
& Farina Ohm & Mat Nr. 6314051\\
& Annika Seidel & Mat Nr. 6420536\\
\\
\hline
\end{tabular}
\end{center}

\subsection*{Aufgabe 25}



\subsection*{Aufgabe 26}
Zu zeigen: Zu den $2 m + 1 $ paarweise verschiedenen Stützstellen $x_0 , \cdots , x_{2m} \in [0, 2\pi)$ mit
Stützwerten $f_0 , \cdots , f_{2m} \in \RR$ existiert genau ein reelles trigonometrisches
Polynom
\begin{align}
  q \in \left\lbrace \left. \frac{a_0}{2} + \sum_{j = 1}^m [a_j \cos(j x) + b_j \sin(j x)] \;
    \right| \; a_0 , \cdots , a_m , b_1 , \cdots , b_m \in \RR\  \right\rbrace
\end{align}
welches $q_k(x_k) = f_k$ für $k = 0 , \cdots , 2m$ erfüllt.

Beweis: In der Vorlesung wurde gezeigt, dass es für diese Stützstellen genau ein
komplexes trigonometrisches Polynom
\begin{align}
\label{eq4}
  p(x) = \sum_{k = 0}^{2m} c_k e^{i k x}\quad \text{mit eindeutig definierten }c_k \in \CC
\end{align}
gibt, das ebenfalls $p_k(x_k) = f_k$ \footnote{Die Formel auf dem Aufgabeblatt ist fehlerhaft}
für $k = 0 , \cdots , 2m$ erfüllt. Wir betrachten nun
\begin{align}
\label{eq5}
  \tilde{p}(x) &:= \overline{p(x)} \cdot e^{2 i m x } =
    \sum_{k = 0}^{2m} \overline{c_k} e^{- i k x} e^{2 i m x } =
    \sum_{k = 0}^{2m} \overline{c_k} e^{ i (2m - k) x }
\end{align}
Da \eqref{eq4} ein eindeutiges Stützstellenpolynom definiert und \eqref{eq5} die
gleichen orthogonalen Basisfunktionen verwendet, müssen die Koeffizienten
(bis auf die umgekehrte Reihenfolge) in beiden Formeln identisch sein.

Daraus folgt $c_k = \overline{c_{2 m -k}}$ für $k = 0 , \cdots , 2m$ oder
$c_{m + j} = \overline{c_{m - j}}$ für $j = 0 , \cdots , m$. Insbesondere ist
für $j = 0: \quad c_{m + 0} = \overline{c_{m - 0}} = c_m$ reell. Der Zusammenhang mit
den reellen Koeffizienten $a_j, b_j$ ist dann:
\begin{align}
\label{eq6}
\boxed{
\left.
\begin{array}{l}
a_0 = 2 \cdot c_m  \; \in \RR\\
a_j = c_{m + j} + c_{m - j} =  2 \cdot \Re(c_{m - j}) \; \in \RR\\
b_j = i \cdot (c_{m + j} - c_{m - j}) = 2 \cdot \Im(c_{m - j}) \; \in \RR\\
\end{array}
\right\rbrace \quad j = 1 , \cdots , m
}
\end{align}
Gleichung \eqref{eq6} ist äquivalent zu
\begin{align}
\label{eq7}
\left.
\begin{array}{l}
c_m = \frac{1}{2} a_0\\
c_{m + j} = \frac{1}{2} (a_j - i b_j) \; \in \CC\\
c_{m - j} = \frac{1}{2} (a_j + i b_j) \; \in \CC\\
\end{array}
\right\rbrace \quad j = 1 , \cdots , m
\end{align}
Um \eqref{eq6} bzw. \eqref{eq7} zu verifizieren, untersuchen wir
das folgende Polynom unter Berücksichtigung von \eqref{eq4}. Das hat die gleichen
Basisfunktionen, und erfüllt somit die Interpolsationsanforderungen.
\begin{align*}
p(x) \cdot e^{-i m x} & = \sum_{k = 0}^{2m} c_k e^{i (k - m) x}\\
  & =  \sum_{k = 0}^{m - 1} c_k e^{i (k - m) x} + c_m e^{i m x} e^{- i m x} +
  \sum_{k = m + 1}^{2 m} c_k e^{i (k - m) x}
\intertext{Nach einer Umbenennung $j = m - k$ für die erste und $j = k - m$
für die zweite der beiden letzen Summen:}
p(x) \cdot e^{-i m x} & = c_m +
\sum_{j = m}^{1} c_{m - j} e^{- i j x} + \sum_{j = 1}^{m} c_{m + j} e^{i j x}
 = c_m +\sum_{j = 1}^{m} \left(c_{m + j}  e^{i j x} + c_{m - j} e^{-i j x}  \right)
\intertext{Einsetzen von \eqref{eq7} und unter Berücksichtigung von $e^{ix} = \cos(x) + i \sin(x)$}
p(x) \cdot e^{-i m x} & = \frac{a_0}{2} + \sum_{j = 1}^{m}
  \frac{1}{2} \left[ \vphantom{\frac{1}{2}} (a_j - i b_j)(\cos(jx) + i \sin(jx)) + (a_j + i b_j) (\cos(jx) - i \sin(jx)) \right]\\
\intertext{Die gemischten Terme heben sich auf, somit:}
p(x) \cdot e^{-i m x} & = \frac{a_0}{2} + \sum_{j = 1}^m \left[a_j \cos(j x) + b_j \sin(j x)\right] = q(x)
\end{align*}


Wir haben somit durch umkehrbar eindeutige Operationen die reellen Koeffizienten
$a_k, b_k$ aus den komplexen Koeffizienten $c_k$ ermittelt. Daraus folgt, dass auch
das reelle Stützstellenpolynom $q$ existiert und eindeutig ist.









\subsection*{Aufgabe 27}
Zu zeigen: Unter allen trigonometrischen Polynomen $q_s \in \mathcal{T}_s$ der Form
\[
	q_s= \sum_{k=0}^{s}d_ke^{ikx}
\]
mimimiert gerade $p_s$ die Fehlerquadratsumme
\begin{align} \label{eq:sq}
	S(q):=\sum_{k=0}^{n-1}|f_k - q_s(x_k)|^2
\end{align} \newline
Beweis: Führt man auf dem n-dimentionalen komplexen Vektorraum $\CC^n$ aller komplexer Vektoren\\
 $u=[u_0 \quad u_1 \quad \cdots \quad u_{n-1}]$ der Länge $n$ das Standardskalarprodukt
\[
	[u,v] := \sum_{j=0}^{n-1}u_j\bar{v}_j
\]
ein, so gilt, dass die Vektoren
\[
	\omega {(j)}:=[\omega_0^{(j)} \quad \omega_1^{(j)} \quad \cdots \quad \omega_{n-1}^{(j)}], \qquad j=0,1,\dots,n-1
\]

mit $\omega_k = e^{ix_k}=e^{i\frac{2\pi k}{N}}$
eine Orthogonalbasis des $\CC^n$ bilden, da:
\[
	[\omega^{(j)},\omega^{(h)}] = \left\{\begin{array}{ll} n, & j = h \\
	0, & j\neq h\end{array}\right.
\]
Wir definieren:
\begin{align*}
	p_s := [p_s(x_0)\quad \cdots \quad p_s(x_{n-1})], \qquad q:=[q(x_0) \quad \cdots \quad q(x_{n-1})]
\end{align*}
Mit Hilfe der vom Standardskalarprodukt induzierten Norm lässt sich \eqref{eq:sq} umschreiben zu:
\begin{align*}
S(q)=\sum_{k=0}^{n-1}|f_k - q_s(x_k)|^2 = [f-q,f-q]
\end{align*}
Es gilt wegen $p(x_k)=f_k$ für $k=0, \dots ,n-1$, $[f,w^{(k)}]=nc_k$ für $k=0, \dots ,n-1$. \\Daher folgt für $k=0, \dots ,s,$:
\begin{align*}
	[f-p_s,w^{(k)}] = [f-\sum_{h=0}^{s}c_kw^{(h)}, w^{(k)}]=[f, w^{(k)}]-[\sum_{h=0}^{s}c_kw^{(h)}, w^{(k)}]=nc_k-nc_k=0
\end{align*}
sowie
\begin{align*}
	[f-p_s,p_s-q] = \sum_{j=0}^{s}[f-p_s,(c_j-d_j)w^{(j)}]=0.
\end{align*}
Damit gilt aber
\begin{align*}
	S(q) &= [f-q,f-q]\\
	&= [(f-p_s)+(p_s-q),(f-p_s)+(p_s-q)]\\
	&= [f-p_s,f-p_s]+[p_s-q,p_s-q]\\
	&\geq [f-p_s,f-p_s] = S(p_s)
\end{align*}
mit Gleichheit nur für $[p_s-q,p_s,q]=0$, d.h. $q(x_j)=p_s(x_j)$ für $0 \leq j \leq n-1$. Durch die Eindeutigkeit der trigonometrischen Polynome folgt damit $p_s=q$.\\
\newline
Unter allen trigonometrischen Polynomen $q_s \in \mathcal{T}_s$ mimimiert also gerade $p_s$ die Fehlerquadratsumme.
\begin{flushright}
Q.e.d
\end{flushright}

\end{document}
