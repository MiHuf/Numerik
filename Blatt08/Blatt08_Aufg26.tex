\subsection*{Aufgabe 26}
Zu zeigen: Zu den $2 m + 1 $ paarweise verschiedenen Stützstellen $x_0 , \cdots , x_{2m} \in [0, 2\pi)$ mit
Stützwerten $f_0 , \cdots , f_{2m} \in \RR$ existiert genau ein reelles trigonometrisches
Polynom
\begin{align}
  q \in \left\lbrace \left. \frac{a_0}{2} + \sum_{j = 1}^m [a_j \cos(j x) + b_j \sin(j x)] \;
    \right| \; a_0 , \cdots , a_m , b_1 , \cdots , b_m \in \RR\  \right\rbrace
\end{align}
welches $q_k(x_k) = f_k$ für $k = 0 , \cdots , 2m$ erfüllt.

Beweis: In der Vorlesung wurde gezeigt, dass es für diese Stützstellen genau ein
komplexes trigonometrisches Polynom
\begin{align}
\label{eq4}
  p(x) = \sum_{k = 0}^{2m} c_k e^{i k x}\quad \text{mit eindeutig definierten }c_k \in \CC
\end{align}
gibt, das ebenfalls $p_k(x_k) = f_k$ \footnote{Die Formel auf dem Aufgabeblatt ist fehlerhaft}
für $k = 0 , \cdots , 2m$ erfüllt. Wir betrachten nun
\begin{align}
\nonumber
  \tilde{p}(x) &:= \overline{p(x)} \cdot e^{2 i m x } =
    \sum_{k = 0}^{2m} \overline{c_k} e^{- i k x} e^{2 i m x } =
    \sum_{k = 0}^{2m} \overline{c_k} e^{ i (2m - k) x } \\
\intertext{und mit einer Umbenennung $j := 2 m - k$}
\label{eq5}
  \tilde{p}(x) &= \sum_{j = 2m}^0 \overline{c_j} e^{ i j x }
\end{align}
Da \eqref{eq4} ein eindeutiges Stützstellenpolynom definiert und \eqref{eq5} die gleichen
orthogonalen Basisfunktionen verwendet, müssen die Koeffizienten (bis auf die Reihenfolge)
in beiden Formeln identisch sein.

Daraus folgt $c_k = \overline{c_{2 m -k}}$ für $k = 0 , \cdots , 2m$ oder
$c_{m + k} = \overline{c_{m - k}}$ für $k = 0 , \cdots , m$. Insbesondere ist
$c_m = c_{m + 0} = \overline{c_{m - 0}}$ reell. Der Zusammenhang mit
den reellen Koeffizienten $a_k, b_k$ ist dann:
\begin{align}
\label{eq6}
\boxed{
\left.
\begin{array}{l}
a_0 = 2 \cdot c_m  \; \in \RR\\
a_k = c_{m + k} + c_{m - k} =  2 \cdot \Re(c_{m - k}) \; \in \RR\\
b_k = i \cdot (c_{m + k} - c_{m - k}) = 2 \cdot \Im(c_{m - k}) \; \in \RR\\
\end{array}
\right\rbrace \quad k = 1 , \cdots , m
}
\end{align}
Um das zu zeigen, untersuchen wir das folgende Polynom unter Berücksichtigung
von \eqref{eq4} und \eqref{eq6}:
\begin{align}
% \nonumber
  p(x) \cdot e^{-i m x} & = \sum_{k = 0}^{2m} c_k e^{i (k - m) x} =
  \sum_{k = 0}^{m - 1} c_k e^{i (k - m) x} + c_m e^{i m x} e^{- i m x} +
  \sum_{k = m + 1}^{2 m} c_k e^{i (k - m) x}\\
\intertext{nach einer Umbenennung $j' = k - m$ und $j = m - k$ für die beiden letzen Summen:}
p(x) \cdot e^{-i m x} & = c_m +
\sum_{j' = 1}^{m} c_{m - k} e^{i j' x} + \sum_{j = 1}^{m} c_{m - k} e^{- i j x} % \\
= c_m +\sum_{j = 1}^{m} c_{m - k} \left( e^{i j x} + e^{- i j x}  \right)
\intertext{und unter Berücksichtigung von \eqref{eq6} und  $e^{ix} = \cos(x) + i \sin(x)$}
p(x) \cdot e^{-i m x} & = \frac{a_0}{2} + \sum_{j = 1}^m \left[a_j \cos(j x) + b_j \sin(j x)\right] = q(x)
\end{align}


Wir haben somit durch umkehrbar eindeutige Operationen die reellen Koeffizienten
$a_k, b_k$ aus den komplexen Koeffizienten $c_k$ ermittelt. Daraus folgt, dass auch
das reelle Stützstellenpolynom $q$ existiert und eindeutig ist.







