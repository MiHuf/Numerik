\subsection*{Aufgabe 8}
F�r $x \in \KK^m \; , \; x_1 \ne 0$ sei
$v = \frac{1}{\|x\|_2} + \sign(x_1) \cdot e_1$ und
$Q_v$  die Householder-Speigelung
\footnote{War zwar nicht auf dem �bungsblatt angegeben, aber wir nehmen das mal so an.}
$Q_v = I - \frac{2}{v^T \cdot v} \cdot v \cdot v^T$.

\paragraph*{a)}
Zu zeigen: $\|v\|_2 = \sqrt{ 2 \cdot \left( 1 + \frac{|x|}{\|x\|_2} \right) }$. Es gilt:
\begin{align*}
  \|v\|_2 & = \sqrt{v^T \cdot v} =
  \sqrt{\left( \frac{x^T}{\|x\|_2} + \sign(x_1) \cdot e_1^T \right) \cdot \left( \frac{x}{\|x\|_2} + \sign(x_1) \cdot e_1 \right) } \\
  & = \left( \frac{x^T \cdot x }{\|x\|_2^2} + \sign(x_1) \cdot e_1^T \cdot \frac{x}{\|x\|_2} +
   \frac{x^T}{\|x\|_2} \cdot \sign(x_1) \cdot e_1  +  \sign(x_1) \cdot e_1^T \cdot \sign(x_1) \cdot e_1 \right)^{1/2}\\
  \intertext{Mit $x^T \cdot x = \| x \|_2^2 $  und $e_1^T \cdot x = x \cdot e_1 = x_1$ und
    $\sign(x_1) \cdot x_1 = |x_1|$ ergibt das:}
 \|v\|_2 & = \sqrt {1 +  \frac{|x_1|}{\|x\|_2} + \frac{|x_1|}{\|x\|_2} +
  \underbrace{\sign(x_1) \cdot \sign(x_1) }_{ = +1} \cdot \underbrace{e_1^T \cdot e_1}_{ = 1} } =
   \sqrt{ 2 \cdot \left( 1 + \frac{|x|}{\|x\|_2} \right) }
\end{align*}

\paragraph*{b)}
Michael denkt nach
