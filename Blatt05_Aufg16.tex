\subsection*{Aufgabe 16}

Es seien $A,M \in \mathbf{Gl}_n(\RR), N \in \RR^{n\times n}$ so dass $A=M-N$, sowie $b \in \RR^n$. Weiterhin sei $T=M^{-1}N$ mit $\rho(T)\ge 1$ und $c=M^{-1}b$.\\
zu zeigen: Es existiert ein Startwert $x^{(0)} \in \RR^n$, so dass die rekursiv definierte Folge $(x^{(k)})$ mit $x^{(k+1)}=Tx^{(k)}+c$ nicht gegen $x=A^{-1}b$ konvergiert.\\
Beweis:\\
Wir transformieren $y := x - A^{-1} b$, dann gilt $A y = 0 \; \Leftrightarrow \; A x = b$.
Sei $\rho(T)\ge 1$, dann gibt es einen Eigenvektor $v \neq 0$ von $T$ zum Eigenwert $\lambda$ mit $|\lambda|\ge 1$. Wählen wir nun $x^{(0)} = v-A^{-1}b$ als Startwert, dann gilt:

Jetzt müsste es klappen, die Rechnung muss aber noch ausgeführt werden.
\begin{align*}
&x^{(k+1)}=Tx^{(k)}+ {c}=\lambda^{k+1}x^{(0)}\\
&\Rightarrow x^{(k)}=\lambda^kx^{(0)}
\end{align*}
Die Folge konvergiert insbesondere nicht gegen die Lösung $y = 0$.\\
