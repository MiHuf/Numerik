\subsection*{Aufgabe 16}

Es seien $A,M \in \mathbf{Gl}_n(\RR), N \in \RR^{n\times n}$ so dass $A=M-N$, sowie $b \in \RR^n$. Weiterhin sei $T=M^{-1}N$ mit $\rho(T)\ge 1$ und $c=M^{-1}b$.\\
zu zeigen: Es existiert ein Startwert $x^{(0)} \in \RR^n$, so dass die rekursiv definierte Folge $(x^{(k)})$ mit $x^{(k+1)}=Tx^{(k)}+c$ nicht gegen $x=A^{-1}b$ konvergiert.\\
\newline
Beweis:\\
Wir transformieren $y := x - A^{-1} b$, dann gilt $A y = 0 \; \Leftrightarrow \; A x = b$.
Sei $\rho(T)\ge 1$, dann gibt es einen Eigenvektor $v \neq 0$ von $T$ zum Eigenwert $\lambda$ mit $|\lambda|\ge 1$.\\
\newline
Um zu zeigen, dass ein Startwert $x^{(0)}$ existiert, sodass die Folge nicht gegen die Lösung von $x=A^{-1}b$ konvergiert, reicht es also zu zeigen, dass ein $y^{(0)}$ existiert, sodass die Folge nicht gegen die Lösung $y=0$ konvergiert.\\
\newline
Wählen wir $y^{(0)}=v$ dann gilt:
\begin{align*}
&y^{(k+1)}=Ty^{(k)}+ {c}=Ty^{(k)}\\
&\Rightarrow y^{(k)}=\lambda^kv
\end{align*}
Die Folge konvergiert insbesondere nicht gegen die Lösung $y = 0$.\\
%Und da $A y = 0 \; \Leftrightarrow \; A x = b$. gilt, insbesondere auch nicht gegen die Lösung von $x=A^{-1}b$.
Somit existiert ein $x^{(0)}$ mit $x^{(0)}=v+A^{-1}b$, sodass die Folge $x^{(k)}$ nicht gegen die Lösung $x=A^{-1}b$ konvergiert.
\begin{flushright}
Q.e.d.
\end{flushright}
