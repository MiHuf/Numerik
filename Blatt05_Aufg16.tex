\subsection*{Aufgabe 16}

Es seien $A,M \in \mathbf{Gl}_n(\RR), N \in \RR^{n\times n}$ so dass $A=M-N$, sowie $b \in \RR^n$. Weiterhin sei $T=M^{-1}N$ mit $\rho(T)\ge 1$ und $c=M^{-1}b$.\\\\
zu zeigen: Es existiert ein Startwert $x^{(0)} \in \RR^n$, so dass die rekursiv definierte Folge $(x^{(k)})$ mit $x^{(k+1)}=Tx^{(k)}+c$ nicht gegen $x=A^{-1}b$ konvergiert.\\
\newline
Beweis:\\
Sei $\rho(T)\ge 1$, dann gibt es einen Eigenvektor $x \neq 0$ von $T$ zum Eigenwert $\lambda$ mit $|\lambda|\ge 1$.\\
Wählen wir dieses x als Startwert, also $x^{(0)}=x$, dann gilt:\\\\
1.Fall: $b=0$
\begin{align*}
&x^{(k+1)}=Tx^{(k)}+ \underset{=0}{c}=\lambda^{k+1}x^{(0)}\\
&\Rightarrow x^{(k)}=\lambda^kx^{(0)}
\end{align*}
Die Folge konvergiert insbesondere nicht gegen die Lösung 0.\\
2. Fall: $b<0$\\
3. Fall: $b>0$