\subsection*{Aufgabe 23}
Es seien $x_0,...,x_n \in \RR$ gegegeben. Berechnen Sie die Determinante der Vandermonde-Matrix
\begin{align*}
 \label{eq8}
 V(x_0,...,x_n)=
 \begin{split}
 \begin{pmatrix}
 1 & x_0 & x_0^2 & \cdots & x_0^{n} \\
 1 & x_1 & x_1^2 & \cdots & x_1^{n} \\
 \vdots & \vdots & \vdots &  & \vdots \\
 1 & x_n & x_n^2 & \cdots & x_n^{n}
 \end{pmatrix}
 \end{split}
 = A
\end{align*}
und begründen Sie, warum sich damit zeigen lässt, dass die Polynominterpolation mit $(n+1)$ paarweise verschiedenen Stützstellen für Polynome aus $\Pi_n$ immer eindeutig lösbar ist.\\

Zu Berechnen: $\det(V(x_0,...,x_n))=\det(A)$\\
Behauptung: $\det(A) = \prod \limits_{1 \leq i < j \leq n}(x_j-x_i)$\\
Beweis durch vollständige Induktion:\\
\begin{itemize}
	\item[IA:] $n=2$\\
	\begin{align*}
	A =
	\begin{split}
	\begin{pmatrix}
	1 & x_0 & x_0^2 \\
	1 & x_1 & x_1^2 \\
	1 & x_2 & x_2^2 
	\end{pmatrix}
	\end{split}\\
	\Rightarrow \det(A) &= x_1x_2^2-x_2x_1^2-(x_0x_2^2-x_2x_0^2)+x_0x_1^2-x_1x_0^2\\
	&= (x_2-x_0)(x_2-x_1)(x_1-x_0)
	\end{align*}
	\item[IAn:] $\det(A)=\prod \limits_{1 \leq i < j \leq n}(x_j-x_i)$ gilt für $n+1$
	\item[IS:] $n+1$\\
	\begin{align*}
	A_{n+1}=
	\begin{split}
	\begin{pmatrix}
	1 & x_0 & x_0^2 & \cdots & x_0^{n} \\
	1 & x_1 & x_1^2 & \cdots & x_1^{n} \\
	\vdots & \vdots & \vdots &  & \vdots \\
	1 & x_n & x_n^2 & \cdots & x_n^{n}\\
	1 & x_{n+1} & x_{n+1}^2 & \cdots & x_{n+1}^{n}
	\end{pmatrix}
	\end{split}
	\end{align*}
	Das $x_0$-fache der vorletzten wird von der letzten Spalte abgezogen.\\
	Das $x_0$-fache der drittletzte wird von der vorletzten Spalte abgezogen.\\
	$\vdots$\\
	Bis das $x_0$-fache der ersten von der zweiten Spalte abgezogen wird.\\
	Daraus folgt:\\
	\begin{align*}
	A_{n+1}=
	\begin{split}
	\begin{pmatrix}
	1 & (x_0-x_0) & x_0(x_0-x_0) & \cdots & x_0^{n-1}(x_0-x_0) \\
	1 & (x_1-x_0) & x_1(x_1-x_0) & \cdots & x_1^{n-1}(x_1-x_0) \\
	\vdots & \vdots & \vdots &  & \vdots \\
	1 & (x_n-x_0) & x_n(x_n-x_0) & \cdots & x_n^{n-1}(x_n-x_0)\\
	1 & (x_{n+1}-x_0) & x_{n+1}(x_{n+1}-x_0) & \cdots & x_{n+1}^{n-1}(x_{n+1}-x_0)
	\end{pmatrix}
	\end{split}
	\end{align*}
	Entwickeln der 1. Zeile nach Laplace\\
	Daraus folgt:\\
	\begin{align*}
	(x_1-x_0)\cdot \cdots \cdot (x_{n+1}-x_0) \cdot \det
	\begin{split}
	\begin{pmatrix}
	1 & x_1 & \cdots & x_1^{n-1} \\
	1 & x_2 & \cdots & x_2^{n-1} \\
	\vdots & \vdots &  & \vdots \\
	1 & x_n & \cdots & x_n^{n-1}\\
	1 & x_{n+1} & \cdots & x_{n+1}^{n-1}
	\end{pmatrix}
	\end{split}
	=A_{n+1}
	\end{align*}\\
	\begin{align*}
	\Rightarrow A_{n+1} &= \prod \limits_{2 \leq j \leq n+1}(x_j-x_0)\cdot \det(A'_{n+1})\\
	&\stackrel{IA}{=} (\prod \limits_{2 \leq j \leq n+1}(x_j-x_0))\cdot (\prod \limits_{2 \leq i < j \leq n+1}(x_j-x_i))\\
	&= \prod \limits_{1 \leq i < j \leq n+1}(x_j-x_i)
	\end{align*}
\end{itemize}
	\begin{flushright}
		Q.e.d.
	\end{flushright}
	Aus der Determinate $\det(A)=\prod \limits_{1 \leq i < j \leq n}(x_j-x_i)$ lässt sich zeigen, dass die Polynominterpolation mit $n+1$ paarweise verschiedenen Stützstellen für Polynome aus $\Pi_n$ immer eindeutig lösbar ist.\\
	Gegeben seien $n+1$ Punkte in der Ebene $(x_0,y_0),(x_1,y_1),\cdots,(x_n,y_n)$.\\
	Wir suchen ein Polynom $p$, mit möglichst niedrigen Grad, das die Eigenschaft $p(x_0)=y_0,p(x_1)=y_1, \cdots,p(x_n)=y_n$ erfüllt.\\
	Da ein Polynom $n$-ten Grades $n+1$ unbekannte Koeffizienten aufweist und $n+1$ Punkte vorgegeben sind, kann man bei der Interpolation ein lineares Gleichungssystem bilden.\\
	Mit $p(x) = a_0+a_1x_0+a_2x_0^2+ \cdots + a_nx_0^n$ erhalten wir $n+1$ Gleichungen:
	\begin{align*}
	p(x_0) &= a_0+a_1x_0+a_2x_0^2+ \cdots + a_nx_0^n &= y_0\\
	p(x_1) &= a_0+a_1x_1+a_2x_1^2+ \cdots + a_nx_1^n &= y_1\\
	p(x_2) &= a_0+a_1x_2+a_2x_2^2+ \cdots + a_nx_2^n &= y_2\\
	\vdots\\
	p(x_n) &= a_0+a_1x_n+a_2x_n^2+ \cdots + a_nx_n^n &= y_n\\
	\end{align*}
	Die $a_i$ mit $i=0,\cdots,n$ sind die Unbekannten.\\
	Das LGS hat genau eine Lösung, wenn die Determinate des Systems nicht verschwindet.\\
	Da die Determinate von $A$: $\det(A)=\prod \limits_{1 \leq i < j \leq n}(x_j-x_i)$ ist, setzen wir, dass die Determinate nur ungleich Null ist, wenn die Stützstellen paarweise verschieden sind.
	\begin{flushright}
		Q.e.d.
	\end{flushright}



