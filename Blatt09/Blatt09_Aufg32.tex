\subsection*{Aufgabe 32}
Gegeben $z_i,  y_i := f(z_i) \; ; i = 1, \cdots , m-1$. Gesucht ein Polynom
$p(z) = \sum_{i = 0}^{m-1} x_i z^i$ so dass $\sum_{i = 0}^{m-1} (p(z_i) - y_i)^2$
minimal wird.

\paragraph*{a) Rechenvorschrift}
Die Normalengleichung für das Ausgleichsproblem lautet:
\begin{align}
\label{eq-norm}
\underbrace{
  \begin{pmatrix}z_0^0 & z_0^1 & \cdots \\ \vdots & \vdots  & \cdots\\ z_{m-1}^0 & z_{m-1}^1 &\cdots\end{pmatrix}}_{= A} \cdot
\underbrace{
  \begin{pmatrix}x_0 \\ x_1 \\ \vdots  \end{pmatrix}
  }_{= x} =
\underbrace{
  \begin{pmatrix}y_0 \\ \vdots \\y_{m-1}\end{pmatrix}
  }_{= y}
\end{align}
bzw. für ein Ausgleichspolynom 1. Grades:
\begin{align*}
  \begin{pmatrix}1 & z_0 \\ \vdots & \vdots \\ 1 & z_{m-1} \end{pmatrix} \cdot
  \begin{pmatrix}x_0 \\ x_1\end{pmatrix} =
  \begin{pmatrix}y_0 \\ \vdots \\y_{m-1}\end{pmatrix}
\end{align*}
Aus der Nromalengleichung \eqref{eq-norm} $A x = y$ folgt $A^T A x = A^T y$. Da
$A^T A$ eine invertierbare Matrix ist, ist letzteres ein lösbares lineares
Gleichungssystem für $x$. Kochrezept:
\begin{enumerate}
  \item Berechne $b := A^T y$
  \item Berechne $B := A^T A$
  \item Löse das LGS $B x = b$ und erhalte $x_0, x_1$
  \item Preise dich glücklich und lobe den Herrn.
\end{enumerate}

\paragraph*{b) mit Zahlen}
\begin{align*}
  & A =  \begin{pmatrix}1 & 0 \\ 1 & 1 \\ 1 & 2 \\ 1 & 3 \\ 1 & 4 \end{pmatrix} \; ; \quad
  A^T =  \begin{pmatrix} 1 & 1 & 1 & 1 & 1 \\ 0 & 1 & 2 & 3 & 4 \end{pmatrix} \; ; \quad
  y =  \begin{pmatrix} 0 \\ 0 \\ 3 \\ 4 \\ 3 \end{pmatrix} \\
  & A^T A = \begin{pmatrix} 5 &  10 \\10 & 3 0\end{pmatrix}
    = 5 \begin{pmatrix} 1 &  2 \\ 2 & 6 \end{pmatrix}  \; ; \quad
  A^T y = \begin{pmatrix} 10 \\ 30 \end{pmatrix} = 5 \begin{pmatrix} 2 \\ 6 \end{pmatrix}
\intertext{Damit ergibt sich das LGS:}
& 5 \cdot \begin{pmatrix} 1 &  2 \\ 2 & 6 \end{pmatrix}
\begin{pmatrix} x_0 \\ x_1 \end{pmatrix} =
5 \cdot \begin{pmatrix} 2 \\ 6 \end{pmatrix} \quad \text{mit der Lösung} \quad
x_1 = 1 \; ; \quad x_0 = 0
\end{align*}
Die Ausgleichsgerade ist also $p(z) = z$.

\paragraph*{c)}
Wir verschieben das Koordinatensystem, so dass der Punkt (2,3) im Ursprung liegt:\\
$z' = z - 2\; ; \; f'(z') = f(z - 2) - 3$. Damit ergeben sich die Werte:

\begin{tabular}{c|ccccc}
  $z_i$ & 0 & 1 & 2 & 3 & 4\\
  $y_i = f(z_i)$ & 0 & 0 & 3 & 4 & 3 \\
  \hline
  $z'_i = z_i - 2$ & -2 & -1 & 0 & 1 & 2\\
  $y'_i = y_i - 3$ & -3 & -3 & 0 & 1 & 0
\end{tabular} \quad
Für die Normalengleichung ergibt sich damit:
\begin{align*}
  & A'^T A' =  \begin{pmatrix} 0 & 0 & 0 & 0 & 0 \\ -2 & -1 & 0 & 1 & 2 \end{pmatrix}
   \begin{pmatrix} 0 & -2 \\ 0 & -1 \\ 0 & 0 \\ 0 & 1 \\ 0 & 2 \end{pmatrix} =
   \begin{pmatrix} 0 & 0  \\ 0 & 10 \end{pmatrix} \\
  & A'^T y'= \begin{pmatrix} 0 & 0 & 0 & 0 & 0 \\ -2 & -1 & 0 & 1 & 2 \end{pmatrix}
   \begin{pmatrix} -3 \\ -3 \\ 0 \\ 1 \\ 0 \end{pmatrix} = \begin{pmatrix}0 \\10 \end{pmatrix}
\end{align*}
Das LGS $A'^T A' z' = A'^T y'$ hat die Lösung $z'_0 = 0$ und $z'_1 = 1$. Das
Ausgleichspolynom lautet also im verschobenen Koordinatensystem $y' = 1 \cdot z'$.
Einsetzen von $y' = y - 3$ und $z' = z - 2$ liefert dann $y - 3 = z - 2$ bzw. $y = z + 1 = p_1(z)$.


