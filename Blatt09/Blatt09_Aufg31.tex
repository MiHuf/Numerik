\subsection*{Aufgabe 31}
Michaels-Version:
\paragraph*{a) Newton Verfahren} Gegeben sind
\begin{align*}
  F(x, y) = \begin{pmatrix}f(x,y) \\ g(x,y)\end{pmatrix} =
  \begin{pmatrix}x^2 +3 y^2 - 6 y + 2 \\ x^2 - 4 y^2 + x + y + 1\end{pmatrix}
  \;  \text{und }\;  X^{(0)} = \begin{pmatrix}x^{(0)} \\ y^{(0)} \end{pmatrix} =
  \begin{pmatrix} \frac{1}{2} \\ 1 \end{pmatrix}
\end{align*}
gesucht ist die Nullstelle von $F$ mit dem Newton-Verfahren.

Die Newton-Interationsvorschrift für eine eindimensionale Funktion $f(x)$ lautet
$x^{(n+1)} = x^{(n)} - \frac{f(x^{(n)})}{f'(x^{(n)})}$. Im mehrdimensionalen Fall
wird die Ableitung im Nenner durch die inverse Jacobi-Matrix ersetzt. Das ergibt in
diesem Fall:
\begin{align}
\nonumber
  & J(x,y) = \frac{dF(x,y)}{d(x,y)} =
\begin{pmatrix}\frac{\partial f}{\partial x}& \frac{\partial f}{\partial y}\\
  \frac{\partial g}{\partial x}& \frac{\partial g}{\partial y} \end{pmatrix} =
\begin{pmatrix}2 x & 6 y - 6 \\ 2 x + 1 & -8 y + 1 \end{pmatrix} \\
  & X^{(n+1)} = X^{(n)} - J^{-1}(X^{(n)}) \cdot F(X^{(n)}) \quad
  \text{= mehrdimensionales Newton-Verfahren}
\intertext{oder mit $D^{(n+1)} := X^{(n+1)} - X^{(n)}$}
\nonumber
& D^{(n+1)} = - J^{-1}(X^{(n)}) \cdot F(X^{(n)}) \; ; \;\Rightarrow\;
    J(X^{(n)}) \cdot D^{(n+1)} = -F(X^{(n)})
\intertext{und mit $X^{(0)}$ aus der Aufgabenstellung:}
& F(X^{(0)}) = \begin{pmatrix}-\frac{3}{4} \\ -\frac{5}{4}\end{pmatrix}\; ; \quad
 J(X^{(0)}) = \begin{pmatrix}1 & 0 \\ 2 & -7\end{pmatrix}
\end{align}

\paragraph*{b) LGS lösen}
Somit ergibt sich $D^{(1)}$ als Lösung des LGS:
\begin{align*}
  \begin{pmatrix}1 & 0 \\ 2 & -7\end{pmatrix} \cdot
  \begin{pmatrix}d_x^{(1)} \\ d_y^{(1)}\end{pmatrix} =
  - \begin{pmatrix}-\frac{3}{4} \\ -\frac{5}{4}\end{pmatrix} =
  \begin{pmatrix}\frac{3}{4} \\ \frac{5}{4}\end{pmatrix} \; ; \quad
  \text{mit } D^{(n)} = \begin{pmatrix}d_x^{(n)} \\ d_y^{(n)}\end{pmatrix}
\end{align*}
Die Lösung ergibt sich direkt durch Vorwärtseinsetzen: $d_x^{(1)} = \frac{3}{4}$
und $d_y^{(1)} = \frac{1}{28}$. Mit $X^{(1)} = D^{(1)} + X^{(0)}$ ergibt sich dann
für den ersten Newton-Iterationsschritt:
$x^{(1)} = \frac{5}{4}$ und $^{(1)} = \frac{29}{28}$.

\paragraph*{c) Matlab}
Alter Code:
\begin{verbatim}
...
tol = 1e-5; Nmax = 50; x = x0;
%
for it = 1:Nmax
  dx = -DF(x(1),x(2))\F(x(1),x(2));
  x = x+dx;
end
\end{verbatim}
Neuer Code:
\begin{verbatim}
...
tol = 1e-5; Nmax = 50; x = x0;
%
it = 0 ;
dx = [1.0 ; 1.0] ;
while (it < NMax) && (norm(dx) > tol))
  it = it + 1 ;
  % dx = -DF(x(1),x(2))\F(x(1),x(2));
  dx = - inv(DF(x(1),x(2))) * F(x(1),x(2));
  x = x+dx;
end
\end{verbatim}\\\\
im Folgenden Annika-Version: Aufgabenteil b) mit Einzelschrittverfahren
\paragraph*{a) Newton Verfahren}\hfill\\ Gegeben:
\begin{align*}
  F(x, y) = \begin{pmatrix}f(x,y) \\ g(x,y)\end{pmatrix} =
  \begin{pmatrix}x^2 +3 y^2 - 6 y + 2 \\ x^2 - 4 y^2 + x + y + 1\end{pmatrix}
\end{align*}
gesucht ist die Nullstelle von $F$ mit dem Newton-Verfahren.\\
\newline
Die allgmeine Berechnungsvorschrift für das Newton-Verfahren lautet im mehrdimensionalen Fall:\\ $X^{(k+1)} = X^{(k)} - J^{-1}(X^{(k)}) \cdot F(X^{(k)})$ mit $X^{(k)} = \begin{pmatrix}x^{(k)} \\ y^{(k)} \end{pmatrix}$\\
Mit der Bezeichnung $\Delta X=X^{(k+1)} - X^{(k)}$ folgt dann:
\begin{align*}
&\Delta X=- J^{-1}(X^{(k)}) \cdot F(X^{(k)})\\
&X^{(k+1)}=\Delta X + X^{(k)}
\end{align*}
Im konkreten Fall gilt für die Jacobi-Matrix:
\begin{align*}
 & J(x,y) = \frac{dF(x,y)}{d(x,y)} =
\begin{pmatrix}\frac{\partial f}{\partial x}& \frac{\partial f}{\partial y}\\
  \frac{\partial g}{\partial x}& \frac{\partial g}{\partial y} \end{pmatrix} =
\begin{pmatrix}2 x & 6 y - 6 \\ 2 x + 1 & -8 y + 1 \end{pmatrix}
\end{align*}
Also ergibt sich hier folgende Berechnungsvorschrift:
\begin{align*}
&\Delta X=- \begin{pmatrix}2 x & 6 y - 6 \\ 2 x + 1 & -8 y + 1 \end{pmatrix}^{-1} \cdot \begin{pmatrix}x^2 +3 y^2 - 6 y + 2 \\ x^2 - 4 y^2 + x + y + 1\end{pmatrix}\\\\
&\text{ bzw. in der Form } J (X^{(k)})\cdot \Delta X= -F(X^{(k)})\\
&\begin{pmatrix}2 x & 6 y - 6 \\ 2 x + 1 & -8 y + 1 \end{pmatrix} \cdot \Delta X=-\begin{pmatrix}x^2 +3 y^2 - 6 y + 2 \\ x^2 - 4 y^2 + x + y + 1\end{pmatrix}
\end{align*}
\paragraph{b)}
Wir wenden nun obige Berechnungsvorschrift mit dem Startvektor $\begin{pmatrix}
\frac{1}{2}\\1
\end{pmatrix}$an.
\begin{align*}
 \begin{pmatrix}2 \cdot \frac{1}{2} & 6 - 6 \\ 2 \cdot \frac{1}{2} + 1 & -8 + 1 \end{pmatrix} \cdot \Delta X &= - \begin{pmatrix}(\frac{1}{2})^2 +3 \cdot1^2 - 6 + 2 \\ (\frac{1}{2})^2 - 4 \cdot 1^2 + \frac{1}{2} + 1 + 1\end{pmatrix} \\\\ \begin{pmatrix}1 & 0 \\ 2& -7 \end{pmatrix} \cdot \Delta X &=-\begin{pmatrix}-\frac{3}{4} \\ -\frac{5}{4}\end{pmatrix}
\end{align*}
Das Gleichungssystem der Form $Ax=b$ lösen wir mithilfe des Einzelschrittverfahrens:
Dafür stellen wir folgende Gleichung für das Einzelschrittverfahren mit der Zerlegung $A=L+D+R$ auf:
\begin{align*}
\Delta X^{(k+1)}&=(D+L)^{-1}(-R\Delta X^{(k)}+b) \text{ mit }D=\begin{pmatrix} 1 & 0\\0 & -7 \end{pmatrix},\; L=\begin{pmatrix}
0 & 0 \\ 2 & 0\end{pmatrix} \text{ und } R=\begin{pmatrix}
0 & 0 \\ 0 & 0 \end{pmatrix}\\
\Delta X^{(k+1)}&= \begin{pmatrix} 1 & 0\\2 & -7 \end{pmatrix}^{-1}\cdot\left(-\begin{pmatrix}0 & 0 \\ 0 & 0 \end{pmatrix} \Delta X^{(k)}+ \begin{pmatrix}\frac{3}{4} \\ \frac{5}{4}\end{pmatrix}\right)\\
&= \begin{pmatrix} 1 & 0\\2 & -7 \end{pmatrix}^{-1}\cdot \begin{pmatrix}\frac{3}{4} \\ \frac{5}{4}\end{pmatrix}=\begin{pmatrix} 1 & 0\\\frac{2}{7} & -\frac{1}{7} \end{pmatrix}\cdot \begin{pmatrix}\frac{3}{4} \\ \frac{5}{4}\end{pmatrix}=\begin{pmatrix}
\frac{3}{4}\\\frac{1}{28}
\end{pmatrix}
\end{align*}
Aus $\Delta X^{(k+1)}=\begin{pmatrix}
\frac{3}{4}\\\frac{1}{28}
\end{pmatrix}$ ergibt sich dann:
\begin{align*}
&\Delta X^{(k+1)}=\begin{pmatrix}
x^{(k+1)}-x^{(k)} \\ y^{(k+1)}-y^{(k)}\end{pmatrix}=\begin{pmatrix}
\frac{3}{4}\\\frac{1}{28}
\end{pmatrix}\\
&\Rightarrow x^{(1)}=x^{(0)}+\frac{3}{4} \text{ und } y^{(1)}=y^{(0)}+\frac{31}{28}\\
&\Rightarrow x^{(1)}=\frac{5}{4} \text{ und } y^{(1)}=\frac{29}{28}
\end{align*}
\paragraph{c)} analog zu Michaels Lösung
