\subsection*{Aufgabe 31}

\paragraph*{a Newton Verfahren)} Gegeben sind
\begin{align*}
  F(x, y) = \begin{pmatrix}f(x,y) \\ g(x,y)\end{pmatrix} =
  \begin{pmatrix}x^2 +3 y^2 - 6 y + 2 \\ x^2 - 4 y^2 + x + y + 1\end{pmatrix}
  \;  \text{und }\;  X^{(0)} = \begin{pmatrix}x^{(0)} \\ y^{(0)} \end{pmatrix} =
  \begin{pmatrix} \frac{1}{2} \\ 1 \end{pmatrix}
\end{align*}
gesucht ist die Nullstelle von $F$ mit dem Newton-Verfahren.

Die Newton-Interationsvorschrift für eine eindimensionale Funktion $f(x)$ lautet
$x^{(n+1)} = x^{(n)} - \frac{f(x^{(n)})}{f'(x^{(n)})}$. Im mehrdimensionalen Fall
wird die Ableitung im Nenner durch die Inverse Jacobi-Matrix ersetzt. Das ergibt in
diesem Fall:
\begin{align}
\nonumber
  & J(x,y) = \frac{dF(x,y)}{d(x,y)} =
\begin{pmatrix}\frac{\partial f}{\partial x}& \frac{\partial f}{\partial y}\\
  \frac{\partial g}{\partial x}& \frac{\partial g}{\partial y} \end{pmatrix} =
\begin{pmatrix}2 x & 6 y - 6 \\ 2 x + 1 & -8 y + 1 \end{pmatrix} \\
  & X^{(n+1)} = X^{(n)} - J^{-1}(X^{(n)}) \cdot F(X^{(n)}) \quad
  \text{= mehrdimensionales Newton-Verfahren}
\intertext{oder mit $D^{(n+1)} := X^{(n+1)} - X^{(n)}$}
\nonumber
& D^{(n+1)} = - J^{-1}(X^{(n)}) \cdot F(X^{(n)}) \; ; \;\Rightarrow\;
    J(X^{(n)}) \cdot D^{(n+1)} = -F(X^{(n)})
\intertext{und mit $X^{(0)}$ aus der Aufgabenstellung:}
& F(X^{(0)}) = \begin{pmatrix}-\frac{3}{4} \\ -\frac{5}{4}\end{pmatrix}\; ; \quad
 J(X^{(0)}) = \begin{pmatrix}1 & 0 \\ 2 & -7\end{pmatrix}
\end{align}

\paragraph*{b) LGS lösen}
Somit ergibt sich $D^{(1)}$ als Lösung des LGS:
\begin{align*}
  \begin{pmatrix}1 & 0 \\ 2 & -7\end{pmatrix} \cdot
  \begin{pmatrix}d_x^{(1)} \\ d_y^{(1)}\end{pmatrix} =
  - \begin{pmatrix}-\frac{3}{4} \\ -\frac{5}{4}\end{pmatrix} =
  \begin{pmatrix}\frac{3}{4} \\ \frac{5}{4}\end{pmatrix} \; ; \quad
  \text{mit } D^{(n)} = \begin{pmatrix}d_x^{(n)} \\ d_y^{(n)}\end{pmatrix}
\end{align*}
Die Lösung ergibt sich direkt durch Vorwärtseinsetzen: $d_x^{(1)} = \frac{3}{4}$
und $d_y^{(1)} = \frac{1}{28}$. Mit $X^{(1)} = D^{(1)} + X^{(0)}$ ergibt sich dann
für den ersten Newton-Iterationsschritt:
$x^{(1)} = \frac{5}{4}$ und $^{(1)} = \frac{29}{28}$.

\paragraph*{c)}
Alter Code:
\begin{verbatim}
...
tol = 1e-5; Nmax = 50; x = x0;
%
for it = 1:Nmax
  dx = -DF(x(1),x(2))\F(x(1),x(2));
  x = x+dx;
end
\end{code}
\end{verbatim}
Neuer Code:
\begin{verbatim}
...
tol = 1e-5; Nmax = 50; x = x0;
%
it = 0 ;
dx = [1.0 ; 1.0] ;
while (it < NMax) && (norm(dx) > tol))
  it = it + 1 ;
  % dx = -DF(x(1),x(2))\F(x(1),x(2));
  dx = - inv(DF(x(1),x(2))) * F(x(1),x(2));
  x = x+dx;
end
\end{code}
\end{verbatim}
