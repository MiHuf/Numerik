\subsection*{Aufgabe 29}
Gegeben $f(x) = \sin(x)$  im Intervall $x \in [0, \frac{\pi}{2}]$.

\paragraph*{a)}
Bestimme das Interpolationspolynom für die Stützstellen $x = 0,  \frac{\pi}{4},  \frac{\pi}{2}$.
Wertetabelle $i = 0, 1, 2$ für Teil a), $i = 0, 1, 2, 3$ für Teil c): \newline
\begin{tabular}{l|cccc}
  $i$ & 0 & 1 & 2 & 3 \\
  \hline
  $x_i$ & 0 & $\frac{\pi}{4}$ &  $\frac{\pi}{2}$ &  $\frac{\pi}{6}$\\
  \hline
  $f(x_i)$ & 0 & $\frac{1}{2}\sqrt{2}$ &  $1$  &  $\frac{1}{2}$
\end{tabular} \\\\
Bestimmung der Koeffizienten $\gamma_i$ nach der Methode der dividierten
Differenzen (Neville-Aitken-Schema):
\begin{align}
\label{eq-schema}
\begin{array}{l|l|l|l|l}
  i & x_i & f[x_i] & f[x_i , \cdot] & f[x_i , \cdot, \cdot]\\
 \hline
  0 & 0 & f[x_0] = 0 = \gamma_0 &
    f[x_0, x_1] = \frac{f[x_0] - f[x_1]}{x_0 - x_1} = \frac{2 \sqrt{2}}{\pi} = \gamma_1&
    \frac{f[x_0, x_1] - f[x_1, x_2]}{x_0 - x_2} = \frac{8 - 8 \sqrt{2}}{\pi^2} = \gamma_2\\
  1 & \frac{\pi}{4} & f[x_1] = \frac{1}{2}\sqrt{2} &
    f[x_1, x_2] = \frac{f[x_1] - f[x_2]}{x_1 - x_2} = \frac{4 - 2 \sqrt{2}}{\pi} & \nearrow\\
  2 & \frac{\pi}{2} & f[x_2] = 1 & \nearrow
% \\ 3 & \frac{\pi}{6} & f[x_2] = \frac{1}{2} &
\end{array}
\end{align}
Damit ergibt sich das Interpolationspolynom $p(x) = \gamma_0 + \gamma_1(x-x_0) + \gamma_2(x-x_0)(x-x_1)$ zu:
\begin{align*}
p(x) & = 0 +  \frac{2 \sqrt{2}}{\pi} \left(x - 0\right) +
\left.\frac{8 - 8 \sqrt{2}}{\pi^2} \left(x- 0\right)\left(x-\frac{\pi}{4}\right)\right. \\
& = \frac{2 \sqrt{2}}{\pi} x + \frac{8 - 8 \sqrt{2}}{\pi^2} x^2 - \frac{8 - 8 \sqrt{2}}{\pi^2} \frac{\pi}{4} x
= \frac{4 \sqrt{2} - 2}{\pi} x + \frac{8 - 8 \sqrt{2}}{\pi^2} x^2
\end{align*}

\paragraph*{b)}
Der Fehler $|f(x)-p(x)|$ im Intervall $[0, \frac{\pi}{2}]$ beträgt
\begin{align}
\nonumber
  & |f(x)-p(x)| = \frac{f^{(n)}(\xi)}{(n+1)!} \prod_{i = 0}^n (\xi - x_i) \quad
  \text{mit einem unbekannten $\xi \in \left[0, \frac{\pi}{2}\right]$.}
\intertext{Er lässt sich abschätzen, mit $\left|\sin^{(n)}(\xi)\right| \le 1$,
 $\left|\prod_{i = 0}^n (\xi - x_i)\right| \le n! \cdot (\max(x_{i+1} - x_i))^n$ und $n = 2$:}
 \label{eq_schaetz}
 & |f(x)-p(x)| \le \frac{1}{n+1} \left(\frac{\pi}{4}\right)^n = \frac{\pi^2}{48}
\approx 0{,}206
\end{align}

\paragraph*{c)}
Wir ergänzen \eqref{eq-schema} um eine weitere Zeile:
\begin{align*}
& \begin{array}{l|l|l|l|l}
 i & x_i & f[x_i] & f[x_i , \cdot] & f[x_i , \cdot, \cdot]\\
 \hline
  0 & 0 & f[x_0] = 0 = \gamma_0 &
    \frac{f[x_0] - f[x_1]}{x_0 - x_1} = \frac{2 \sqrt{2}}{\pi} = \gamma_1&
    \frac{f[x_0, x_1] - f[x_1, x_2]}{x_0 - x_2} = \frac{8 - 8 \sqrt{2}}{\pi^2} = \gamma_2\\
  1 & \frac{\pi}{4} & f[x_1] = \frac{1}{2}\sqrt{2} &
    \frac{f[x_1] - f[x_2]}{x_1 - x_2} = \frac{4 - 2 \sqrt{2}}{\pi} &
    \frac{f[x_1, x_2] - f[x_2, x_3]}{x_1 - x_3} =  \frac{30-24\sqrt{2}}{\pi^2}\\
  2 & \frac{\pi}{2} & f[x_2] = 1 &
  \frac{f[x_2] - f[x_3]}{x_2 - x_3} =  \frac{3}{2 \pi} & \nearrow\\
  3 & \frac{\pi}{6} & f[x_2] = \frac{1}{2} & \nearrow
\end{array} \\
&f[x_0, x_1, x_2, x_3] = \frac{f[x_0, x_1, x_2] - f[x_1, x_2, x_3]}{x_0 - x_3}
= \frac{132 - 96 \sqrt{2}}{\pi^3} = \gamma_3
\end{align*}
Damit wird das Interpolationspolynom
\begin{align*}
  p(x) &= \gamma_0 + \gamma_1(x - x_0) + \gamma_2(x - x_0)(x - x_1) + \gamma_3(x - x_0)(x - x_1)(x - x_3)\\
  &= \frac{2 \sqrt{2}}{\pi} x +  \frac{8 - 8 \sqrt{2}}{\pi^2} x \left(x - \frac{\pi}{4}\right)
  + \frac{132 - 96 \sqrt{2}}{\pi^3} x\left(x - \frac{\pi}{4}\right) \left(x - \frac{1}{2}\right)
\end{align*}
Das ergibt nach Ausmultiplizieren das gesuchte Polynom 3. Grades in $x$.

\paragraph*{d)}
Eine ähnliche Abschätzung wie \eqref{eq_schaetz} liefert, nun mit $n = 3$:
\begin{align*}
  |f(x)-p(x)| \le \frac{1}{4} \left(\frac{\pi}{4}\right)^3 = \frac{\pi^3}{256} \approx 0{,}121
\end{align*}
