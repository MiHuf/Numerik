\subsection*{Aufgabe 30}
Gegeben das LGS $A x = b$ mit
\begin{align}
\label{eq-A}
  A = \begin{pmatrix}4&1&0\\1&4&1\\0&1&4\end{pmatrix} \; ;\quad
  x = \begin{pmatrix}x_1\\x_2\\x_3\end{pmatrix} \; ;\quad
  b = \begin{pmatrix}-1\\0\\1\end{pmatrix}
\end{align}
\paragraph*{a) Lösung mit Cholesky-Zerlegung:}
Dies ist möglich, weil (i) $A = A^T$ symmetrisch ist und (ii) alle
Hauptdeterminanten von $A$ positiv sind:
$\det (A_1) = 4; \det (A_2) = 15, \det(A_3) = 56$. $A$ ist also positiv definit.
Somit existiert eine untere Dreiecksmatrix $L$ mit $A = L \cdot L^T$:
\begin{align}
\label{eq-L}
  L L^T = \begin{pmatrix}l_{11}&0&0\\l_{21}&l_{22}&0\\l_{31}&l_{32}&l_{33}\end{pmatrix}
  \begin{pmatrix}l_{11}&l_{21}&l_{31}\\0 &l_{22}&l_{32}\\0&0&l_{33}&\end{pmatrix} =
  \begin{pmatrix}l_{11}^2&l_{11}l_{21}&l_{11}l_{31}\\
  *&l_{21}^2+l_{22}^2&l_{21}l_{31}+l_{22}l_{32}\\
  *&*&l_{31}^2+l_{32}^2+l_{33}^2\end{pmatrix}
\end{align}
Ein Vergleich von \eqref{eq-A} mit \eqref{eq-L} liefert dann zeilenweise:
\begin{align*}
  & l_{11}^2 = 4 \;;\quad\Rightarrow\; l_{11} = 2\\
  & l_{11}l_{21} = 1 \;;\quad\Rightarrow\; l_{21} = \frac{1}{2}\\
  & l_{11}l_{31} = 0  \;;\quad\Rightarrow\;l_{31} = 0\\
  & l_{21}^2+l_{22}^2 = 4 \;;\quad\Rightarrow\; l_{22} = \sqrt{4 -\frac{1}{4}} = \frac{1}{2}\sqrt{15}\\
  & l_{21}l_{31}+l_{22}l_{32} = 1 \;;\quad\Rightarrow\; l_{22}l_{32} = 1
  \;;\quad\Rightarrow\; l_{32} = \frac{1}{l_{22}} = \frac{2}{\sqrt{15}}\\
  &l_{31}^2+l_{32}^2+l_{33}^2 = 4 \;;\quad\Rightarrow\;l_{33}^2 = 4 -\frac{4}{15} = \frac{56}{15}
  \;;\quad\Rightarrow\;l_{33} = 2 \sqrt{\frac{14}{15}}
\end{align*}
Somit ergeben sich für $L$ und $L^T$:
\begin{align}
L = \begin{pmatrix}2 & 0 & 0 \\
       \frac{1}{2} & \frac{1}{2}\sqrt{15} & 0 \\
       0 & \frac{2}{\sqrt{15}} &  2 \sqrt{\frac{14}{15}} \end{pmatrix} \; ;\quad
L^T = \begin{pmatrix}2 & \frac{1}{2} & 0 \\
      0 & \frac{1}{2}\sqrt{15} & \frac{2}{\sqrt{15}} \\
      0 & 0& 2 \sqrt{\frac{14}{15}}\end{pmatrix}
\end{align}
Die Lösung des LGS $A x = L L^T x = b$ erfolgt mit der Definition $y:= L^T x$ in
zwei Schritten. Zunächst wird $y$ aus
\begin{align*}
& L y = b = \begin{pmatrix}2 & 0 & 0 \\
       \frac{1}{2} & \frac{1}{2}\sqrt{15} & 0 \\
       0 & \frac{2}{\sqrt{15}} &  2 \sqrt{\frac{14}{15}} \end{pmatrix}
      \begin{pmatrix}y_1\\y_2\\y_3\end{pmatrix} =
      \begin{pmatrix}-1\\0\\1\end{pmatrix}
\intertext {durch Vorwärts-Einsetzen bestimmt. Ergebnis:}
& y_1 = -\frac{1}{2} \;;\quad y_2 =  \frac{1}{2\sqrt{15}} \;;\quad y_3 = \sqrt{\frac{7}{30}}
\intertext {Anschließend wird $x$ aus der Gleichung $L^T x = y$ dürch Rückwärts-Einsetzen ermittelt:}
& L^T x = y =  \begin{pmatrix}2 & \frac{1}{2} & 0 \\
      0 & \frac{1}{2}\sqrt{15} & \frac{2}{\sqrt{15}} \\
      0 & 0& 2 \sqrt{\frac{14}{15}}\end{pmatrix}
      \begin{pmatrix}x_1\\x_2\\x_3\end{pmatrix} =
      \begin{pmatrix}-\frac{1}{2} \\  \frac{1}{2\sqrt{15}} \\ \sqrt{\frac{7}{30}} \end{pmatrix}
\intertext{Ergebnis:}
& x_3 = \frac{1}{2} \cdot \sqrt{\frac{15}{14}} \cdot \sqrt{\frac{7}{30}}  = \frac{1}{4}
\;;\quad x_2 = 0 \;;\quad x_1 = -\frac{1}{4} \;;\quad
x = \begin{pmatrix}-\frac{1}{4} \\0 \\ \frac{1}{4}\end{pmatrix}
\end{align*}


\paragraph*{b) LR-Zerlegung von A:} Ziel ist es, die Matrix $A$ in zwei
Teilmatrizen $L, R$ mit $L \cdot R = A$ zu zelegen. Die Teilmatrizen haben die Form:
\begin{align*}
  L = \begin{pmatrix} 1 & 0 & 0 \\ * & 1 & 0 \\ * & * & 1\end{pmatrix} \; ;\quad
  R = \begin{pmatrix} * & * & * \\ 0 & * & * \\ 0 & 0 & *\end{pmatrix}
\end{align*}
Schritt 1: Bestimme eine Matrix $L_1$, so das $L_1 \cdot A$ die folgende Form hat:
\begin{align*}
  & L_1 \cdot A = \begin{pmatrix} 1 & 0 & 0 \\ * & 1 & 0 \\ * & 0 & 1\end{pmatrix}
  \begin{pmatrix} 4 & 1 & 0 \\ 1 & 4 & 1 \\ 0 & 1 & 4 \end{pmatrix} =
  \begin{pmatrix} 4 & 1 & 0 \\ 0 & * & * \\ 0 & * & * \end{pmatrix}
\intertext{Dies wird mit $L_1$ wie folgt erfüllt:}
  & L_1 \cdot A = \begin{pmatrix} 1 & 0 & 0 \\ -\frac{1}{4} & 1 & 0 \\ 0 & 0 & 1\end{pmatrix}
  \begin{pmatrix} 4 & 1 & 0 \\ 1 & 4 & 1 \\ 0 & 1 & 4 \end{pmatrix} =
  \begin{pmatrix} 4 & 1 & 0 \\ 0 & \frac{15}{4} & 1 \\ 0 & 1 & 4 \end{pmatrix}
\end{align*}
Schritt 2: Bestimme eine Matrix $L_2$, so das $L_2 \cdot L_1 A$ die folgende Form hat:
\begin{align*}
& L_2 \cdot L_1  A = \begin{pmatrix} 1 & 0 & 0 \\ 0 & 1 & 0 \\ 0 & * & 1\end{pmatrix}
\underbrace{
  \begin{pmatrix} 4 & 1 & 0 \\ 0 & \frac{15}{4} & 1 \\ 0 & 1 & 4 \end{pmatrix}
 }_{ =  L_1 \cdot A} =
  \begin{pmatrix} 4 & 1 & 0 \\ 0 & \frac{15}{4} & 1 \\ 0 & 0 & * \end{pmatrix}
\intertext{dies wird mit $L_2$ wie folgt erfüllt:}
& L_2 \cdot L_1  A = \begin{pmatrix} 1 & 0 & 0 \\ 0 & 1 & 0 \\ 0 & -\frac{4}{15}  & 1\end{pmatrix}
  \begin{pmatrix} 4 & 1 & 0 \\ 0 & \frac{15}{4} & 1 \\ 0 & 1 & 4 \end{pmatrix} =
  \begin{pmatrix} 4 & 1 & 0 \\ 0 & \frac{15}{4} & 1 \\ 0 & 0 & \frac{56}{15} \end{pmatrix} = R
\end{align*}
Somit haben wir $R$ ermittelt. Zur Berechnung von $L$ multiplizieren wir die letzte Gleichung
von links mit $L_1^{-1} \cdot L_2^{-1}$ und erhalten $A = L_1^{-1} \cdot L_2^{-1} \cdot R$.
Die Matrizen $L_1, L_2$ lassen sich dadurch invertieren, dass man die Vorzeichen
der Nicht-Diagonalelemente umkehrt. Damit wird:
\begin{align*}
  & L = L_1^{-1} \cdot L_2^{-1} =
  \begin{pmatrix} 1 & 0 & 0 \\ \frac{1}{4} & 1 & 0 \\ 0 & 0 & 1\end{pmatrix}
  \begin{pmatrix} 1 & 0 & 0 \\ 0 & 1 & 0 \\ 0 & \frac{4}{15}  & 1\end{pmatrix}
  =   \begin{pmatrix} 1 & 0 & 0 \\ \frac{1}{4} & 1 & 0 \\ 0 & \frac{4}{15} & 1\end{pmatrix}\\
& \text{Probe:}\quad L \cdot R =
\begin{pmatrix} 1 & 0 & 0 \\ \frac{1}{4} & 1 & 0 \\ 0 & \frac{4}{15} & 1\end{pmatrix}
\begin{pmatrix} 4 & 1 & 0 \\ 0 & \frac{15}{4} & 1 \\ 0 & 0 & \frac{56}{15} \end{pmatrix} =
\begin{pmatrix} 4 & 1 & 0 \\ 1 & 4 & 1 \\ 0 & 1 & 4 \end{pmatrix} = A
\end{align*}

