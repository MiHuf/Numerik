%  DOCUMENT CLASS
\documentclass[11pt]{article}

%PACKAGES
\usepackage[utf8]{inputenc}
\usepackage[ngerman]{babel}
\usepackage[reqno,fleqn]{amsmath}
\setlength\mathindent{10mm}
\usepackage{amssymb}
\usepackage{amsthm}
\usepackage{color}
\usepackage{delarray}
% \usepackage{fancyhdr}
\usepackage{units}
\usepackage{times, eurosym}
\usepackage{verbatim} %Für Verwendung von multiline Comments mittels \begin{comment}...\end{comment}
\usepackage{wasysym} % Für Smileys


% FORMATIERUNG
\usepackage[paper=a4paper,left=25mm,right=25mm,top=25mm,bottom=25mm]{geometry}
\usepackage{array}
\usepackage{fancybox} %zum Einrahmen von Formeln
\setlength{\parindent}{0cm}
\setlength{\parskip}{1mm plus1mm minus1mm}


% PAGESTYLE

%MATH SHORTCUTS
\newcommand{\NN}{\mathbb N}
\newcommand{\ZZ}{\mathbb Z}
\newcommand{\QQ}{\mathbb Q}
\newcommand{\RR}{\mathbb R}
\newcommand{\CC}{\mathbb C}
\newcommand{\KK}{\mathbb K}
\newcommand{\U}{\mathbb O}
\newcommand{\eqx}{\overset{!}{=}}
\newcommand{\Det}{\mathrm{Det}}
\newcommand{\Gl}{\mathrm{Gl}}
\newcommand{\diag}{\mathrm{diag}}
\newcommand{\sign}{\mathrm{sign}}
\newcommand{\rang}{\mathrm{rang}}
\newcommand{\cond}{\mathrm{cond}_{\| \cdot \|}}
\newcommand{\conda}{\mathrm{cond}_{\| \cdot \|_1}}
\newcommand{\condb}{\mathrm{cond}_{\| \cdot \|_2}}
\newcommand{\condi}{\mathrm{cond}_{\| \cdot \|_\infty}}
\newcommand{\eps}{\epsilon}

\setlength{\extrarowheight}{1ex}

\begin{document}

\begin{center}
\textbf{
Übungen zur Vorlesung Numerische Mathematik, WS 2014/15\\
Blatt 09 zum 07.01.2015\\
}

\begin{tabular}{lll}
& \\
von & Janina Geiser & Mat Nr. 6420269\\
& Michael Hufschmidt & Mat.Nr. 6436122\\
& Farina Ohm & Mat Nr. 6314051\\
& Annika Seidel & Mat Nr. 6420536\\
\\
\hline
\end{tabular}
\end{center}

\subsection*{Aufgabe 29}

\paragraph*{a)}

\paragraph*{b)}

\paragraph*{c)}

\paragraph*{d)}



\subsection*{Aufgabe 30}
Gegeben das LGS $A x = b$ mit
\begin{align}
\label{eq-A}
  A = \begin{pmatrix}4&1&0\\1&4&1\\0&1&4\end{pmatrix} \; ;\quad
  x = \begin{pmatrix}x_1\\x_2\\x_3\end{pmatrix} \; ;\quad
  b = \begin{pmatrix}-1\\0\\1\end{pmatrix}
\end{align}
\paragraph*{a) Lösung mit Cholesky-Zerlegung:}
Dies ist möglich, weil (i) $A = A^T$ symmetrisch ist und (ii) alle
Hauptdeterminanten von $A$ positiv sind:
$\det (A_1) = 4; \det (A_2) = 15, \det(A_3) = 56$. $A$ ist also positiv definit.
Somit existiert eine untere Dreiecksmatrix $L$ mit $A = L \cdot L^T$:
\begin{align}
\label{eq-L}
  L L^T = \begin{pmatrix}l_{11}&0&0\\l_{21}&l_{22}&0\\l_{31}&l_{32}&l_{33}\end{pmatrix}
  \begin{pmatrix}l_{11}&l_{21}&l_{31}\\0 &l_{22}&l_{32}\\0&0&l_{33}&\end{pmatrix} =
  \begin{pmatrix}l_{11}^2&l_{11}l_{21}&l_{11}l_{31}\\
  *&l_{21}^2+l_{22}^2&l_{21}l_{31}+l_{22}l_{32}\\
  *&*&l_{31}^2+l_{32}^2+l_{33}^2\end{pmatrix}
\end{align}
Ein Vergleich von \eqref{eq-A} mit \eqref{eq-L} liefert dann zeilenweise:
\begin{align*}
  & l_{11}^2 = 4 \;;\quad\Rightarrow\; l_{11} = 2\\
  & l_{11}l_{21} = 1 \;;\quad\Rightarrow\; l_{21} = \frac{1}{2}\\
  & l_{11}l_{31} = 0  \;;\quad\Rightarrow\;l_{31} = 0\\
  & l_{21}^2+l_{22}^2 = 4 \;;\quad\Rightarrow\; l_{22} = \sqrt{4 -\frac{1}{4}} = \frac{1}{2}\sqrt{15}\\
  & l_{21}l_{31}+l_{22}l_{32} = 1 \;;\quad\Rightarrow\; l_{22}l_{32} = 1
  \;;\quad\Rightarrow\; l_{32} = \frac{1}{l_{22}} = \frac{2}{\sqrt{15}}\\
  &l_{31}^2+l_{32}^2+l_{33}^2 = 4 \;;\quad\Rightarrow\;l_{33}^2 = 4 -\frac{4}{15} = \frac{56}{15}
  \;;\quad\Rightarrow\;l_{33} = 2 \sqrt{\frac{14}{15}}
\end{align*}
Somit ergeben sich für $L$ und $L^T$:
\begin{align}
L = \begin{pmatrix}2 & 0 & 0 \\
       \frac{1}{2} & \frac{1}{2}\sqrt{15} & 0 \\
       0 & \frac{2}{\sqrt{15}} &  2 \sqrt{\frac{14}{15}} \end{pmatrix} \; ;\quad
L^T = \begin{pmatrix}2 & \frac{1}{2} & 0 \\
      0 & \frac{1}{2}\sqrt{15} & \frac{2}{\sqrt{15}} \\
      0 & 0& 2 \sqrt{\frac{14}{15}}\end{pmatrix}
\end{align}
Die Lösung des LGS $A x = L L^T x = b$ erfolgt mit der Definition $y:= L^T x$ in
zwei Schritten. Zunächst wird $y$ aus
\begin{align*}
& L y = b = \begin{pmatrix}2 & 0 & 0 \\
       \frac{1}{2} & \frac{1}{2}\sqrt{15} & 0 \\
       0 & \frac{2}{\sqrt{15}} &  2 \sqrt{\frac{14}{15}} \end{pmatrix}
      \begin{pmatrix}y_1\\y_2\\y_3\end{pmatrix} =
      \begin{pmatrix}-1\\0\\1\end{pmatrix}
\intertext {durch Vorwärts-Einsetzen bestimmt. Ergebnis:}
& y_1 = -\frac{1}{2} \;;\quad y_2 =  \frac{1}{2\sqrt{15}} \;;\quad y_3 = \sqrt{\frac{7}{30}}
\intertext {Anschließend wird $x$ aus der Gleichung $L^T x = y$ dürch Rückwärts-Einsetzen ermittelt:}
& L^T x = y =  \begin{pmatrix}2 & \frac{1}{2} & 0 \\
      0 & \frac{1}{2}\sqrt{15} & \frac{2}{\sqrt{15}} \\
      0 & 0& 2 \sqrt{\frac{14}{15}}\end{pmatrix}
      \begin{pmatrix}x_1\\x_2\\x_3\end{pmatrix} =
      \begin{pmatrix}-\frac{1}{2} \\  \frac{1}{2\sqrt{15}} \\ \sqrt{\frac{7}{30}} \end{pmatrix}
\intertext{Ergebnis:}
& x_3 = \frac{1}{2} \cdot \sqrt{\frac{15}{14}} \cdot \sqrt{\frac{7}{30}}  = \frac{1}{4}
\;;\quad x_2 = 0 \;;\quad x_1 = -\frac{1}{4} \;;\quad
x = \begin{pmatrix}-\frac{1}{4} \\0 \\ \frac{1}{4}\end{pmatrix}
\end{align*}


\paragraph*{b) LR-Zerlegung von A:}


\subsection*{Aufgabe 31}

\paragraph*{a Newton Verfahren)} Gegeben sind
\begin{align*}
  F(x, y) = \begin{pmatrix}f(x,y) \\ g(x,y)\end{pmatrix} =
  \begin{pmatrix}x^2 +3 y^2 - 6 y + 2 \\ x^2 - 4 y^2 + x + y + 1\end{pmatrix}
  \;  \text{und }\;  X^{(0)} = \begin{pmatrix}x^{(0)} \\ y^{(0)} \end{pmatrix} =
  \begin{pmatrix} \frac{1}{2} \\ 1 \end{pmatrix}
\end{align*}
gesucht ist die Nullstelle von $F$ mit dem Newton-Verfahren.

Die Newton-Interationsvorschrift für eine eindimensionale Funktion $f(x)$ lautet
$x^{(n+1)} = x^{(n)} - \frac{f(x^{(n)})}{f'(x^{(n)})}$. Im mehrdimensionalen Fall
wird die Ableitung im Nenner durch die Inverse Jacobi-Matrix ersetzt. Das ergibt in
diesem Fall:
\begin{align}
\nonumber
  & J(x,y) = \frac{dF(x,y)}{d(x,y)} =
\begin{pmatrix}\frac{\partial f}{\partial x}& \frac{\partial f}{\partial y}\\
  \frac{\partial g}{\partial x}& \frac{\partial g}{\partial y} \end{pmatrix} =
\begin{pmatrix}2 x & 6 y - 6 \\ 2 x + 1 & -8 y + 1 \end{pmatrix} \\
  & X^{(n+1)} = X^{(n)} - J^{-1}(X^{(n)}) \cdot F(X^{(n)}) \quad
  \text{= mehrdimensionales Newton-Verfahren}
\intertext{oder mit $D^{(n+1)} := X^{(n+1)} - X^{(n)}$}
\nonumber
& D^{(n+1)} = - J^{-1}(X^{(n)}) \cdot F(X^{(n)}) \; ; \;\Rightarrow\;
    J(X^{(n)}) \cdot D^{(n+1)} = -F(X^{(n)})
\intertext{und mit $X^{(0)}$ aus der Aufgabenstellung:}
& F(X^{(0)}) = \begin{pmatrix}-\frac{3}{4} \\ -\frac{5}{4}\end{pmatrix}\; ; \quad
 J(X^{(0)}) = \begin{pmatrix}1 & 0 \\ 2 & -7\end{pmatrix}
\end{align}

\paragraph*{b) LGS lösen}
Somit ergibt sich $D^{(1)}$ als Lösung des LGS:
\begin{align*}
  \begin{pmatrix}1 & 0 \\ 2 & -7\end{pmatrix} \cdot
  \begin{pmatrix}d_x^{(1)} \\ d_y^{(1)}\end{pmatrix} =
  - \begin{pmatrix}-\frac{3}{4} \\ -\frac{5}{4}\end{pmatrix} =
  \begin{pmatrix}\frac{3}{4} \\ \frac{5}{4}\end{pmatrix} \; ; \quad
  \text{mit } D^{(n)} = \begin{pmatrix}d_x^{(n)} \\ d_y^{(n)}\end{pmatrix}
\end{align*}
Die Lösung ergibt sich direkt durch Vorwärtseinsetzen: $d_x^{(1)} = \frac{3}{4}$
und $d_y^{(1)} = \frac{1}{28}$. Mit $X^{(1)} = D^{(1)} + X^{(0)}$ ergibt sich dann
für den ersten Newton-Iterationsschritt:
$x^{(1)} = \frac{5}{4}$ und $^{(1)} = \frac{29}{28}$.

\paragraph*{c)}
Alter Code:
\begin{verbatim}
...
tol = 1e-5; Nmax = 50; x = x0;
%
for it = 1:Nmax
  dx = -DF(x(1),x(2))\F(x(1),x(2));
  x = x+dx;
end
\end{verbatim}
Neuer Code:
\begin{verbatim}
...
tol = 1e-5; Nmax = 50; x = x0;
%
it = 0 ;
dx = [1.0 ; 1.0] ;
while (it < NMax) && (norm(dx) > tol))
  it = it + 1 ;
  % dx = -DF(x(1),x(2))\F(x(1),x(2));
  dx = - inv(DF(x(1),x(2))) * F(x(1),x(2));
  x = x+dx;
end
\end{verbatim}


\subsection*{Aufgabe 32}

\paragraph*{a)}

\paragraph*{b)}

\paragraph*{c)}



\end{document}
