\subsection*{Aufgabe 7}
Es sei $A \in \Gl_n (\KK)$: 
\paragraph*{a)}
%reserviert für Annika
TODO






\paragraph*{b)}
zu zeigen: Wenn alle Hauptabschnittsdeterminanten von $A$ verschieden von 0 sind, dann ist die LR-Zerlegung von $A$ ohne Pivotisierung eindeutig.\\
\newline
Aus Aufgabenteil a) folgt, dass \glqq wenn alle Hauptabschnittsdeterminanten von $A$ verschieden von 0 sind, $A$ eine LR-Zerlegung ohne Pivotisierung \underline{besitzt}.\grqq\\
\newline
Wir zeigen nun, dass diese eindeutig ist:\\
\newline
Seien $ A = L_1R_1 $ und $ A = L_2R_2 $ zwei LR-Zerlegungen von A mit $L_1,L_2$ untere $\triangle$-Matrix mit Einsen auf der Diagonalen und $R_1,R_2$ obere $\triangle$-Matrix.
\begin{align}
	 A = L_1R_1 \text{ und } A = L_2R_2 \;\Rightarrow \; L_1R_1 &=  L_2R_2\\
	 \underbrace{(L_2)^{-1}L_1}_{\substack{\text{untere }\triangle\text{-Matrix}}}
	 &=  \underbrace{R_2(R_1)^{-1}}_{\substack{\text{obere }\triangle\text{-Matrix}}} = I
\end{align}
Hinweis zu (2):\\ Die Inverse einer invertierbaren unteren (oberen) $\triangle$-Matrix ist eine untere (obere) $\triangle$-Matrix.\\
\begin{align}
&\Rightarrow \; (L_2)^{-1}L_1 = I \text{ und } R_2(R_1)^{-1}=I\\
&\Rightarrow L_1=L_2 \text{ und } R_2=R_1
\end{align}
Damit folgt die \underline{Eindeutigkeit} der LR-Zerlegung von $A$