\subsection*{Aufgabe 7}
Es sei $A \in \Gl_n (\KK)$: 
\paragraph*{a)}
zu zeigen: $A$ besitzt eine $LR$-Zerlegung ohne Pivotisierung, wenn alle Hauptabschnittsdeterminanten von $A$ verschieden von 0 sind.\\
\newline
\textit{F\"ur den folgenden Aufgabenteil meinen wir mit der Schreibweise $X[k]$ stets eine $(k \times k)$-Teilmatrix von X aus den ersten k-Zeilen und k-Spalten von $X$.\\}
\newline
"$\Rightarrow$"\\
Angenommen $A \in \Gl_n (\KK)$ besitzt eine $LR$-Zerlegung ohne Pivotisierung, mit $L$ untere $\triangle$-Matrix mit Einsen auf der Diagonalen und $R$ obere $\triangle$-Matrix, dann gilt:
\textit{} $A=LR$.
\begin{align}\label{eins}
0 \neq \mathrm{det}(A) = \mathrm{det}(LR) = \mathrm{det}(L)\cdot \mathrm{det}(R)
\end{align}
Da $L$ nach Definition eine untere $\triangle$-Matrix mit Einsen auf der Diagonalen ist, folgt:
\begin{align}
\mathrm{det}(L)=\prod_{i=1}^{n} l_{ii}=1
\end{align} 
Da jede linke obere $(k \times k)$-Teilmatrix von L wieder eine untere $\triangle$-Matrix mit Einsen auf der Diagonalen ist, gilt dann ebenfalls:
\begin{align}\label{drei}
\mathrm{det}(L[k])=\prod_{i=1}^{k} l_{ii}=1
\end{align}
Damit folgt aus \eqref{eins}
\begin{align}
\mathrm{det}(R) \neq 0 \overset{R\text{ obere }\triangle-Matrix}{\Rightarrow} 0 \neq \mathrm{det}(R)=\prod_{i=1}^{n} r_{ii}
\end{align}
Da, das Produkt \"uber alle Diagonalelemente $r_{ii}$ ungleich Null ist, muss jedes Element  $r_{ii} \; \forall i \in\{1,\cdots,n\}$ ungleich Null sein. Damit gilt f\"ur alle $(k \times k)$-Teilmatrizen von R mit  $k \in\{1,\cdots,n\}$ ebenfalls, dass ihre Determiante ungleich Null ist:
\begin{align}\label{fuenf}
\mathrm{det}(R[k])= \prod_{i=1}^{k} r_{ii} \neq 0 \; \forall k \in\{1,\cdots,n\}
\end{align} 
Weiterhin gilt $\forall k \in\{1,\cdots,n\}$:
\begin{align}
(LR)[k]&=L[k]\cdot R[k]\\
\mathrm{det}(A[k])&=\mathrm{det}((LR)[k])= \mathrm{det}(L[k]\cdot R[k])= \mathrm{det}(L[k])\cdot\mathrm{det}(R[k]) \overset{\eqref{drei}}{=} \mathrm{det}(R[k])\overset{\eqref{fuenf}}{\neq}0
\end{align}
Damit sind alle Hauptabschnittsdeterminanten $\left(\mathrm{det}(A[k])\right)$ von $A$ verschieden von Null.\\
\newline
"$\Leftarrow$"\\
Angenommen es seien alle Hauptabschnittsdeterminanten von 0 verschieden, dann...\\
%Rückrichtung fehlt noch
TODO!!!



\paragraph*{b)}
zu zeigen: Wenn alle Hauptabschnittsdeterminanten von $A$ verschieden von 0 sind, dann ist die LR-Zerlegung von $A$ ohne Pivotisierung eindeutig.\\
\newline
Aus Aufgabenteil a) folgt, dass \glqq wenn alle Hauptabschnittsdeterminanten von $A$ verschieden von 0 sind, $A$ eine LR-Zerlegung ohne Pivotisierung \underline{besitzt}.\grqq\\
\newline
Wir zeigen nun, dass diese eindeutig ist:\\
\newline
Seien $ A = L_1R_1 $ und $ A = L_2R_2 $ zwei LR-Zerlegungen von A mit $L_1,L_2$ untere $\triangle$-Matrix mit Einsen auf der Diagonalen und $R_1,R_2$ obere $\triangle$-Matrix.
\begin{align}
	 A = L_1R_1 \text{ und } A = L_2R_2 \;\Rightarrow \; L_1R_1 &=  L_2R_2\\\label{neun}
	 \underbrace{(L_2)^{-1}L_1}_{\substack{\text{untere }\triangle\text{-Matrix}}}
	 &=  \underbrace{R_2(R_1)^{-1}}_{\substack{\text{obere }\triangle\text{-Matrix}}} = I
\end{align}
Hinweis zu \eqref{neun}:\\ Die Inverse einer invertierbaren unteren (oberen) $\triangle$-Matrix ist eine untere (obere) $\triangle$-Matrix.\\
\begin{align}
&\Rightarrow \; (L_2)^{-1}L_1 = I \text{ und } R_2(R_1)^{-1}=I\\
&\Rightarrow L_1=L_2 \text{ und } R_2=R_1
\end{align}
Damit folgt die \underline{Eindeutigkeit} der LR-Zerlegung von $A$.
\begin{flushright}Q.e.d\end{flushright}