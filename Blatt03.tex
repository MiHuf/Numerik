%  DOCUMENT CLASS
\documentclass[11pt]{article}

%PACKAGES
\usepackage[latin1]{inputenc}
\usepackage[ngerman]{babel}
\usepackage[reqno,fleqn]{amsmath}
\setlength\mathindent{10mm}
\usepackage{amssymb}
\usepackage{amsthm}
\usepackage{delarray}
% \usepackage{fancyhdr}
\usepackage{units}
\usepackage{times, eurosym}
\usepackage{verbatim}%F�r Verwendung von multiline Comments mittels \begin{comment}...\end{comment}


% FORMATIERUNG
\usepackage[paper=a4paper,left=25mm,right=25mm,top=25mm,bottom=25mm]{geometry}
\usepackage{array}
\usepackage{fancybox} %zum Einrahmen von Formeln
\setlength{\parindent}{0cm}
\setlength{\parskip}{1mm plus1mm minus1mm}


% PAGESTYLE


%MATH SHORTCUTS
\newcommand{\NN}{\mathbb N}
\newcommand{\RR}{\mathbb R}
\newcommand{\CC}{\mathbb C}
\newcommand{\KK}{\mathbb K}
\newcommand{\eqx}{\overset{!}{=}}
\newcommand{\Det}{\mathrm{Det}}
\newcommand{\Gl}{\mathrm{Gl}}
\newcommand{\diag}{\mathrm{diag}}
\newcommand{\sign}{\mathrm{sign}}
\newcommand{\cond}{\mathrm{cond}_{\| \cdot \|}}
\newcommand{\conda}{\mathrm{cond}_{\| \cdot \|_1}}
\newcommand{\condb}{\mathrm{cond}_{\| \cdot \|_2}}
\newcommand{\condi}{\mathrm{cond}_{\| \cdot \|_\infty}}

\setlength{\extrarowheight}{1ex}
\begin{document}

\begin{center}
\textbf{
�bungen zur Vorlesung Numerische Mathematik, WS 2014/15\\
Blatt 03 zum 11.11.2014\\
}

\begin{tabular}{lll}
& \\
von & Janina Geiser & Mat Nr. 6420269\\
& Michael Hufschmidt & Mat.Nr. 6436122\\
& Farina Ohm & Mat Nr. 6314051\\
& Annika Seidel & Mat Nr. 6420536\\
\\
\hline
\end{tabular}
\end{center}

\subsection*{Aufgabe 7}
Es sei $A \in \Gl_n (\KK)$: 
\paragraph*{a)}
zu zeigen: $A$ besitzt eine $LR$-Zerlegung ohne Pivotisierung, wenn alle Hauptabschnittsdeterminanten von $A$ verschieden von 0 sind.\\
\newline
\textit{F\"ur den folgenden Aufgabenteil meinen wir mit der Schreibweise $X[k]$ stets eine $(k \times k)$-Teilmatrix von X aus den ersten k-Zeilen und k-Spalten von $X$.\\}
\newline
"$\Rightarrow$"\\
Angenommen $A \in \Gl_n (\KK)$ besitzt eine $LR$-Zerlegung ohne Pivotisierung, mit $L$ untere $\triangle$-Matrix mit Einsen auf der Diagonalen und $R$ obere $\triangle$-Matrix, dann gilt:
\textit{} $A=LR$.
\begin{align}
0 \neq \mathrm{det}(A) = \mathrm{det}(LR) = \mathrm{det}(L)\cdot \mathrm{det}(R)
\end{align}
Da $L$ nach Definition eine untere $\triangle$-Matrix mit Einsen auf der Diagonalen ist, folgt:
\begin{align}
\mathrm{det}(L)=\prod_{i=1}^{n} l_{ii}=1
\end{align} 
Da jede linke obere $(k \times k)$-Teilmatrix von L wieder eine untere $\triangle$-Matrix mit Einsen auf der Diagonalen ist, gilt dann ebenfalls:
\begin{align}
\mathrm{det}(L[k])=\prod_{i=1}^{k} l_{ii}=1
\end{align}
Damit folgt aus (1):
\begin{align}
\mathrm{det}(R) \neq 0 \overset{R\text{ obere }\triangle-Matrix}{\Rightarrow} 0 \neq \mathrm{det}(R)=\prod_{i=1}^{n} r_{ii}
\end{align}
Da, das Produkt \"uber alle Diagonalelemente $r_{ii}$ ungleich Null ist, muss jedes Element  $r_{ii} \; \forall i \in\{1,\cdots,n\}$ ungleich Null sein. Damit gilt f\"ur alle $(k \times k)$-Teilmatrizen von R mit  $k \in\{1,\cdots,n\}$ ebenfalls, dass ihre Determiante ungleich Null ist:
\begin{align}
\mathrm{det}(R[k])= \prod_{i=1}^{k} r_{ii} \neq 0 \; \forall k \in\{1,\cdots,n\}
\end{align} 
Weiterhin gilt $\forall k \in\{1,\cdots,n\}$:
\begin{align}
(LR)[k]&=L[k]\cdot R[k]\\
\mathrm{det}(A[k])&=\mathrm{det}((LR)[k])= \mathrm{det}(L[k]\cdot R[k])= \mathrm{det}(L[k])\cdot\mathrm{det}(R[k]) \overset{(3)}{=} \mathrm{det}(R[k])\overset{(5)}{\neq}0
\end{align}
Damit sind alle Hauptabschnittsdeterminanten $\left(\mathrm{det}(A[k])\right)$ von $A$ verschieden von Null.\\
\newline
"$\Leftarrow$"\\
Angenommen es seien alle Hauptabschnittsdeterminanten von 0 verschieden, dann...\\
%Rückrichtung fehlt noch
TODO!!!



\paragraph*{b)}
zu zeigen: Wenn alle Hauptabschnittsdeterminanten von $A$ verschieden von 0 sind, dann ist die LR-Zerlegung von $A$ ohne Pivotisierung eindeutig.\\
\newline
Aus Aufgabenteil a) folgt, dass \glqq wenn alle Hauptabschnittsdeterminanten von $A$ verschieden von 0 sind, $A$ eine LR-Zerlegung ohne Pivotisierung \underline{besitzt}.\grqq\\
\newline
Wir zeigen nun, dass diese eindeutig ist:\\
\newline
Seien $ A = L_1R_1 $ und $ A = L_2R_2 $ zwei LR-Zerlegungen von A mit $L_1,L_2$ untere $\triangle$-Matrix mit Einsen auf der Diagonalen und $R_1,R_2$ obere $\triangle$-Matrix.
\begin{align}
	 A = L_1R_1 \text{ und } A = L_2R_2 \;\Rightarrow \; L_1R_1 &=  L_2R_2\\
	 \underbrace{(L_2)^{-1}L_1}_{\substack{\text{untere }\triangle\text{-Matrix}}}
	 &=  \underbrace{R_2(R_1)^{-1}}_{\substack{\text{obere }\triangle\text{-Matrix}}} = I
\end{align}
Hinweis zu (9):\\ Die Inverse einer invertierbaren unteren (oberen) $\triangle$-Matrix ist eine untere (obere) $\triangle$-Matrix.\\
\begin{align}
&\Rightarrow \; (L_2)^{-1}L_1 = I \text{ und } R_2(R_1)^{-1}=I\\
&\Rightarrow L_1=L_2 \text{ und } R_2=R_1
\end{align}
Damit folgt die \underline{Eindeutigkeit} der LR-Zerlegung von $A$.
\begin{flushright}Q.e.d\end{flushright}

\subsection*{Aufgabe 8}
F�r $x \in \KK^m \; , \; x_1 \ne 0$ sei
$v = \frac{1}{\|x\|_2} + \sign(x_1) \cdot e_1$ und
$Q_v$  die Householder-Speigelung
\footnote{War zwar nicht auf dem �bungsblatt angegeben, aber wir nehmen das mal so an.}
$Q_v = I - \frac{2}{v^T \cdot v} \cdot v \cdot v^T$.

\paragraph*{a)}
Zu zeigen: $\|v\|_2 = \sqrt{ 2 \cdot \left( 1 + \frac{|x|}{\|x\|_2} \right) }$. Es gilt:
\begin{align*}
  \|v\|_2 & = \sqrt{v^T \cdot v} =
  \sqrt{\left( \frac{x^T}{\|x\|_2} + \sign(x_1) \cdot e_1^T \right) \cdot \left( \frac{x}{\|x\|_2} + \sign(x_1) \cdot e_1 \right) } \\
  & = \left( \frac{x^T \cdot x }{\|x\|_2^2} + \sign(x_1) \cdot e_1^T \cdot \frac{x}{\|x\|_2} +
   \frac{x^T}{\|x\|_2} \cdot \sign(x_1) \cdot e_1  +  \sign(x_1) \cdot e_1^T \cdot \sign(x_1) \cdot e_1 \right)^{1/2}\\
  \intertext{Mit $x^T \cdot x = \| x \|_2^2 $  und $e_1^T \cdot x = x \cdot e_1 = x_1$ und
    $\sign(x_1) \cdot x_1 = |x_1|$ ergibt das:}
 \|v\|_2 & = \sqrt {1 +  \frac{|x_1|}{\|x\|_2} + \frac{|x_1|}{\|x\|_2} +
  \underbrace{\sign(x_1) \cdot \sign(x_1) }_{ = +1} \cdot \underbrace{e_1^T \cdot e_1}_{ = 1} } =
   \sqrt{ 2 \cdot \left( 1 + \frac{|x|}{\|x\|_2} \right) }
\end{align*}

\paragraph*{b)}
Michael denkt nach


\subsection*{Aufgabe 9}
Gegeben:
\begin{align*} \label{eq8}
A=
\begin{split}
\begin{pmatrix}
1 & 2 & 1\\
2 & 5 & 2\\
1 & 2 & 10
\end{pmatrix}
\end{split}
\end{align*}
Da die Matrix $A$ symmetrisch und positiv definit ist besitzt sie eine Cholesly-Zerlegung $A = L\cdot L^*$\\
Wir berechnen also $L$ und $L^*$ mithilfe der Cholesky-Zerlegung\\
Dabei gelten folgende Formeln:\\
\begin{align}
i < k \Rightarrow l_{ik} &= 0\\
i = k \Rightarrow l_{kk} &= \sqrt{a_{kk}-\sum_{j=1}^{k-1}|l_{kj}|^2}\\
i > k \Rightarrow l_{ik} &= \frac{1}{l_{kk}}\cdot (a_{ik}-\sum_{j=1}^{k-1}l_{ij}\cdot \overline{l_{kj}})
\end{align}

\underline{Berechnung von $L$:}\\
Wir bestimmen mit Hilfe der Formeln (1) - (3)
\begin{align*} \label{eq8}
L=
\begin{split}
\begin{pmatrix}
l_{11} & 0 & 0\\
l_{21} & l_{22} & 0\\
l_{31} & l_{32} & l_{33}
\end{pmatrix}
\end{split}
\end{align*}
Alle Eintr�ge f�r die $i < k$ gilt wurden bereits auf die 0 gesetzt.\\

$l_{11}: i=k$ es gilt also Formel (2)\\
\begin{align*}
	l_{11} &= \sqrt{a_{kk}-\sum_{j=1}^{k-1}|l_{kj}|^2}
	= \sqrt{a_{11}}
	= \sqrt{1}
	= 1
\end{align*}

$l_{21}: i > k$ es gilt also Formel (3)\\
\begin{align*}
	l_{21} &= \frac{1}{l_{11}}(a_{21}-\sum_{j=1}^{1-1}l_{2j}\cdot \overline{l_{1j}})
	= \frac{1}{l_{11}}\cdot a_{21}
	= \frac{1}{1}\cdot 2
	=2
\end{align*}

$l_{31}: i > k$ es gilt also Formel (3)\\
\begin{align*}
	l_{31} &= \frac{1}{l_{11}}\cdot (a_{31}-\sum_{j=1}^{1-1}l_{3j}\cdot \overline{l_{1j}})
	= \frac{1}{l_{11}}\cdot a_{13}
	= \frac{1}{1}\cdot 1
	= 1
\end{align*}

$l_{22}: i=k$ es gilt also Formel (2)\\
\begin{align*}
	l_{22} &= \sqrt{a_{22}-\sum_{j=1}^{2-1}|l_{2j}|^2}
	= \sqrt{a_{kk}-l_{21}^2}
	= \sqrt{5-2^2}
	= 1
\end{align*}

$l_{32}: i > k$ es gilt also Formel (3)\\
\begin{align*}
	l_{32} &= \frac{1}{l_{22}}\cdot (a_{32}-\sum_{j=1}^{2-1}l_{3j}\cdot \overline{l_{2j}})
	= \frac{1}{l_{22}}\cdot a_{32} - l_{21} \cdot l_{31}
	= \frac{1}{1}\cdot 2 - 2 \cdot 1
	= 0
\end{align*}

$l_{33}: i=k$ es gilt also Formel (2)\\
\begin{align*}
	l_{33} &= \sqrt{a_{33}-\sum_{j=1}^{3-1}|l_{3j}|^2}
	= \sqrt{a_{33}-l_{31}^2-l_{32}^2}
	= \sqrt{10-1^2-0}
	= 3
\end{align*}

Daraus ergibt sich f�r $L$:
\begin{align*}
L=
\begin{split}
\begin{pmatrix}
 1 & 0 & 0\\
 2 & 1 & 0\\
 1 & 0 & 3
\end{pmatrix}
\end{split}
\end{align*}

Aus $L$ l�sst sich also auch $L^*$ ablesen mit:
\begin{align*}
L^*=
\begin{split}
\begin{pmatrix}
 1 & 2 & 1\\
 0 & 1 & 0\\
 0 & 0 & 3
\end{pmatrix}
\end{split}
\end{align*}

$A$ wurde somit mittels der Cholesky-Zerlegung faktorisiert und es gilt:
\begin{align*}
A = L \cdot L^* \Rightarrow
\begin{pmatrix}
1 & 2 & 1\\
2 & 5 & 2\\
1 & 2 & 10
\end{pmatrix}
=
\begin{pmatrix}
 1 & 0 & 0\\
 2 & 1 & 0\\
 1 & 0 & 3
\end{pmatrix}
\cdot
\begin{pmatrix}
 1 & 2 & 1\\
 0 & 1 & 0\\
 0 & 0 & 3
\end{pmatrix}
\end{align*}

\end{document}
