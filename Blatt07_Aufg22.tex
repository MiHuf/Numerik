\subsection*{Aufgabe 22}
gegebene Wertepaare $(0,2), (1,1), (3,2), (4,4)$\\
Berechnung des Interpolationspolynoms $p\in \Pi_3$:

\paragraph*{a)}
Schema von Neville-Aitken:\\
\begin{align*}
&x_0: \quad f_0=P_{0,0}(x)=2\\
&x_1: \quad f_1=P_{1,0}(x)=1\\
&x_2: \quad f_2=P_{2,0}(x)=2\\
&x_3: \quad f_3=P_{3,0}(x)=4
\end{align*}
\begin{align*}
&P_{0,1}(x)=\frac{(x-x_0)P_{1,0}(x)-(x-x_1)P_{0,0}(x)}{x_1-x_0}=\frac{(x-0)\cdot 1 -(x-1)\cdot2}{1-0}= x-2x+2=-x+2\\
&P_{1,1}(x)=\frac{(x-x_1)P_{2,0}(x)-(x-x_2)P_{1,0}(x)}{x_2-x_1}=\frac{(x-1)\cdot 2 -(x-3)\cdot 1}{3-1}= \frac{2x-2-x+3}{2}=\frac{x+1}{2}\\
&P_{2,1}(x)=\frac{(x-x_2)P_{3,0}(x)-(x-x_3)P_{2,0}(x)}{x_3-x_2}=\frac{(x-3)\cdot 4 -(x-4)\cdot 2}{4-3}= 4x-12-2x+8=2x-4\\
\end{align*}
\begin{align*}
P_{0,2}(x)&=\frac{(x-x_0)P_{1,1}(x)-(x-x_2)P_{0,1}(x)}{x_2-x_0}=\frac{(x-0)\cdot (\frac{x+1}{2}) -(x-3) \cdot(-x+2)}{3-0}\\ &=\frac{\frac{x^2+x}{2}-(-x^2+3x+2x-6)}{3}=\frac{1}{2}x^2-\frac{3}{2}x+2\\
P_{1,2}(x)&=\frac{(x-x_1)P_{2,1}(x)-(x-x_3)P_{1,1}(x)}{x_3-x_1}=\frac{(x-1)\cdot (2x-4) -(x-4) \cdot(x+1)}{2}\\ &=\frac{2x^2-6x+4-\frac{x^2}{2}+\frac{3x}{2}+2}{3}=\frac{\frac{3x^2}{2}-\frac{9x}{2}+6}{3}=\frac{1}{2}x^2-\frac{3}{2}x+2
\end{align*}
Da $P_{0,2}=P_{1,2}$ gilt, folgt:
\begin{align*}
\Rightarrow P_{0,3}=\frac{1}{2}x^2-\frac{3}{2}x+2
\end{align*}
Berechnung von $P_{0,3}$ an der Stelle 2:
\begin{align*}
P_{0,3}(2)=\frac{1}{2}\cdot 2^2-\frac{3}{2}\cdot 2+2=2-3+2=1
\end{align*}
\paragraph*{b)}


