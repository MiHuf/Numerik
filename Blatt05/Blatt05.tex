%  DOCUMENT CLASS
\documentclass[11pt]{article}

%PACKAGES
\usepackage[utf8]{inputenc}
\usepackage[ngerman]{babel}
\usepackage[reqno,fleqn]{amsmath}
\setlength\mathindent{10mm}
\usepackage{amssymb}
\usepackage{amsthm}
\usepackage{color}
\usepackage{delarray}
% \usepackage{fancyhdr}
\usepackage{units}
\usepackage{times, eurosym}
\usepackage{verbatim} %Für Verwendung von multiline Comments mittels \begin{comment}...\end{comment}
\usepackage{wasysym} % Für Smileys


% FORMATIERUNG
\usepackage[paper=a4paper,left=25mm,right=25mm,top=25mm,bottom=25mm]{geometry}
\usepackage{array}
\usepackage{fancybox} %zum Einrahmen von Formeln
\setlength{\parindent}{0cm}
\setlength{\parskip}{1mm plus1mm minus1mm}


% PAGESTYLE

%MATH SHORTCUTS
\newcommand{\NN}{\mathbb N}
\newcommand{\RR}{\mathbb R}
\newcommand{\CC}{\mathbb C}
\newcommand{\KK}{\mathbb K}
\newcommand{\U}{\mathbb O}
\newcommand{\eqx}{\overset{!}{=}}
\newcommand{\Det}{\mathrm{Det}}
\newcommand{\Gl}{\mathrm{Gl}}
\newcommand{\diag}{\mathrm{diag}}
\newcommand{\sign}{\mathrm{sign}}
\newcommand{\rang}{\mathrm{rang}}
\newcommand{\cond}{\mathrm{cond}_{\| \cdot \|}}
\newcommand{\conda}{\mathrm{cond}_{\| \cdot \|_1}}
\newcommand{\condb}{\mathrm{cond}_{\| \cdot \|_2}}
\newcommand{\condi}{\mathrm{cond}_{\| \cdot \|_\infty}}
\newcommand{\eps}{\epsilon}

\setlength{\extrarowheight}{1ex}

\begin{document}

\begin{center}
\textbf{
Übungen zur Vorlesung Numerische Mathematik, WS 2014/15\\
Blatt 05 zum 26.11.2014\\
}

\begin{tabular}{lll}
& \\
von & Janina Geiser & Mat Nr. 6420269\\
& Michael Hufschmidt & Mat.Nr. 6436122\\
& Farina Ohm & Mat Nr. 6314051\\
& Annika Seidel & Mat Nr. 6420536\\
\\
\hline
\end{tabular}
\end{center}

\subsection*{Aufgabe 15}



\subsection*{Aufgabe 16}

Es seien $A,M \in \mathbf{Gl}_n(\RR), N \in \RR^{n\times n}$ so dass $A=M-N$, sowie $b \in \RR^n$. Weiterhin sei $T=M^{-1}N$ mit $\rho(T)\ge 1$ und $c=M^{-1}b$.\\
zu zeigen: Es existiert ein Startwert $x^{(0)} \in \RR^n$, so dass die rekursiv definierte Folge $(x^{(k)})$ mit $x^{(k+1)}=Tx^{(k)}+c$ nicht gegen $x=A^{-1}b$ konvergiert.\\
\newline
Beweis:\\
Wir transformieren $y := x - A^{-1} b$, dann gilt $A y = 0 \; \Leftrightarrow \; A x = b$.
Sei $\rho(T)\ge 1$, dann gibt es einen Eigenvektor $v \neq 0$ von $T$ zum Eigenwert $\lambda$ mit $|\lambda|\ge 1$.\\
\newline
Um zu zeigen, dass ein Startwert $x^{(0)}$ existiert, sodass die Folge nicht gegen die Lösung von $x=A^{-1}b$ konvergiert, reicht es also zu zeigen, dass ein $y^{(0)}$ existiert, sodass die Folge nicht gegen die Lösung $y=0$ konvergiert.\\
\newline
Da $\lambda$ möglicherweise komplex ist, ist auch $v$ möglicherweise komplex. Wegen
$|\lambda| > 1$ ist entweder $\Re(\lambda) > 1$ oder $\Im(\lambda) > 1$ (oder beides).
Wir wählen daher den reellen Startvektor
\begin{align*}
  y^{(0)} = \begin{cases}
             \Re(v) \quad \text{falls } &\Re(\lambda) > 1\\
             \Im(v) \quad \text{falls } &\Im(\lambda) > 1\\
           \end{cases} \\
  \lambda_\RR := \begin{cases}
             \Re(\lambda) \quad \text{falls } &\Re(\lambda) > 1\\
             \Im(\lambda) \quad \text{falls } &\Im(\lambda) > 1\\
           \end{cases}
\end{align*}
Dann gilt:
\begin{align*}
&y^{(k+1)}=Ty^{(k)}+ {c}=Ty^{(k)}\\
&\Rightarrow y^{(k)}=\lambda_\RR^kv
\end{align*}
Die Folge konvergiert insbesondere nicht gegen die Lösung $y = 0$.\\
%Und da $A y = 0 \; \Leftrightarrow \; A x = b$. gilt, insbesondere auch nicht gegen die Lösung von $x=A^{-1}b$.
Somit existiert ein $x^{(0)}$ mit $x^{(0)}=v+A^{-1}b$, sodass die Folge $x^{(k)}$ nicht gegen die Lösung $x=A^{-1}b$ konvergiert.
\begin{flushright}
Q.e.d.
\end{flushright}


\subsection*{Aufgabe 17}
Gegeben die symmetrische positiv definite Matrix $A \in \RR^{n \times n}$, ein Vektor
$b \in \RR^n$ und die Funktion
\begin{align}
  \nonumber
  & f : \RR^n \rightarrow \RR \quad \text{mit}\\
  \label{eq-def-f}
  & f(x) = \frac{1}{2} x^T A x - x^T b
\end{align}

\paragraph*{a)}
Zu zeigen: f hat genau ein lokales Minimum $x_*$; dies ist auch ein globales Minimum
und erfüllt $A x_* = b$.

Beweis: \eqref{eq-def-f} lässt sich mit $A = (a_{ij})$ auch schreiben als:
\begin{align}
  \nonumber
  & f(x) =  \frac{1}{2} \sum_{i, j}x_i a_{ij} x_j - \sum_i x_i b_i
  \intertext{damit wird die $k$-te Komponente des Gradienten mit Anwendung der Produktregel:}
  \label{eq-grad1}
  & \nabla_k f(x) = \frac{\partial}{\partial x_k}f(x) =
  \frac{1}{2}\sum_{i,j}\delta_{ik} a_{ij}x_j+\frac{1}{2}\sum_{i,j}x_i a_{ij}\delta_{jk} - b_k =
  \sum_{j} a_{kj} x_j - b_k
  \intertext{und für den Gradienten gilt:}
  \label{eq-grad2}
  & \nabla f = A x - b
\end{align}
Die notwendige Bedingung für ein Mininum ist, dass der Gradient an der Stelle $x_*$
verschwindet, also $\nabla f|_{x = x_*} = 0$, und mit \eqref{eq-grad2}:
$ A x_* - b = 0 \; \Rightarrow \;  x_* = A^{-1} b$. Der Ausdruck existiert,
da $A$ als positiv definite Matrix invertierbar ist.

Für die hinreichende Bedingung dass dass $x_*$ ein Minimum ist,
muss noch gezeigt werden, dass die Hesse-Matrix $H$ an dieser Stelle positiv ist:
\begin{align}
  H_f (x) = (h_{ij}(x)) =  \left( \frac{\partial ^2}{\partial x_i \partial x_j}f(x) \right)
  =  \left( \frac{\partial}{\partial x_i}  \right) \nabla_j f(x)
\end{align}
Mit  \eqref{eq-grad1} ergibt sich $h_{ij} = a_{ij}$, also $H = A$ unabhängig von $x$.
Da $A$ positiv definit ist, ist auch $H$ positiv definit, also ist $f$ konvex, d.h.
$f(x) \ge f(x_*) \; \forall x \in \RR^n$ und somit ist $x_*$ globales Minimum.

\paragraph*{b)}



\end{document}
