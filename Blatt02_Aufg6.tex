\newpage
\subsection*{Aufgabe 6}
Zu zeigen: F�r $ A \in \textbf{Gl}_n (\KK) $ gilt
\begin{align*}
\condb(A)=\sqrt{\frac{\lambda_{max} (A* A)}{\lambda_{min} (A* A)}}
\end{align*}
wobei $\lambda_{max}(A* A)$ und $\lambda_{min}(A* A)$ jeweils der gr��te und kleinste Eigenwert von $A* A$ ist.\\
\newline
Betrachtet man:
\begin{align}
\|A\|_2^{2}&=<A;A>\\
\|A^{-1}\|_2^{2}&=<A^{-1};A^{-1}> 
\end{align}
Aus (17) und (18) die Definitionen des Spektralradius sind woraus folgt:
\begin{align}
<A;A>&=\lambda_{max}(A^* A)\\
<A^{-1};A^{-1}>&=\lambda_{max}(A^{* -1} A^{-1}) 
\end{align}
Aus der Definition der Kondition folgt also:
\begin{align*}
\condb(A)&=\|A\| \cdot \|A^{-1}\| \\
&=\sqrt{\lambda_{max}(A^{*} A) \cdot \lambda_{max}(A^{* -1} A^{-1}) }
\end{align*}

TODO: Michael: Jetzt m�sste man sich nur noch �berlegen, wieso lamdamax(A* -1A) das minimum von A* A ist.