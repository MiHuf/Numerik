\subsection*{Aufgabe 6}
Zu zeigen: F�r $ A \in \Gl_n (\KK) $ gilt
\begin{align*}
\condb(A)=\sqrt{\frac{\lambda_{\max} (A^* \cdot A)}{\lambda_{\min} (A^* \cdot A)}}
\end{align*}
wobei $\lambda_{\max}(A^* \cdot A)$ und $\lambda_{\min}(A^* \cdot A)$ jeweils der
gr��te und kleinste Eigenwert von $A^* \cdot A$ ist.

Die Matrix $A^* \cdot A$ ist hermitesch, denn $(A^* \cdot A)^* = A^* \cdot A$.
Sie hat somit $n$ reelle, nicht-negative Eigenwerte. Da $ A \in \Gl_n (\KK) $,
kann es auch keinen Eigenwert $0$ geben, die Eigenwerte von  $A^* \cdot A$ sind
also positive reelle Zahlen $\lambda_i > 0$. Diese ordnen wir nun so an, dass
$\lambda_1 \ge \lambda_2 \ge \cdots \ge \lambda_n > 0$ gilt. Damit ist
$\lambda_{\max} (A^* \cdot A) = \lambda_1$ und $\lambda_{\min} (A^* \cdot A) = \lambda_n$

Da  $A^* \cdot A$ hermitesch ist, ist $A^* \cdot A$ auch diagonalisierbar, es
gibt also eine unit�re Matrix $U$, so dass\\
$A^* \cdot A = U^* \cdot \diag(\lambda_1 \cdots \lambda_n) \cdot U $ gilt.

Seien $A, B, C$ beliebige invertierbare Matrizen, dann gilt:
$(A \cdot B \cdot C)^{-1} = C^{-1} \cdot B^{-1} \cdot A^{-1}$, denn\\
$(A \cdot B \cdot C)^{-1} \cdot (A \cdot B \cdot C) =
  (C^{-1} \cdot B^{-1} \cdot A^{-1}) \cdot (A \cdot B \cdot C) = I $.
Damit berechnen wir nun
$(A^* \cdot A)^{-1} = (U^* \cdot \diag(\lambda_1 \cdots \lambda_n) \cdot U)^{-1}
= U^{-1} \cdot \diag(\lambda_1 \cdots \lambda_n)^{-1}  \cdot (U^*)^{-1}$, bzw.
wegen der Unitari�t $U = (U^*)^{-1} \; \text{:}
\quad (A^* \cdot A)^{-1} = U^* \cdot \diag(\lambda_1 \cdots \lambda_n)^{-1}  \cdot U$

\begin{align*}
  \diag(\lambda_1 \cdots \lambda_n) =
  \begin{pmatrix}
    \lambda_1 &  & 0\\
      & \ddots &  \\
    0 &   & \lambda_n
  \end{pmatrix} \quad \text{und} \quad
  \diag(\lambda_1 \cdots \lambda_n)^{-1} =
  \begin{pmatrix}
    1 / \lambda_1 & & 0\\
     & \ddots &  \\
    0 &  & 1/ \lambda_n
  \end{pmatrix}
\end{align*}
Denn man sieht leicht, dass
\begin{align} \label{eq17}
  \begin{pmatrix}
    \lambda_1 & & 0\\
    & \ddots & \\
    0 & & \lambda_n
  \end{pmatrix} \cdot
  \begin{pmatrix}
    1 / \lambda_1 & & 0\\
    & \ddots & \\
    0 & & 1 / \lambda_n
  \end{pmatrix}
 =  \begin{pmatrix}
    1 & & 0\\
    & \ddots & \\
    0 & & 1
  \end{pmatrix}
 = I
\end{align}

Die Diagonalmatrizen enthalten die Eigenwerte. Der gr��te Eigenwert von
$(A^* \cdot A)^{-1}$ ist also \\
$\lambda_{\max} (A^* \cdot A) ^{-1} = 1 / \lambda_n$, damit wird
\begin{align*}
\condb(A) & = \|A^{-1}\|_2  \cdot \| A \|_2 =
\sqrt{\lambda_{\max} (A^* \cdot A)^{-1}} \cdot \sqrt{\lambda_{\max} (A^* \cdot A)}
= \sqrt{\frac{1}{\lambda_n}} \cdot \sqrt{\lambda_1} \\
&= \sqrt{\frac{\lambda_1}{\lambda_n}} = \sqrt{\frac{\lambda_{\max}}{\lambda_{\min}}}
\end{align*}
\begin{flushright}Q.e.d.\end{flushright}

Noch zu zeigen: Sei $U$ unit�re Matrix. $A = U^k \; \Leftrightarrow \;  \condb(A) = 1$

"`$\Rightarrow$"': Unit�re Matrizen sind definiert durch $U^* \cdot U = I$. Ihre
Eigenwerte haben stets den Betrag 1, Beweis: Sei $\lambda$ Eigenwert von $U$ zum Eigenbektor $x$,
dann gilt:
\begin{align*}
  & 1 \cdot x = I \cdot x = U^* \cdot U \cdot x = \lambda \cdot  U^* \cdot x =
   \lambda \cdot \overline{\lambda} \cdot x = |\lambda|^2 \cdot x\\
  & \quad \Rightarrow  |\lambda|^2  = 1 \; ; \quad \Rightarrow  |\lambda|  = 1
\end{align*}
Die Matrix $A = U^k$ ist ebenfalls unit�r, denn
\begin{align} \label{eq18}
  A^* \cdot A = (U^*)^k \cdot U^k = (U^*)^{k-1} \cdot U^* \cdot U \cdot U^{k-1}
  = (U^*)^{k-1} \cdot I \cdot U^{k-1} = \dots = I^k = I
\end{align}
Die Matrix $A^* \cdot A$ hat als hermitesche Matrix nur positive reelle Eigenwerte,
da $A$ zudem unit�r ist, haben alle Eigenwerte den Wert 1, d.h. $\lambda_{\max} = 1$.
Die inverse Matrix $(A^* \cdot A)^{-1}$ ist - da $A$ unit�r ist, gleich der Matrix
$(A^* \cdot A)^* = A^* \cdot A$ und hat den gleichen Eigenwert $\lambda_{\min} = 1$.
Also ist $\frac{\lambda_{\max}}{\lambda_{\min}} = 1$ und somit $\condb(A) = 1$.
\begin{flushright}Q.e.d.\end{flushright}

"`$\Leftarrow$"': Sei $ \condb(A) = 1$, dann gilt nach dem Obigen
$\lambda_{\max} (A^* \cdot A) \cdot \lambda_{\max} (A^* \cdot A)^{-1} = 1$.
Die jeweils gr��ten Eigenwerte dieser Matrizen findet man in Gleichung (\ref{eq17}), d.h.
$\lambda_1 \cdot \frac{1}{\lambda_n} = 1 \; \Rightarrow \; \lambda_1 = \lambda_n = 1$.
Daraus folgt $(A^* \cdot A)\; x = 1 \cdot x \; \Rightarrow \; (A^* \cdot A) = I$ und
somit ist $A$ unit�r und wegen Gleichung (\ref{eq18}) gilt das auch f�r $U := \sqrt[k]{A}$.
\begin{flushright}Q.e.d.\end{flushright}
