\subsection*{Aufgabe 35}
Gegeben: Interpolationsproblem mit kubischen Splines und natürlichen Randbedingungen für
die $n + 1$ äquidistanten Stützstellen
$x_j = \frac{j}{n}\; ;\; j = 0, \cdots , n$ im Intervall $[0,1]$ für die Funktion
$f(x) = \sqrt{x}$. Diese hat also die Stützwerte $\left\lbrace f_0, f_1, f_2 , \cdots , f_n \right\rbrace = \left\lbrace 0, \sqrt{\frac{1}{n}}, \sqrt{\frac{2}{n}} , \cdots , 1\right\rbrace$.
Für die Länge der $n$ Intervalle gilt also
$h_j = x_j - x_{j-1} = \frac{1}{n} =: h \; ;\; j = 1, \cdots , n$.
Die Momente $M_j$ sind die zweiten Ableitungen der Splines $s_j$ an den Stützstellen:
$M_j := s_j^{''}(x_j) = s_{j+1}^{''}(x_j) \; ;\; j = 1, \cdots , n-1$ mit den natürlichen
Randbedingungen $M_0 = 0$ und $M_n = 0$.
In der Vorlesung wurde das LGS für die $M_j\; ;\; j = 1, \cdots , n-1$
mit einer $(n-1)\times(n-1)$-Matrix wie folgt hergeleitet:
\begin{align}
 & \begin{pmatrix}
 \label{eq-spline}
   2k_1&h_1& 0 &\cdots&0\\
   h_1&2k_1&h_2& & \vdots\\
  \vdots & \ddots &\ddots&\ddots&\vdots\\
   \vdots & &\ddots&2k_{n-2}&h_{n-2}\\
  0 & \cdots &\cdots&h_{n-2}&2k_{n-1}
  \end{pmatrix} \cdot
  \begin{pmatrix}M_1\\M_2 \\ \vdots \\M_{n-2} \\M_{n-1}\end{pmatrix} =
  \begin{pmatrix}c_1\\c_2 \\ \vdots \\c_{n-2} \\c_{n-1}\end{pmatrix} \\
\intertext{hierbei bedeuten, jeweils für $j = 1, \cdots , n-1$:}
\nonumber
& k_j = h_{j-1} + h_j\\
\nonumber
& c_j = 6 \left(\frac{f_{j+1}- f_j}{h_j} - \frac{f_{j}- f_{j-1}}{h_{j-1}} \right)
\end{align}
Mit den äquidistanten Stützstellen und den Stützwerten der Aufgabenstellung ergibt das:
\begin{align*}
& k_j = \frac{2}{n}\; ; \quad \Rightarrow \; 2 k_j = 4 h\\
& c_j = 6 n \left(f_{j+1} - 2 f_j+ f_{j-1}\right) = 6 \sqrt{n} \left(\sqrt{j+1} - 2 \sqrt{j} + \sqrt{j-1}  \right)
\end{align*}
Der gemeinsame Faktor $h$ lässt sich vor die Matrix ziehen, damit wird \eqref{eq-spline}
nach Division durch $h = \frac{1}{n}$:
\begin{align*}
\begin{pmatrix}
   4 &1 & 0 &\cdots&0\\
   1 &4 &1 & & \vdots\\
  \vdots & \ddots &\ddots&\ddots&\vdots\\
   \vdots & &\ddots&4 &1\\
  0 & \cdots &\cdots& 1&4
\end{pmatrix} \cdot
  \begin{pmatrix}M_1\\M_2 \\ \vdots \\M_{n-2} \\M_{n-1}\end{pmatrix} =
  6 n \sqrt{n}
  \begin{pmatrix}  \sqrt{2} -2\sqrt{1} + \sqrt{0}\\
    \sqrt{3  } - 2\sqrt{2  } + \sqrt{1  } \\ \vdots \\
    \sqrt{n-1} - 2\sqrt{n-2} + \sqrt{n-3} \\
    \sqrt{n  } - 2\sqrt{n-1} + \sqrt{n-2}
  \end{pmatrix}
\end{align*}
Das ist das gesuchte LGS für die $M_j$.
