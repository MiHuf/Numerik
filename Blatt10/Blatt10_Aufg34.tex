\subsection*{Aufgabe 34}

\paragraph*{a)}
Mit der Ergebnissen aus Aufgabe 33 gilt:
\begin{align*}
  \sum_{i = 0}^n (i - n x)^2 B_i^n(x) &= \sum_{i = 0}^n (i^2 - 2 i n x + n^2x^2) B_i^n(x) \\
  & = n^2 \left(\sum_{i = 0}^n \left(\frac{i}{n}\right)^2   B_i^n(x) -
    2 x \sum_{i = 0}^n \left(\frac{i}{n}\right) B_i^n(x) + x^2 \sum_{i = 0}^n B_i^n(x) \right) \\
  & =  n x + n(n-1) x^2 - 2 n^2 x^2 + n^2 x^2 = n x - n x^2 = n x (1 - x) \qquad \Box
\end{align*}

\paragraph*{b)}
zu zeigen:\begin{align*}
\sum_{\underset{|i/n-x|\ge \delta}{i=0}}^n B_i^n(x) \le \frac{1}{4n\delta^2} \qquad \text{für } x \in[0,1], \delta >0
\end{align*}

Beweis:\\\newline
Vorbereitung:
\begin{align}\label{eq_max}
% x(1-x) \le \max_{x\in[0,1]}(x(1-x))=\frac{1}{4}
\max_{x\in[0,1]}(x(1-x))=\frac{1}{4}
\end{align}
Dies gilt da: \begin{align*}
&f(x)=(x(1-x))\\
&\frac{df}{dx}(x)=\frac{d(x(1-x))}{dx}=\frac{d(x-x^2)}{dx}=1-2x\\
&1-2x=0 \Rightarrow 2x=1 \Rightarrow x=\frac{1}{2}, f\left(\frac{1}{2}\right)=\frac{1}{4}
\end{align*}
Weiterhin nutzen wir hier die Erkenntnisse aus Aufgabenteil a)
$ \sum_{i = 0}^n (i - n x)^2 B_i^n(x)  = nx (1-x)$ und kommen damit auf folgende Ungleichung:
\begin{align}
nx (1-x) &=\sum_{i = 0}^n (i - n x)^2 B_i^n(x) %\\
% &=\sum_{i = 0}^n n(i/n - x)n(i/n-x) B_i^n(x)\\ &
=\sum_{i = 0}^n n^2 {\underbrace{(i/n - x )}_{= \delta}}^2 B_i^n(x) \\
&= \sum_{i=0}^n (n\delta)^2 B_i^n(x)
\ge \sum_{\underset{|i/n-x|\ge \delta}{i=0}}^n (n\delta)^2 B_i^n(x)
\end{align}
Nach dem Umformen nach $\sum_{\underset{|i/n-x|\ge \delta}{i=0}}^n B_i^n(x)$ ergibt sich:
\begin{align}
\label{eq-delta}
\sum_{\underset{|i/n-x|\ge \delta}{i=0}}^n B_i^n(x) \le
\frac{nx(1-x)}{(n\delta)^2}\overset{\eqref{eq_max}}{\le}
\frac{n\cdot\frac{1}{4}}{(n\delta)^2}=\frac{n}{4(n\delta)^2}=\frac{1}{4n\delta^2}
\end{align}

\paragraph*{c)}
zu zeigen: Für alle $f \in C[0,1]$ und $\eps >0$ gibt es ein $n\in \NN$, so dass
\begin{align*}\|f-B_nf\|_\infty < \eps \end{align*}

\paragraph*{}
Es gilt:
\begin{align*}
 \|f-B_nf\|_\infty &= \sup_{x \in [0,1]} \left|f(x) - \sum_{i = 0}^n f\left(\frac{i}{n}\right)B_i^n(x) \right| \\
  & = \sup_{x \in [0,1]} \left|  f(x) \underbrace{\sum_{i = 0}^n B_i^n(x)}_{=1} - \sum_{i = 0}^n f\left(\frac{i}{n}\right)B_i^n(x)\right| \\
  & = \sup_{x \in [0,1]} \left|  \sum_{i = 0}^n \left( f(x) - f\left(\frac{i}{n}\right) \right) B_i^n(x) \right|
\intertext{Da $B_i^n(x) \ge 0$ auf $[0,1]$ gilt mit der Dreiecksungleichung:}
  &\le \sup_{x \in [0,1]} \sum_{i = 0}^n  \left| f(x) - f\left(\frac{i}{n}\right) \right| B_i^n(x)
\end{align*}
Wir definieren  nun zwei Teilmengen von $i \in \NN$:
\begin{align*}
  A_n & := \left\lbrace i : 0 \le i \le n ; \left| x - \frac{i}{n} \right| < \delta \right\rbrace \\
  B_n & := \left\lbrace i : 0 \le i \le n ; \left| x - \frac{i}{n} \right| \ge \delta \right\rbrace
\end{align*}
Damit wird:
\begin{align*}
  & \sum_{i = 0}^n  \left| f(x) - f\left(\frac{i}{n}\right) \right| B_i^n(x) =
  \sum_{i \in A_n} \underbrace{\left| f(x) - f\left(\frac{i}{n}\right) \right|}_{=S_A} B_i^n(x) +
  \sum_{i \in B_n} \underbrace{\left| f(x) - f\left(\frac{i}{n}\right) \right|}_{=S_B} B_i^n(x)
\end{align*}
Da $f$ auf $[0,1]$ gleichmäßig stetig ist, gibt es zu jedem $\delta > 0$ ein $\epsilon >0 $
so dass für alle $|x_1 - x_2 | < \delta$ gilt $|f(x_1) - f(x_2) | < \frac{\epsilon}{2}$.
Insbesondere kann man $|x - \frac{i}{n}| < \delta$ wählen und erhält damit $S_A < \frac{\epsilon}{2}$.
Mit der umgekehrten Dreiecksungleichung und $M := \max_{x \in [0,1]} |f(x)|$ kann man $S_B$ abschätzen:
\begin{align*}
  S_B = \left| f(x) - f\left(\frac{i}{n}\right) \right|
  \le  |f(x)| + \left| f\left(\frac{i}{n}\right) \right| \le M + M = 2 M
\end{align*}
Mit Gleichung \eqref{eq-delta} wird dann die Summe
\begin{align*}
  & \sum_{i \in B_n} \left| f(x) - f\left(\frac{i}{n}\right) \right| B_i^n(x) \le
  \sum_{i \in B_n} 2 M B_i^n(x) = 2 M \sum_{i \in B_n} B_i^n(x) \le  \frac{2 M}{4n\delta^2}
\intertext{Insgesamt also:}
 & \sum_{i = 0}^n \left| f(x) - f\left(\frac{i}{n}\right) \right| B_i^n(x) \le \frac{\epsilon}{2} + \frac{M}{2 n\delta^2}
\intertext{und für $n > n_0 = \frac{M}{\epsilon \delta^2}$ wird dann}
 & \sum_{i = 0}^n \left| f(x) - f\left(\frac{i}{n}\right) \right| B_i^n(x) \le \frac{\epsilon}{2} + \frac{\epsilon}{2} = \epsilon
\end{align*}
Also gilt auch
\begin{align*}
\|f-B_nf\|_\infty  \le \epsilon \quad \text{ für }
   n > \max_{x \in [0,1]} |f(x)| \cdot \frac{1}{\epsilon \cdot \delta^2}
\end{align*}
