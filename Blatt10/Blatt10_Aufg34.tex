\subsection*{Aufgabe 34}

\paragraph*{a)}
Mit der Ergebnissen aus Aufgabe 33 gilt:
\begin{align*}
  \sum_{i = 0}^n (i - n x)^2 B_i^n(x) &= \sum_{i = 0}^n (i^2 - 2 i n x + n^2x^2) B_i^n(x) \\
  & = n^2 \left(\sum_{i = 0}^n \left(\frac{i}{n}\right)^2   B_i^n(x) -
    2 x \sum_{i = 0}^n \left(\frac{i}{n}\right) B_i^n(x) + x^2 \sum_{i = 0}^n B_i^n(x) \right) \\
  & =  n x + n(n-1) x^2 - 2 n^2 x^2 + n^2 x^2 = n x - n x^2 = n x (1 - x) \qquad \Box
\end{align*}

\paragraph*{b)}
zu zeigen:\begin{align*}
\sum_{\underset{|i/n-x|\ge \delta}{i=0}}^n B_i^n(x) \le \frac{1}{4n\delta^2} \qquad \text{für } x \in[0,1], \delta >0
\end{align*}

Beweis:\\\newline
Vorbereitung:
\begin{align}\label{eq_max}
x(1-x) \le \max_{x\in[0,1]}(x(1-x))=\frac{1}{4}
\end{align}
Dies gilt da: \begin{align*}
&f(x)=(x(1-x))\\
&\frac{df}{dx}(x)=\frac{d(x(1-x))}{dx}=\frac{d(x-x^2)}{dx}=1-2x\\
&1-2x=0 \Rightarrow 2x=1 \Rightarrow x=\frac{1}{2}, f\left(\frac{1}{2}\right)=\frac{1}{4}
\end{align*}
Weiterhin nutzen wir hier die Erkenntnisse aus Aufgabenteil a)
$ \sum_{i = 0}^n (i - n x)^2 B_i^n(x)  = nx (1-x)$ und kommen damit auf folgende Ungleichung:
\begin{align}
nx (1-x) &=\sum_{i = 0}^n (i - n x)^2 B_i^n(x) \\&=\sum_{i = 0}^n n(i/n - x)n(i/n-x) B_i^n(x)\\&=\sum_{i = 0}^n n^2(i/n - x )^2 B_i^n(x) \\&\ge \sum_{\underset{|i/n-x|\ge \delta}{i=0}}^n (n\delta)^2 B_i^n(x)
\end{align}
Nach dem Umformen nach $\sum_{\underset{|i/n-x|\ge \delta}{i=0}}^n B_i^n(x)$ ergibt sich:
\begin{align}
\sum_{\underset{|i/n-x|\ge \delta}{i=0}}^n B_i^n(x) \le\frac{nx(1-x)}{(nd)^2}\overset{\eqref{eq_max}}{\le} \frac{n\cdot\frac{1}{4}}{(nd)^2}=\frac{n}{4(nd)^2}=\frac{n}{4n\delta^2}
\end{align}
\paragraph*{c)}
zu zeigen: Für alle $f \in C[0,1]$ und $\eps >0$ gibt es ein $n\in \NN$, so dass\begin{align*}
\|f-B_nf\|_\infty < \eps
\end{align*}

TODO Michael