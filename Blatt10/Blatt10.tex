%  DOCUMENT CLASS
\documentclass[11pt]{article}

%PACKAGES
\usepackage[utf8]{inputenc}
\usepackage[ngerman]{babel}
\usepackage[reqno,fleqn]{amsmath}
\setlength\mathindent{10mm}
\usepackage{amssymb}
\usepackage{amsthm}
\usepackage{color}
\usepackage{delarray}
% \usepackage{fancyhdr}
\usepackage{units}
\usepackage{times, eurosym}
\usepackage{verbatim} %Für Verwendung von multiline Comments mittels \begin{comment}...\end{comment}
\usepackage{wasysym} % Für Smileys


% FORMATIERUNG
\usepackage[paper=a4paper,left=25mm,right=25mm,top=25mm,bottom=25mm]{geometry}
\usepackage{array}
\usepackage{fancybox} %zum Einrahmen von Formeln
\setlength{\parindent}{0cm}
\setlength{\parskip}{1mm plus1mm minus1mm}


% PAGESTYLE

%MATH SHORTCUTS
\newcommand{\NN}{\mathbb N}
\newcommand{\ZZ}{\mathbb Z}
\newcommand{\QQ}{\mathbb Q}
\newcommand{\RR}{\mathbb R}
\newcommand{\CC}{\mathbb C}
\newcommand{\KK}{\mathbb K}
\newcommand{\U}{\mathbb O}
\newcommand{\eqx}{\overset{!}{=}}
\newcommand{\Det}{\mathrm{Det}}
\newcommand{\Gl}{\mathrm{Gl}}
\newcommand{\diag}{\mathrm{diag}}
\newcommand{\sign}{\mathrm{sign}}
\newcommand{\rang}{\mathrm{rang}}
\newcommand{\cond}{\mathrm{cond}_{\| \cdot \|}}
\newcommand{\conda}{\mathrm{cond}_{\| \cdot \|_1}}
\newcommand{\condb}{\mathrm{cond}_{\| \cdot \|_2}}
\newcommand{\condi}{\mathrm{cond}_{\| \cdot \|_\infty}}
\newcommand{\eps}{\epsilon}

\setlength{\extrarowheight}{1ex}

\begin{document}

\begin{center}
\textbf{
Übungen zur Vorlesung Numerische Mathematik, WS 2014/15\\
Blatt 10 zum 14.01.2015\\
}

\begin{tabular}{lll}
& \\
von & Janina Geiser & Mat Nr. 6420269\\
& Michael Hufschmidt & Mat.Nr. 6436122\\
& Farina Ohm & Mat Nr. 6314051\\
& Annika Seidel & Mat Nr. 6420536\\
\\
\hline
\end{tabular}
\end{center}

\subsection*{Aufgabe 33}
Gegeben für $i, n \in \NN_0$ mit $i < n$ Bernstein Polynome $B_i^n \in \RR[x]$ mit
\begin{align}
  B_i^n(x) = \binom{n}{i} x^i(1 - x)^{n-i}
\end{align}
und für eine Funktion $f: [0,1] \rightarrow \RR$
\begin{align}
  (B_n f)(x) = \sum_{i = 0}^n f\left(\frac{i}{n}\right) B_i^n(x) =
  \sum_{i = 0}^n f\left(\frac{i}{n}\right) \binom{n}{i} x^i(1 - x)^{n-i}
\end{align}
Berechne $(B_n f)(x)$ für $f_0(x) = x^0, f_1(x) = x^1, f_2(x) = x^2$.

\begin{align*}
  (B_n f_0)(x) &= \sum_{i = 0}^n 1 \cdot \binom{n}{i} x^i(1 - x)^{n-i}
    \; \overset{\text{Binom. Lehrsatz}}{=} \; \left( x + (1 -x) \right)^n = 1^n = 1 \quad \Box\\
  (B_n f_1)(x) &= \sum_{i = 0}^n \frac{i}{n} \; \frac{n!}{i! (n-i)!} x^i(1 - x)^{n-i} =
    \sum_{i = 1}^n \frac{i}{n} \; \frac{n!}{i! (n-i)!} x^i(1 - x)^{n-i} \\
    & = \sum_{i = 1}^n \frac{(n-1)!}{(i-1)! (n-i)!} \cdot x \cdot x^{(i-1)}(1 - x)^{n-i}
    \quad \text{mit $k = i - 1$} \\
    & = x  \cdot \sum_{k = 0}^{n-1} \frac{(n-1)!}{k! (n-1-k)!} x^k(1 - x)^{n-1-k}
    = x \cdot \left( x + (1 -x) \right)^{n-1} = x \cdot 1^{n-1} = x  \quad \Box \\
  (B_n f_2)(x) &= \sum_{i = 0}^n \left(\frac{i}{n}\right)^2 \frac{n!}{i! (n-i)!} x^i(1 - x)^{n-i} =
    \sum_{i = 1}^n \left(\frac{i}{n}\right)^2 \frac{n!}{i! (n-i)!} x^i(1 - x)^{n-i} \\
  & = \frac{1}{n} \sum_{i = 1}^n \left(\frac{i}{n}\right) \frac{n!}{(i-1)! (n-i)!} x \cdot x^{i-1}(1 - x)^{n-i}
  \quad \text{mit $k = i - 1$} \\
   & = \frac{x}{n} \sum_{k = 0}^{n-1} \frac{k+1}{n} \cdot \frac{n!}{k! (n-1-k)!} x^{k}(1 - x)^{n -1-k}\\
   & = \frac{x}{n} \sum_{k = 0}^{n-1} \frac{n-1}{n-1} \cdot (k+1) \cdot \frac{(n-1)!}{k! (n-1-k)!} x^{k}(1 - x)^{n -1-k}\quad \text{mit $m = n - 1$} \\
   & = \frac{x}{n} \left[\sum_{k = 0}^{m} (n-1) \cdot \frac{k+1}{m} \frac{m!}{k! (m-k)!} x^{k}(1 - x)^{m-k} \right]\\
   & = \frac{x}{n} \left[(n-1) \underbrace{\sum_{k = 0}^{m} \frac{k}{m} B_k^m}_{= x} \; + \;
   \underbrace{(n-1)\frac{1}{m}}_{= 1} \underbrace{\sum_{k = 0}^{m} B_k^m}_{= 1}\right]\\
   & = \frac{n-1}{n} x^2   + \frac{x}{n} \quad \Box
\end{align*}



\subsection*{Aufgabe 34}

\paragraph*{a)}

\paragraph*{b)}

\paragraph*{c)}


\subsection*{Aufgabe 35}


\end{document}
