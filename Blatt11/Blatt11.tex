%  DOCUMENT CLASS
\documentclass[11pt]{article}

%PACKAGES
\usepackage[utf8]{inputenc}
\usepackage[ngerman]{babel}
\usepackage[reqno,fleqn]{amsmath}
\setlength\mathindent{10mm}
\usepackage{amssymb}
\usepackage{amsthm}
\usepackage{color}
\usepackage{delarray}
% \usepackage{fancyhdr}
\usepackage{units}
\usepackage{times, eurosym}
\usepackage{verbatim} %Für Verwendung von multiline Comments mittels \begin{comment}...\end{comment}
\usepackage{wasysym} % Für Smileys


% FORMATIERUNG
\usepackage[paper=a4paper,left=25mm,right=25mm,top=25mm,bottom=25mm]{geometry}
\usepackage{array}
\usepackage{fancybox} %zum Einrahmen von Formeln
\setlength{\parindent}{0cm}
\setlength{\parskip}{1mm plus1mm minus1mm}

\allowdisplaybreaks[1]


% PAGESTYLE

%MATH SHORTCUTS
\newcommand{\NN}{\mathbb N}
\newcommand{\ZZ}{\mathbb Z}
\newcommand{\QQ}{\mathbb Q}
\newcommand{\RR}{\mathbb R}
\newcommand{\CC}{\mathbb C}
\newcommand{\KK}{\mathbb K}
\newcommand{\U}{\mathbb O}
\newcommand{\eqx}{\overset{!}{=}}
\newcommand{\Det}{\mathrm{Det}}
\newcommand{\Gl}{\mathrm{Gl}}
\newcommand{\diag}{\mathrm{diag}}
\newcommand{\sign}{\mathrm{sign}}
\newcommand{\rang}{\mathrm{rang}}
\newcommand{\cond}{\mathrm{cond}_{\| \cdot \|}}
\newcommand{\conda}{\mathrm{cond}_{\| \cdot \|_1}}
\newcommand{\condb}{\mathrm{cond}_{\| \cdot \|_2}}
\newcommand{\condi}{\mathrm{cond}_{\| \cdot \|_\infty}}
\newcommand{\eps}{\epsilon}

\setlength{\extrarowheight}{1ex}

\begin{document}

\begin{center}
\textbf{
Übungen zur Vorlesung Numerische Mathematik, WS 2014/15\\
Blatt 11 zum 21.01.2015\\
}

\begin{tabular}{lll}
& \\
von & Janina Geiser & Mat Nr. 6420269\\
& Michael Hufschmidt & Mat.Nr. 6436122\\
& Farina Ohm & Mat Nr. 6314051\\
& Annika Seidel & Mat Nr. 6420536\\
\\
\hline
\end{tabular}
\end{center}

\subsection*{Aufgabe 37}
\paragraph*{a)}

\paragraph*{b)}



\subsection*{Aufgabe 38}
Gegeben seien Stützstellen $a \le x_0 <  x_1 <  \cdots < x_{n-1} < x_n \le b$, eine positive Gewichtsfunktion $ \omega \in L_1[a,b]$ und eine zugehörige Interpolations-Quadraturformel
\begin{align*}
Q_n(f)=\sum_{j=0}^{n}a_jf(x_j)
\end{align*}
Außerdem gilt\\
(i) $\omega(x)=\omega(a+b-x)$ für alle $x \in [a,b]$, und\\
(ii) $x_j-a=b-x_{n-j}$ für alle $j\in\{0,\cdots,n\}$.
\paragraph{a)} zu zeigen: Die Interpolations-Quadraturformel ist symmetrisch; d.h., es gilt $a_j=a_{n-j}$ für $j=\{0,\cdots,n\}$ \\
\newline
Wir definieren $\tilde{Q}_n(f)=\sum_{j=0}^{n}a_{n-j}f(x_j)$. Dann reicht es zu zeigen, dass $\tilde{Q}$ auf $\Pi_n$ exakt ist.\\
\begin{align*}
&\tilde{Q}_n \left[\left(x-\frac{a+b}{2}\right)^i \right]\qquad i=0,\cdots, n\\
&=\sum_{j=0}^{n}a_{n-j}\left(x_j-\frac{a+b}{2}\right)^i\overset{(ii)}{=}\sum_{j=0}^{n}a_{n-j}\left(\frac{a+b}{2}-x_{n-j}\right)^i=Q_n\left[\left(\frac{a+b}{2}-x\right)^i \right]\\
&=\int_a^b \omega(x)\left(\frac{a+b}{2}-x\right)^i dx\\
&\overset{w \text{ symmetrisch}}{=}\int_a^b \omega (x)\left(x-\frac{a+b}{2}\right)^i dx \Rightarrow \tilde{Q_n} \text{ exakt} \Rightarrow a_j=a_{n-j} \text{ für } j=\{0,\cdots,n\} 
\end{align*}
\paragraph{b)} zu zeigen: Ist $n$ gerade, so ist $Q_n$ sogar exakt auf $\Pi_{n+1}$\\
\newline
Wir definieren $R_n(f)=I(f)-Q_n(f)$ mit $n=2m$. Dann reicht es zu zeigen, dass $R_{2m}(x^{2m+1})=0$.\\
Es gilt:\begin{align}
x^{2m+1}=\left(x-\frac{a+b}{2}\right)^{2m+1}+q(x) \text{ mit } q\in \Pi_{2m}
\end{align}
Daher gilt also:\begin{align*}
&R_{2m}\left(x^{2m+1}\right)\\&=R_{2m}\left[\left(x-\frac{a+b}{2}\right)^{2m+1}\right]\\
&=\underbrace{\int_0^{b}\omega (x)\left(x-\frac{a+b}{2}\right)^{2m+1}dx}_{=0}-\sum_{j=0}^{2m}a_j\left(x_j-\frac{a+b}{2}\right)^{2m+1}\\
&=-\sum_{j=1}^{m}\left[a_{m-j}\left(x_{m-j}-\frac{a+b}{2}\right)^{2m+1}+a_{m+j}\left(x_{m+j}-\frac{a+b}{2}\right)^{2m+1}\right]\\
&\underset{a_{m-j}=a_{m+j}}{=}-\sum_{j=1}^{m}\left[a_{m-j}\left(x_{m-j}-\frac{a+b}{2}\right)^{2m+1}+a_{m-j}\left(x_{m+j}-\frac{a+b}{2}\right)^{2m+1}\right]\\
&=-\sum_{j=1}^{m}\left[a_{m-j}\left(x_{m-j}-\frac{a+b}{2}+x_{m+j}-\frac{a+b}{2}\right)^{2m+1}\right]\\
&\underset{x_{m-j}+x_{m+j}=a+b}{=}-\sum_{j=1}^{m}\left[a_{m-j}\left(-\frac{a+b}{2}-\frac{a+b}{2}+a+b\right)^{2m+1}\right]=0\\
&\Rightarrow R_{2m}(x^{2m+1})=0 \\&\Rightarrow Q_n \text{ ist exakt auf } \Pi_{n+1}
\end{align*} 

\subsection*{Aufgabe 39}


\subsection*{Aufgabe 40}
Gesucht ist die Interpolations-Quadraturformel $Q_3(f) = \sum_{j=0}^3 a_j f(x_j)$
mit den Stützstellen:
\begin{align*}
  x_j = a + (j + 1) h\;;\quad 0 \le j \le 3 \;;\quad h = \frac{b-a}{5}
\end{align*}
und mit der Gewichtsfunktion $\omega \equiv 1$. Laut Vorlesung ergeben sich die $a_j$ gemäß
\begin{align*}
  a_j = \int \limits_a^b \omega(x) L_j(x) dx \quad \text{mit} \quad
 L_j(x) = \prod_{\substack{k = 0 \\k \neq j}}^n \frac{x - x_k}{x_j - x_k}
\end{align*}
Um das Integral zu vereinfachen substituieren wir
\begin{align*}
  & x' := \frac{x - a}{b - a} = \frac{x - a}{5 h} \;;\quad x = 5 h x' + a \;;\quad dx = 5 h dx' \;;\quad
  [a,b] \rightarrow [0,1]
\end{align*}
Damit wird:
\begin{align*}
  a_j = \int \limits_0^1 \prod_{\substack{k = 0 \\k \neq j}}^3
  \frac{5 h x' + a - (a + (k+1) h)}{a + (j+1)h - a - (k+1) h } 5 h dx' =
  5 h \int \limits_0^1 \prod_{\substack{k = 0 \\k \neq j}}^3
  \frac{5 x' - k - 1}{j-k} d x'
\end{align*}
Die Stützstellen $x_j$ liegen symmetrisch zum Intervall-Mittelpunkt, es gilt
$x_j + x_{3-j} = a + b$. Nach der vorigen Aufgabe gilt dann auch $a_j = a_{3-j}$,
es genügt also, $a_0$ und $a_1$ zu bestimmen:
\begin{align*}
  & a_0 = a_3 = (b-a) \int \limits_0^1 \left(\frac{5x'-2}{0-1} \cdot \frac{5x'-3}{0-2} \cdot \frac{5x'-4}{0-3} \right) dx' \\
  & a_1 = a_2 = (b-a) \int \limits_0^1 \left(\frac{5x'-1}{1-0} \cdot \frac{5x'-3}{1-2} \cdot \frac{5x'-4}{1-3} \right) dx'
\end{align*}
Nach Auswerten der Integrale erhält man die gesuchten Koeffizienten.







\end{document}
