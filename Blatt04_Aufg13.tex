\subsection*{Aufgabe 13}

\paragraph*{a)}
Zu zeigen: Für $\epsilon > 0$ kann das regularisierte lineare Ausgleichsproblem\\
\newline
"Finde $x_\eps \in \KK^n$, so dass $\mathrm{\| Ax_\eps - b \|_2^2} + \eps\mathrm{\| x_\eps \|_2^2} = \min_{x_\eps \in \KK^n}\mathrm{\| Ax - b \|_2^2} + \eps\mathrm{\| x \|_2^2}$"  durch\\
\newline
"Finde $x_\eps \in \KK^n$, so dass $\mathrm{\| A_\eps x_\eps - b_\eps \|_2^2} = \min_{x_\eps \in \KK^n}\mathrm{\| A_\eps x - b_\eps \|_2^2}$" (*) beschrieben werden.\\
\newline
Durch Nullsetzen des Gradienten
\begin{align*}
&2A^T(Ax_\eps-b)+2\eps x_\eps = 0\\
\Leftrightarrow &2A^TAx_\eps -2A^Tb +2\eps x_\eps = 0\\
\Leftrightarrow &A^TAx_\eps -A^Tb +\eps x_\eps = 0\\
\Leftrightarrow &(A^TA +\eps I)x_\eps-A^Tb  = 0\\
\Leftrightarrow &(A^TA +\eps I) x_\eps = A^Tb\\
\end{align*}
Folgt die Normalengleichung:
\begin{align}(A^TA + \eps I)x_\eps = A^Tb\label{NGL}\end{align}

Das Ausgleichsproblem ist genau dann eindeutig lösbar, wenn $rang(A+\eps I)=n$ gilt.
Durch die Regularisierung von A zu $\tilde{A}= A +\eps I$ mit $\eps>0$, also der Suche einer Lösung vom Gleichungssystem $(A^TA+\eps I)x=A^Tb$ (äquivalent zu *) an Stelle von $Ax=b$ erreicht man die gerade diese Invertierbarkeit (Stichwort \textit{Tichonow-Regularisierung}).\\
Da $(A^TA + \eps I)$ immer quadratisch und für $\eps > 0$ nie singulär ist, existiert also eine Inverse und damit immer eine eindeutige Lösung:
\begin{align}x_\eps = (A^TA + \eps I)^{-1}A^Tb\label{xeps}\end{align}


\paragraph*{b)}
Zu zeigen:
\begin{align*}\lim\limits_{\eps \rightarrow 0}{x_\eps} = x_{**}\end{align*}
Beweis:
\begin{align}
\lim\limits_{\eps \rightarrow 0}{x_\eps}\overset{\eqref{xeps}}{=}\lim\limits_{\eps \rightarrow 0}((A^TA + \eps I)^{-1}A^Tb)=(A^TA)^{-1}A^Tb=A^{-1}(A^{T})^{-1}A^Tb=A^{-1}b
\end{align}
Für $\eps\rightarrow 0$, existiert also eine eindeutige Lösung $x_\eps=x_*=A^{-1}b$ des Gleichungssystems $\|Ax-b\|_2^{2}$. Wenn diese eindeutig ist, ist sie auch minimal und somit $\lim\limits_{\eps \rightarrow 0}{x_\eps}=x_*=x_{**}=A^{-1}b$.
%ich weiß nicht ganz ob das so passt, aber im Großen und Ganzen fibt es Sinn


\paragraph*{c)}
Zu zeigen:\\
$(A^TA + \eps I)x_\eps = A^Tb$ führt mit $rang(A) = n$ zu einer besser konditionierten Matrix als die Normalengleichung des ursprünglichen Problems.\\
\newline
Die Matrix $A^TA$ genau dann regulär, wenn die Matrix $A$ den Rang n besitzt. Damit ist das Ausgleichsproblem genau dann eindeutig lösbar, wenn $rang(A) =n$ gilt.
%dann versuche ich nochmal mein Bestes an c) aber keine Ahnung ob das was wird