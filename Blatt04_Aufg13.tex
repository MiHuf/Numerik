\subsection*{Aufgabe 13}

\paragraph*{a)}
Zu zeigen: Für $\epsilon > 0$ kann das regularisierte lineare Ausgleichsproblem\\
\newline
Finde $x_\eps \in \KK^n$, so dass $\mathrm{\| Ax_\eps - b \|_2^2} + \eps\mathrm{\| x_\eps \|_2^2} = \min_{x_\eps \in \KK^n}\mathrm{\| Ax - b \|_2^2} + \eps\mathrm{\| x \|_2^2}$ durch\\
\newline
Finde $x_\eps \in \KK^n$, so dass $\mathrm{\| A_\eps x_\eps - b_\eps \|_2^2} = \min_{x_\eps \in \KK^n}\mathrm{\| A_\eps x - b_\eps \|_2^2}$ beschrieben werden.\\
\newline
Durch Nullsetzen des Gradienten\\
$2A^T(Ax_\eps-b)+2\eps x_\eps = 0$\\
Folgt die Normalengleichung:\\
$(A^TA + \eps I)x_\eps = A^Tb$\\

Da $(A^TA + \eps I)$ immer quadratisch und für $\eps > 0$ nie singulär ist, existiert ein Inverses und damit immer eine eindeutige Lösung:\\
$x_\eps = (A^TA + \eps I)^{-1}A^Tb$


\paragraph*{b)}
Zu zeigen:\\
$\lim\limits_{\eps \rightarrow 0}{x_\eps} = x_{**}$\\
\newline
...Iiiiich hab keine Ahnung :D
%ich versuche das mal



\paragraph*{c)}
Zu zeigen:\\
$(A^TA + \eps I)x_\eps = A^Tb$ führt mit $rang(A) = n$ zu einer besser konditionierten Matrix als die Normalengleichung des ursprünglichen Problems.\\
\newline
Die Matrix $A^TA$ genau dann regulär, wenn die Matrix $A$ den Rang n besitzt. Damit ist das Ausgleichsproblem genau dann eindeutig lösbar, wenn $rang(A) =n$ gilt.
