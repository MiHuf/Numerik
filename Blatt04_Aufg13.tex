\subsection*{Aufgabe 13}

\paragraph*{a)}
Zu zeigen: Für $\epsilon > 0$ kann das regularisierte lineare Ausgleichsproblem\\
\newline
"`Finde $x_\eps \in \KK^n$, so dass $\mathrm{\| Ax_\eps - b \|_2^2} + \eps\mathrm{\| x_\eps \|_2^2} = \min_{x_\eps \in \KK^n}\mathrm{\| Ax - b \|_2^2} + \eps\mathrm{\| x \|_2^2}$"'  durch\\
\newline
"`Finde $x_\eps \in \KK^n$, so dass $\mathrm{\| A_\eps x_\eps - b_\eps \|_2^2} = \min_{x_\eps \in \KK^n}\mathrm{\| A_\eps x - b_\eps \|_2^2}$"' (*) beschrieben werden.\\
\newline
Durch Nullsetzen des Gradienten ergibt sich:
\begin{align*}
&2A^T(Ax_\eps-b)+2\eps x_\eps = 0\\
\Leftrightarrow &2A^TAx_\eps -2A^Tb +2\eps x_\eps = 0\\
\Leftrightarrow &A^TAx_\eps -A^Tb +\eps x_\eps = 0\\
\Leftrightarrow &(A^TA +\eps I)x_\eps-A^Tb  = 0\\
\Leftrightarrow &(A^TA +\eps I) x_\eps = A^Tb\\
\end{align*}
Folgt die Normalengleichung:
\begin{align}(A^TA + \eps I)x_\eps = A^Tb\label{NGL}\end{align}

Das Ausgleichsproblem ist genau dann eindeutig lösbar, wenn $\rang(A+\eps I)=n$ gilt.
Durch die Regularisierung von A zu $\tilde{A}= A +\eps I$ mit $\eps>0$, also der Suche einer Lösung vom Gleichungssystem $(A^TA+\eps I)x=A^Tb$ (äquivalent zu *) an Stelle von $Ax=b$ erreicht man die gerade diese Invertierbarkeit (Stichwort \textit{Tichonow-Regularisierung}).\\
Da $(A^TA + \eps I)$ immer quadratisch und für $\eps > 0$ nie singulär ist, existiert also eine Inverse und damit immer eine eindeutige Lösung:
\begin{align}x_\eps = (A^TA + \eps I)^{-1}A^Tb\label{xeps}\end{align}


\paragraph*{b)}
Zu zeigen:
\begin{align*}\lim\limits_{\eps \rightarrow 0}{x_\eps} = x_{**}\end{align*}
Beweis:
\begin{align}
\lim\limits_{\eps \rightarrow 0}{x_\eps}\overset{\eqref{xeps}}{=}\lim\limits_{\eps \rightarrow 0}((A^TA + \eps I)^{-1}A^Tb)=(A^TA)^{-1}A^Tb=A^{-1}(A^{T})^{-1}A^Tb=A^{-1}b
\end{align}
Für $\eps\rightarrow 0$, existiert also eine eindeutige Lösung $x_\eps=x_*=A^{-1}b$ des Gleichungssystems $\|Ax-b\|_2^{2}$. Wenn diese eindeutig ist, ist sie auch minimal und somit $\lim\limits_{\eps \rightarrow 0}{x_\eps}=x_*=x_{**}=A^{-1}b$.
%ich weiß nicht ganz ob das so passt, aber im Großen und Ganzen gibt es Sinn


\paragraph*{c)}
Zu zeigen:\\
$(A^TA + \eps I)x_\eps = A^Tb$ führt mit $\rang(A) = n$ zu einer besser konditionierten Matrix als die Normalengleichung des ursprünglichen Problems.\\
\newline
Wir müssen also zeigen, dass folgende Ungleichung gilt:
\begin{align*}
\text{cond}(\tilde{A})\le \text{cond}(A) \text{ mit } \tilde{A}=\begin{pmatrix}
A\\\sqrt{\eps}\mathbb{I}
\end{pmatrix}
\end{align*}
Seien $\sigma_1,\ldots,\sigma_n$ die nach der Größe geordneten Singulärwerte von A, dann sind  $\sigma_1^{2},\ldots,\sigma_n^{2}$ die Eigenwerte von $A^TA$.\\ Weiterhin seien $\mu_1,\ldots,\mu_n$ die nach der Größe geordneten Singulärwerte von $\tilde{A}.$
Wir betrachten dann:
\begin{align*}
&\text{cond}(A)=\frac{\sigma_1}{\sigma_n}\\
&\text{cond}(\tilde{A})=\frac{\mu_1}{\mu_n}
\end{align*}
Daraus folgt dann mit der Eigenwertgleichung:
\begin{align*}
&\tilde{A}^{T}\tilde{A}x=\mu_i^{2}x \quad |\text{Definition von $\tilde{A}$ einsetzen}\\
&\begin{pmatrix}
A^T&\sqrt{\eps}\mathbb{I}
\end{pmatrix}
\begin{pmatrix}
A\\\sqrt{\eps}\mathbb{I} 
\end{pmatrix}x=\mu_i^{2}x\quad |\text{ausmultiplizieren}\\
&(\underbrace{A^TA}_{=\sigma_i^{2}}+\eps\mathbb{I})x=\mu_i^2x \Rightarrow \sigma_i^{2}x+\eps x = \mu_i^{2}x\\
&\Rightarrow \mu_i=\sqrt{\sigma_i^{2}+\eps}\\
&\Rightarrow \text{cond}(\tilde{A})= \frac{\sqrt{\sigma_1^{2}+\eps}}{\sqrt{\sigma_n^{2}+\eps}}=\sqrt{\frac{\sigma_1^{2}+\eps}{\sigma_n^{2}+\eps}} \le \frac{\sigma_1}{\sigma_n}=\text{cond}(A)
\end{align*}
Die Matrix $\tilde{A}$, die sich aus der Normalengleichung des regularisierten Ausgleichsproblems ergibt, ist also besser konditioniert als die Matrix der Normalengleichung des ursprünglichen Problems.
\begin{flushright}
	Q.e.d
\end{flushright}
%Problem an folgenden Beweis ist er, beweist das Gegenteil, finde den Fehler nicht, aber eigentlich ist er schöner als der obrige

%Betrachten wir nun die Konditionen der Matrizen $A^TA$ und $A^TA+\eps \mathbb{I}$ aus den Normalengleichungen:\
%\begin{align}
%&\kappa(A^TA)=\frac{\sigma_1^2}{\sigma_n^2}
%\end{align}
%Weiterhin betrachten wir die Eigenwerte von $(A^TA+\eps\mathbb{I})^*(A^TA+\eps\mathbb{I})$:
%\begin{align*}
% &(A^TA+\eps\mathbb{I})^*(A^TA+\eps\mathbb{I})x=((\underbrace{\eps\mathbb{I})^*(A^TA)}_{\eps\sigma^2}+\underbrace{(A^TA)^*(A^TA)}_{\sigma^4}+\underbrace{(\eps\mathbb{I})(A^TA)^*}_{\eps\sigma^2}+\underbrace{(\eps\mathbb{I})^*(\eps\mathbb{I})}_{\eps^2})x\\
% &=\sigma^4+2\eps\sigma^2+\eps^2=(\sigma^2+\eps)^2
%\end{align*}
%Dann gilt für die Matrix der Normalengleichung des regularisierten Ausgleichsproblems:
%\begin{align}
%&\kappa(A^TA+\eps\mathbb{I})=\frac{\sqrt{(\sigma_1^2+\eps)^2}}{\sqrt{(\sigma_n^2+\eps)^2}}=\frac{(\sigma_1^2+\eps)}{(\sigma_n^2+\eps)}
%\end{align}

