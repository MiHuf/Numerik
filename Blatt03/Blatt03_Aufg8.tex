\subsection*{Aufgabe 8}
F�r $x \in \KK^m \; , \; x_1 \ne 0$ sei
$v = \frac{x}{\|x\|_2} + \sign(x_1) \cdot e_1$ und
$Q_v$  die Householder-Spiegelung
\footnote{War zwar nicht auf dem �bungsblatt angegeben, aber wir nehmen das mal so an.} \\
$Q_v = I - \frac{2}{v^T \cdot v} \cdot v \cdot v^T$.

\paragraph*{a)}
Zu zeigen: $\|v\|_2 = \sqrt{ 2 \cdot \left( 1 + \frac{|x_1|}{\|x\|_2} \right) }$. Es gilt:
\begin{align}
\nonumber
\|v\|_2 & = \sqrt{v^T \cdot v} =
  \sqrt{\left( \frac{x^T}{\|x\|_2} + \sign(x_1) \cdot e_1^T \right) \cdot \left( \frac{x}{\|x\|_2} + \sign(x_1) \cdot e_1 \right) } \\
\nonumber
  & = \left( \frac{x^T \cdot x }{\|x\|_2^2} + \sign(x_1) \cdot e_1^T \cdot \frac{x}{\|x\|_2} +
   \frac{x^T}{\|x\|_2} \cdot \sign(x_1) \cdot e_1  +  \sign(x_1) \cdot e_1^T \cdot \sign(x_1) \cdot e_1 \right)^{1/2}\\
  \intertext{Mit $x^T \cdot x = \| x \|_2^2 \; , \; \; e_1^T \cdot x = x^T \cdot e_1 = x_1$ und
    $\sign(x_1) \cdot x_1 = |x_1|$ ergibt das:}
  \label{eq8a}
 \|v\|_2 & = \sqrt {1 +  \frac{|x_1|}{\|x\|_2} + \frac{|x_1|}{\|x\|_2} +
  \underbrace{\sign(x_1) \cdot \sign(x_1) }_{ = +1} \cdot \underbrace{e_1^T \cdot e_1}_{ = 1} } =
   \sqrt{ 2 \cdot \left( 1 + \frac{|x_1|}{\|x\|_2} \right) }
\end{align}
F�r $x \in \KK^m\setminus\{0\} \text{ mit } x_1 = 0$ ergibt sich dann
mit Gleichung \eqref{eq8a} $\|v\|_2 = \sqrt{2}$.

\paragraph*{b)} Zu zeigen: $Q_v \cdot x = - \sign(x_1) \cdot \|x\|_2 \cdot e_1$
\begin{align}
  \nonumber
  Q_v \cdot x & = \left( I - \frac{2}{v^T \cdot v} \cdot v \cdot v^T \right) \cdot x
  \overset{\text{Gleichung \eqref{eq8a}}}{=}
  \left( I - \frac{2}{2 \left( 1 + \frac{|x_1|}{\|x\|_2}\right)} \cdot v \cdot v^T \right) \cdot x \\
  \label{eq8b}
  & = x - \frac{1}{1 + \frac{|x_1|}{\|x\|_2} } \cdot v \cdot v^T  \cdot x \\
  \nonumber
  v \cdot v^T & = \left( \frac{x}{\|x\|_2} + \sign(x_1) \cdot e_1 \right) \cdot
    \left( \frac{x^T}{\|x\|_2} + \sign(x_1) \cdot e_1^T \right) \\
  \nonumber
  & = \underbrace{\frac{x \cdot x^T}{\|x\|_2^2}}_{= 1} + \;  \sign(x_1)\cdot e_1 \cdot  \frac{x^T}{\|x\|_2} +
     \frac{x}{\|x\|_2}\cdot  \sign(x_1) \cdot e_1^T +
     \underbrace{\sign(x_1)\cdot \sign(x_1)}_{ = 1}\cdot  e_1 \cdot e_1^T   \\
  \nonumber
  & = 1 + \frac{\sign(x_1)}{\|x\|_2} \cdot (e_1 \cdot x^T + x \cdot e_1^T) + e_1 \cdot e_1^T \\
  \nonumber
   v \cdot v^T \cdot x & = x + \frac{\sign(x_1)}{\|x\|_2} \cdot e_1 \underbrace{\cdot x^T \cdot x}_{= \|x\|_2^2}
   + \frac{\sign(x_1)}{\|x\|_2} \cdot x \cdot \underbrace{e_1^T \cdot x}_{= x_1}
     + e_1 \cdot \underbrace{e_1^T \cdot x}_{= x_1} \\
  \nonumber
  & = \left(1 + \frac{\sign(x_1) \cdot x_1}{\|x\|_2} \right) \cdot x + ( \sign(x_1) \cdot \|x\|_2  + x_1 ) \cdot  e_1 \\
  \nonumber
  & = \left(1 + \frac{|x_1|}{\|x\|_2} \right) \cdot x + ( \sign(x_1) \cdot \|x\|_2  + x_1 ) \cdot  e_1
    \quad \text{eingesetzt in \eqref{eq8b} ergibt:} \\
  \nonumber
  Q_v \cdot x & = x - \frac{1}{1 + \frac{|x_1|}{\|x\|_2}} \cdot \left( 1 + \frac{|x_1|}{\|x\|_2} \right)  x -
    \frac{1}{1 + \frac{|x_1|}{\|x\|_2}} \cdot ( \sign(x_1) \cdot \|x\|_2 \cdot e_1 + x_1  \cdot  e_1) \\
  \nonumber
  & =  - \left( \frac{\|x\|_2}{\|x\|_2 + |x_1|} \cdot \sign(x_1) \cdot \|x\|_2 \cdot e_1 +
    \frac{\|x\|_2}{\|x\|_2 + |x_1|} \cdot x_1  \cdot  e_1 \right)
\intertext{mit  $x_1 = \frac{|x_1|}{\sign(x_1)} = |x_1| \cdot \sign(x_1)$ ergibt das:}
  \nonumber
  Q_v \cdot x & = - \left( \frac{\|x\|_2}{\|x\|_2 + |x_1|} \cdot \sign(x_1) \cdot \|x\|_2 \cdot e_1 +
     \frac{|x_1|}{\|x\|_2 + |x_1|} \cdot \sign(x_1) \cdot \|x\|_2 \cdot e_1 \right) \\
  \nonumber
  & = - \left( \frac{\|x\|_2}{\|x\|_2 + |x_1|} +
     \frac{|x_1|}{\|x\|_2 + |x_1|} \right) \cdot \sign(x_1) \cdot \|x\|_2 \cdot e_1 \\
     \label{eq8b2}
     & = - \sign(x_1) \cdot \|x\|_2 \cdot e_1 \qquad \smiley{}
\end{align}
F�r $x = \frac{x}{\|x\|_2} + e_1$ ist offensichtlich $\sign(x_1) = +1$, Gleichung
\eqref{eq8b2} liefert dann $Q_v \cdot x = - \|x\|_2 \cdot e_1$.
