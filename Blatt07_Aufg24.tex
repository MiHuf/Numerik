\subsection*{Aufgabe 24}

\paragraph*{a) Existenz}
Zu zeigen, dass die durch
\begin{align}
  p(x) & = \sum_{k = 0}^n \underbrace{\left(1 - 2 L'_k(x_k)(x - x_k)\right)L_k^2(x)}_{A_k(x)} f_k +
  \sum_{k = 0}^n \underbrace{(x-x_k)L_k^2(x)}_{B_k(x)} f'_k \\
  & =: \sum_{k = 0}^n (A_k(x) f_k + B_k(x) f'_k )
  \intertext {mit den Lagrange-Polynomen}
  L_k(x) & = \prod_{\substack{j = 0\\j \ne k}}^n \frac{x - x_j}{x_k - x_j}
\end{align}
definierte Funktion das Hermite-Interpolationsproblem $p(x_j) = f_j$ und $p'(x_j) = f'_j$ löst.

Beweis: In der Vorlesung wurde gezeigt, dass für die Lagrange-Polynome
$L_k(x_j) = \delta_{kj}$ gilt. Damit wird:
\begin{align*}
  A_k(x_j) &=  \left(1 - 2 L'_k(x_k)(x_j - x_k)\right)L_k^2(x_j)
     = \left(1 - 2 L'_k(x_k)(x_j - x_k)\right)  \delta_{kj} \\
  B_k(x_j) &= (x_j - x_k) L_k^2(x_j) = (x_j - x_k) \delta_{kj}\\
  p(x_j) & = \sum_{k = 0}^n (A_k(x_j) f_k + B_k(x_j) f'_k )\\
    & =\sum_{k = 0}^n \left( \left(1 - 2 L'_k(x_k)(x_j - x_k)\right)  \delta_{kj} f_j +
    (x_j - x_k) \delta_{kj} f'_k \right) \\
   & = (1 - 2 L'_j(x_j)(x_j - x_j) )f_j + (x_j - x_j)f'_j = f_j
\intertext{Die Ableitungen nach der Produktregel ergeben mit viel Rechnerei:}
   A'_k(x) &= -2 L'_k(x_k)L_k^2(x) + \left(1 - 2 L'_k(x_k)(x - x_k)\right)2 L_k(x)L'_k(x) \\
   &= -2 L'_k(x_k)L_k^2(x) + 2 L_k(x)L'_k(x) - 4 L'_k(x_k)(x - x_k) L_k(x)L'_k(x) \\
   A'_k(x_j) &= -2 L'_k(x_k)\delta_{kj} + 2 \delta_{kj}L'_k(x_j) -
   4 L'_k(x_k)(x_j - x_k) \delta_{kj} L'_k(x_j)\\
   \sum_{k = 0}^n A'_k(x_j) & = -2 L'_j(x_j) +  2 L'_j(x_j) + 4 (x_j - x_j) L'^2_j(x_j) = 0 \\
   B'_k(x) &= L_k^2(x) + (x-x_k)L_k(x) L'_k(x) \\
   B'_k(x_j) &= L_k^2(x_j) + (x_j-x_k)L_k(x_j) L'_k(x_j)
     = \delta_{kj} + (x_j-x_k) \delta_{kj} L'_k(x_j)\\
   \sum_{k = 0}^n B'_k(x_j) & = 1 + (x_j-x_j)L'_j(x_j) = 1
\intertext{Also:}
p'(x_j) & = \sum_{k = 0}^n A'_k(x_j) f_j + \sum_{k = 0}^n B'_k(x_j) f'_j
  = 0 f_j + 1 f'_j = f'_j
\end{align*}
Somit löst (1) die Hermite Interpolation, wir haben einen
konstruktiven Existenzbeweis.

\paragraph*{b) Eindeutigkeit}
Seien $p_1, p_2 \in \Pi_{(n+1)^2}$ Lösungen des Hermite-Problems. Dann hat die Funktion
$p_1(x) - p_2(x)$  mindestens $n + 1$ Nullstellen  da
$p_1(x_j) - p_2(x_j) = f_j - f_j = 0 \quad j = 0, \cdots , n$
und ebenso hat die Funktion $p'_1(x) - p'_2(x)$  mindestens $n + 1$ Nullstellen bei
$p'_1(x_j) - p'_2(x_j) = f'_j - f'_j = 0$, das heißt die Nullstellen bei $x_i$ sind
doppelte Nullstellen. Insgesamt hat  $p_1(x) - p_2(x)$ also $(n + 1)^2$ Nullstellen,
also muss $p_1(x) - p_2(x) \equiv 0$ sein.

\paragraph*{c)}
Man lernt in der 10. Klasse die Definition der Ableitung:
\begin{align*}
  f'(x) = \lim_{h \rightarrow 0} \frac{f(x + h) - f(x)}{h} \quad \overset{y = x + h}{=}\quad
  \lim_{y \rightarrow x} \frac{f(y) - f(x)}{y - x} =  \lim_{y \rightarrow x} f[x,y]
\end{align*}

\paragraph*{d)}
Für $n=1$ folgt für das Polynom des Hermite-Interpolationsproblems:
\begin{align*}
	p(x) = f[x_0] + f[x_0,x_0](x-x_0) + f[x_0,x_0,x_1](x-x_0)^2 + f[x_0,x_0,x_1,x_1](x-x_0)^2(x-x_1)
\end{align*}
Explizit in Abhängigkeit von $f_0,f_1,f'_0,f'_1$ und $h_0=x_1-x_0$ folgt somit:
\begin{align*}
	p(x) = f_0 + f'_0(x-x_0) + \frac{(f_1-f_0)-f'_0}{h_0}(x-x_0)^2 +  ((f'_1-2(f_1-f_0)+f'_0)(x-x_0)^2(x-x_1)
\end{align*}
\paragraph*{e)}
Für die Darstellung
\begin{align*}
p(x) = a + b(x-x_0) + c(x-x_0)^2 +  d(x-x_0)^3
\end{align*}
Folgt für die Koeffizienten
\begin{align*}
a &= f_0\\
b &= f'_0\\
c &= \frac{(f_1-f_0)-f'_0}{h_0}\\
d &= ???\\
\end{align*}
