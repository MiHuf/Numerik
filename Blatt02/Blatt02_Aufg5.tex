\newpage
\subsection*{Aufgabe 5}
\paragraph*{a)}
zu zeigen: die Zeilensummennorm wird von der Maximumnorm induziert\\
\newline
$\mathrm{Zeilensummennorm: }\|A\|_\infty=\underset{1 \le i \le m}{\mathrm{max}} \sum_{j=1}^{n}|a_{ij}|$\\
\newline
$\mathrm{Maximumnorm: }\|x\|_\infty=\underset{i}{\mathrm{max}} |x_{i}|$\\
\newline
Die der Maximumnorm $\|x\|_\infty$ zugeordnete Matrixnorm $\|A\|_\infty$ ist gegeben durch:
\begin{align*}
&\|A\|_\infty := \underset{x\neq 0}{\mathrm{max}} \frac{\|Ax\|_\infty}{\|x\|_\infty} = 
\underset{\|x\|_\infty=1}{\mathrm{max}} \|Ax\|_\infty =
\underset{\|x\|_\infty=1}{\mathrm{max}}\left\lbrace \underset{i}{\mathrm{max}}\left|\sum_{j=1}^{n}a_{ij}x_j\right|\right\rbrace\\&=
\underset{i}{\mathrm{max}}\left\lbrace\underset{\|x\|_\infty=1}{\mathrm{max}}\left|\sum_{j=1}^{n} a_{ij}x_j\right|\right\rbrace
\overset{x_j=sign(a_{ij})}{=}\underset{i}{\mathrm{max}}\sum_{j=1}^{n}|a_{ij}|=: \mathrm{Zeilensummennorm} \|A\|_\infty
\end{align*}
\begin{flushright}Q.e.d.\end{flushright}

\paragraph*{b)}
zu zeigen: die Spaltensummennorm wird von der Betragssummennorm induziert\\
\newline
$\mathrm{Spaltensummennorm: }\|A\|_1=\underset{1 \le j \le n}{\mathrm{max}} \sum_{i=1}^{m}|a_{ij}|$\\
\newline
$\mathrm{Betragssummenmnorm: }\|x\|_1=\sum_{i=1}^{m}|x_i|$\\
\newline
Die der Betragssummennorm $\|x\|_1$ zugeordnete Matrixnorm $\|A\|_1$ ist gegeben durch:
\begin{align}
\|A\|_1 := \underset{x\neq 0}{\mathrm{max}} \frac{\|Ax\|_1}{\|x\|_1} = \underset{\|x\|_1=1}{\mathrm{max}} \|Ax\|_1 = \underset{\|x\|_1=1}{\mathrm{max}}\left\lbrace\sum_{i=1}^{m}\left|\sum_{j=1}^{n}a_{ij}x_j\right|\right\rbrace \\
\overset{(*)}{=} \underset{1 \le j \le n}{\mathrm{max}} \sum_{i=1}^{m}|a_{ij}| =: \mathrm{Spaltensummennorm} \|A\|_1
\end{align}
(*)Die Summe im Betrag wird f\"ur festes i f\"ur dasjenige j maximal bei dem $a_{ij}$ maximal ist und $x_j = 1$ ist.
\begin{flushright}Q.e.d.\end{flushright}
