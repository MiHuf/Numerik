\subsection*{Aufgabe 15}
Es sei $A \in \KK^{n\times n}$ Hermitesch und positiv definit.\\
Zu zeigen: Für alle Startwerte $x^{(0)}$ konvergiert das Einzelschrittverfahren.\\
Beweis: Wir zeigen, dass die Iterationsmatrix $T = -(D-L)^{-1}R$ des Einzelschrittverfahrenz bezüglich der Energienorm
\begin{align*}
	\| y \|_A := \sqrt{y^*Ay},\qquad \text{für alle $y \in \KK^n$}
\end{align*}
kontrahiert und damit wegen $\rho(T)<1$ gilt. Wir zeigen, dass $\| Ty \|_A < \| y \|_A$ für alle $y \in \KK^n \ensuremath{\backslash} \{0\}$. Denn das Supremum
\begin{align*}
\| T \|_A := \sup_{y \in \KK^n \ensuremath{\backslash} \{0\}} \frac{\| Ty \|_A}{\| y \|_A}
\end{align*}
wird bei einem $y_0 \neq 0$ angenommen und die Kontraktivität von $T$ folgt dann aus $\| T \|_A = \frac{\| Ty \|_A}{\| y \|_A}$ < 1. Wir rechnen zunächst nach Definition
\begin{align*}
\| y \|_A^2 - \| Ty \|_A^2 = y^*Ay - (Ty)^*ATy = y^*(A-T^*AT)y
\end{align*}
Mit $T = -(D-L)^{-1}R = I-(D_L)^{-1}A, D^* = D und L* = R$ erhält man
\begin{align*}
A-T^*AT &= A - (I-A(D_R)^{-1})A(I-(D-L)^{-1}A)\\
&= A(D-R)^{-1}A+A(D-L)^{-1}A-A(D-R)^{-1}A(D-L)^{-1}A\\
&=A(D-R)^{-1}(I+L(D-L)^{-1})A,
\end{align*}
also
\begin{align*}
\| y \|_A^2 - \| Ty \|_A^2 &= y^*A(D-R)^{-1}D(D-L)^{-1}Ay
&=((D-L)^{-1}Ay)^*D(D-L)^{-1}Ay > 0,
\end{align*}
da die Einträge der Diagonalmatrix $D$ positiv sind und $(D-L)^{-1}Ay \neq 0$ gilt.