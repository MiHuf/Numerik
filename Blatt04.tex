%  DOCUMENT CLASS
\documentclass[11pt]{article}

%PACKAGES
\usepackage[utf8]{inputenc}
\usepackage[ngerman]{babel}
\usepackage[reqno,fleqn]{amsmath}
\setlength\mathindent{10mm}
\usepackage{amssymb}
\usepackage{amsthm}
\usepackage{color}
\usepackage{delarray}
% \usepackage{fancyhdr}
\usepackage{units}
\usepackage{times, eurosym}
\usepackage{verbatim} %Für Verwendung von multiline Comments mittels \begin{comment}...\end{comment}
\usepackage{wasysym} % Für Smileys


% FORMATIERUNG
\usepackage[paper=a4paper,left=25mm,right=25mm,top=25mm,bottom=25mm]{geometry}
\usepackage{array}
\usepackage{fancybox} %zum Einrahmen von Formeln
\setlength{\parindent}{0cm}
\setlength{\parskip}{1mm plus1mm minus1mm}


% PAGESTYLE

%MATH SHORTCUTS
\newcommand{\NN}{\mathbb N}
\newcommand{\RR}{\mathbb R}
\newcommand{\CC}{\mathbb C}
\newcommand{\KK}{\mathbb K}
\newcommand{\eqx}{\overset{!}{=}}
\newcommand{\Det}{\mathrm{Det}}
\newcommand{\Gl}{\mathrm{Gl}}
\newcommand{\diag}{\mathrm{diag}}
\newcommand{\sign}{\mathrm{sign}}
\newcommand{\cond}{\mathrm{cond}_{\| \cdot \|}}
\newcommand{\conda}{\mathrm{cond}_{\| \cdot \|_1}}
\newcommand{\condb}{\mathrm{cond}_{\| \cdot \|_2}}
\newcommand{\condi}{\mathrm{cond}_{\| \cdot \|_\infty}}
\newcommand{\eps}{\epsilon}

\setlength{\extrarowheight}{1ex}

\begin{document}

\begin{center}
\textbf{
Übungen zur Vorlesung Numerische Mathematik, WS 2014/15\\
Blatt 04 zum 19.11.2014\\
}

\begin{tabular}{lll}
& \\
von & Janina Geiser & Mat Nr. 6420269\\
& Michael Hufschmidt & Mat.Nr. 6436122\\
& Farina Ohm & Mat Nr. 6314051\\
& Annika Seidel & Mat Nr. 6420536\\
\\
\hline
\end{tabular}
\end{center}

\subsection*{Aufgabe 11}


\subsection*{Aufgabe 12}
Zu zeigen: Sei $\left<\cdot , \cdot \right>: \KK\times\KK \rightarrow \KK$ ein beliebiges
Skalarprodukt auf $\KK$. Dann gibt es eine positiv definite Matrix $A \in \KK^{n \times n}$,
so dass $\left<x,y\right> = x^* A y$ gilt für alle $x,y \in \KK^n$.

Beweis: Es seien $e_i$ die kanonischen Einheitsvektoren des $\KK^n$, dann lassen
sich $x$ und $y$ schreiben als $x = \sum_{i = 1}^n x_i \cdot e_i$ und
$y = \sum_{j = 1}^n y_j \cdot e_j$ mit $x_i, y_j \in \KK$. Für das Skalarprodukt
gilt dann:
\begin{align*}
  \left<x,y\right> &= \left< \sum_{i = 1}^n x_i \cdot e_i \;,\; \sum_{j = 1}^n y_j \cdot e_j \right> \\
  \intertext{Wegen der Sublinearität des Skalarproduktes im ersten und der Linearität im zweiten Term gilt dafür:}
  \left<x,y\right>  &=  \sum_{i = 1}^n \sum_{j = 1}^n \overline{x_i} \cdot \left<e_i,e_j\right> \cdot  y_j
  =\sum_{i, j = 1}^n  \overline{x_i} \cdot \left<e_i,e_j\right> \cdot y_j\\
  \intertext{Für die gesuchte Matrix $A$ ergibt sich dann:}
  A &= (a_{i,j}) = (\left<e_i,e_j\right>) =
   \begin{pmatrix}
     \left<e_1,e_1\right> & \cdots & \left<e_1,e_n\right>\\
     \vdots & \ddots & \vdots \\
     \left<e_n,e_1\right> & \cdots & \left<e_n,e_n\right>
   \end{pmatrix}
  \\
  \intertext{Das liefert dann:}
  \left<x,y\right>  &=  \overline{(x_1, \cdots, x_n)} \cdot A \cdot
    \begin{pmatrix} y_1 \\ \vdots \\ y_n \end{pmatrix} = x^* \cdot A \cdot y
\end{align*}

Es bleibt noch zu zeigen: $A$ ist hermitesch. Aus der Definition des Skalarproduktes
folgt $\left<x,y\right> = (\left<y,x\right>)^*$, also:
\begin{align*}
\left<x,y\right> &= x^* \cdot A \cdot y\\
   &= (\left<y,x\right>)^* = (y^* \cdot A  \cdot x)^* = x^*  \cdot A^*  \cdot (y^*)^*
   = x^* \cdot A^* \cdot y
\end{align*}
Aus dem Vergleich der beiden Gleichungen folgt $A = A^*$, also ist $A$ hermitesch.

Es bleibt noch zu zeigen: $A$ ist positiv definit. Angenommen $A$ hätte einen
Eigenwert $\lambda = 0$ mit dem Eigenvektor $v \ne 0$, dann wäre
$\left<v,v\right> = v^* A v = \lambda v^* v = 0$ im Widerspruch zur Definition
des Skalarpoduktes (muss positiv definit sein).

Da $A$ positiv definit ist, existiert eine Matrix $\sqrt{A}$ mit $\sqrt{A} \cdot \sqrt{A} = A$
(wurde in der Vorlesung gezeigt) und da $A$ hermitesch ist, ist auch $\sqrt{A}$ hermitesch.
Wie in Aufgabe 11 gezeigt, gibt es dann ein $U \in \U_n$, so dass gilt:
$A^* A = U^* \diag(\lambda_1, \cdots,\lambda_n) U$.

Ab hier kommt noch viel Murks:

Wir zeigen nun, dass $T = \diag(1 / \sqrt{\lambda_1}, \cdots, 1 / \sqrt{\lambda_n)}$ das gewünschte liefert:
\begin{align*}
  \left<T x,T y\right> &= \left<T^* T x, y\right> =  \left<A x, y\right> = (A x)^* A y
  = x^* A^* A y\\
  &= x^* U^* \diag(1 / \sqrt{\lambda_1}, \cdots, 1 / \sqrt{\lambda_n)} U   y \\
  &= x^* y = \left<x, y\right>_\text{Euklid}
\end{align*}










\subsection*{Aufgabe 13}

\paragraph*{a)}
Zu zeigen: Für $\epsilon > 0$ kann das regularisierte lineare Ausgleichsproblem\\
\newline
Finde $x_\eps \in \KK^n$, so dass $\mathrm{\| Ax_\eps - b \|_2^2} + \eps\mathrm{\| x_\eps \|_2^2} = \min_{x_\eps \in \KK^n}\mathrm{\| Ax - b \|_2^2} + \eps\mathrm{\| x \|_2^2}$ durch\\
\newline
Finde $x_\eps \in \KK^n$, so dass $\mathrm{\| A_\eps x_\eps - b_\eps \|_2^2} = \min_{x_\eps \in \KK^n}\mathrm{\| A_\eps x - b_\eps \|_2^2}$ beschrieben werden.\\
\newline
Durch Nullsetzen des Gradienten\\
$2A^T(Ax_\eps-b)+2\eps x_\eps = 0$\\
Folgt die Normalengleichung:\\
$(A^TA + \eps I)x_\eps = A^Tb$\\

Da $(A^TA + \eps I)$ immer quadratisch und für $\eps > 0$ nie singulär ist, existiert ein Inverses und damit immer eine eindeutige Lösung:\\
$x_\eps = (A^TA + \eps I)^{-1}A^Tb$


\paragraph*{b)}
Zu zeigen:\\
$\lim\limits_{\eps \rightarrow 0}{x_\eps} = x_{**}$\\
\newline
...Iiiiich hab keine Ahnung :D



\paragraph*{c)}
Zu zeigen:\\
$(A^TA + \eps I)x_\eps = A^Tb$ führt mit $rang(A) = n$ zu einer besser konditionierten Matrix als die Normalengleichung des ursprünglichen Problems.\\
\newline
Die Matrix $A^TA$ genau dann regulär, wenn die Matrix $A$ den Rang n besitzt. Damit ist das Ausgleichsproblem genau dann eindeutig lösbar, wenn $rang(A) =n$ gilt.


\end{document}
