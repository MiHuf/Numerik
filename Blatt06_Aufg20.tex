\subsection*{Aufgabe 20}
Sei $\Omega=(0,1)^2 \subseteq \RR^2$ und $f \in C(\Omega)$ und die Randwertaufgabe
\begin{align*}
-\Delta u(x,y)&=f(x,y) \text{ für alle }(x,y) \in \Omega\\
u(x,y)&=0 \text{ für alle } (x,y) \in \partial\Omega
\end{align*}
Zur Diskretisierung sei $n\in \NN, h=\frac{1}{n+1}$ und $x_i=i \cdot h, y_j=j\cdot h$ für $i,j=0, \cdots, n+1$.
Ferner verwenden wir die Abkürzungen $u_{i,j} = u(x_i, y_j) \text{ und } f_{i,j}= f(x_i, y_j)$
\paragraph*{a)}
Wir approximieren die Funktion $\Delta u(x_i,y_i)$ durch Taylor-Entwicklung von $u_{i-1,j}, u_{i+1,j}, u_{i,j-1}, u_{i,j+1}$ bis zur zweiten Ordnung.\\
\newline
Wir setzen also $N=2$. Dadurch ergeben sich für ein festes $j$ folgende Taylor-Reihen für $u_{i-1,j}, u_{i+1,j}$:
\begin{align}
&T_2 u((i-1)h, ih)=\sum_{n=0}^2 \left. \frac{d^n u}{dx^n} \right|_{(ih)} \cdot \frac{1}{n!}((i-1)h-ih)^n\\ &=\left.\frac{du}{dx}\right |_{ih} ((i-1)h-ih)+\left.\frac{1}{2}\frac{d^2u}{dx^2}\right|_{ih}((i-1)h-ih)^2+u_{ij}\label{i-1}\end{align}
\begin{align}
&T_2 u((i+1)h, ih)=\sum_{n=0}^2 \left. \frac{d^n u}{dx^n} \right|_{(ih)} \cdot \frac{1}{n!}((i+1)h-ih)^n\\ &=\left.\frac{du}{dx}\right |_{ih} ((i+1)h-ih)+\left.\frac{1}{2}\frac{d^2u}{dx^2}\right|_{ih}((i+1)h-ih)^2+u_{ij}\label{i+1}
\end{align}
Addieren der Terme \eqref{i-1} und \eqref{i+1} ergibt:
\begin{align*}
\Rightarrow u_{i-1,j}+u_{i+1,j}=2u_{ij} + h^2 \frac{d^2u}{dx^2}
\end{align*}
Und weiterhin für ein festes $i$ folgende Taylor-Reihen für $u_{i,j-1}, u_{i,j+1}$:
\begin{align}
&T_2 u((j-1)h, jh)=\sum_{n=0}^2 \left. \frac{d^n u}{dy^n} \right|_{(jh)} \cdot \frac{1}{n!}((j-1)h-jh)^n\\ &=\left.\frac{du}{dy}\right |_{jh} ((j-1)h-jh)+\left.\frac{1}{2}\frac{d^2u}{dy^2}\right|_{jh}((j-1)h-jh)^2+u_{ij}\label{j-1}\end{align}
\begin{align}
&T_2 u((j+1)h, jh)=\sum_{n=0}^2 \left. \frac{d^n u}{dy^n} \right|_{(jh)} \cdot \frac{1}{n!}((j+1)h-jh)^n\\ &=\left.\frac{du}{dy}\right |_{jh} ((j+1)h-jh)+\left.\frac{1}{2}\frac{d^2u}{dy^2}\right|_{jh}((j+1)h-jh)^2+u_{ij}\label{j+1}
\end{align}
Addieren der Terme \eqref{j-1} und \eqref{j+1} ergibt:
\begin{align*}
\Rightarrow u_{i, j-1}+u_{i, j+1}=2u_{ij} + h^2 \frac{d^2u}{dy^2}
\end{align*}
Daraus folgt die Approximation von $\Delta u(x_i,y_i)$:
\begin{align*}
&\Delta u(x_i, y_i)=\frac{d^2}{dx^2}u(x_i,y_j)+\frac{d^2}{dy^2}u(x_i,y_j)=\frac{u_{i-1,j}+u_{i+1,j}-2u_{ij}}{h^2}+\frac{u_{i, j-1}+u_{i, j+1}-2u_{ij}}{h^2}\\
&=\frac{u_{i-1,j}+u_{i+1,j}+u_{i, j-1}+u_{i, j+1}-4u_{ij}}{h^2}
\end{align*}
\paragraph*{b)}
lineares Gleichungssystem herleiten\\
\newline
Vorgehensweise:\\
1.) Wir setzen das Ergebnis aus a) in die Gleichung $-\Delta u(x,y)=f(x,y)$ ein\\
2.) Randbedingungen aufstellen\\
\newline
Das gesuchte lineare Gleichungssystem ist daher gegeben durch:
\begin{align*}
&-\frac{u_{i-1,j}+u_{i+1,j}+u_{i, j-1}+u_{i, j+1}-4u_{ij}}{h^2}=f_{i,j}\\
&u_{0,j}=u_{i,0}=u_{n+1,j}=u_{i,n+1}=0 \quad\text{(Randbedingungen)}
\end{align*}
