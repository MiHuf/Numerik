%  DOCUMENT CLASS
\documentclass[11pt]{article}

%PACKAGES
\usepackage[latin1]{inputenc}
\usepackage[ngerman]{babel}
\usepackage[reqno,fleqn]{amsmath}
\setlength\mathindent{10mm}
\usepackage{amssymb}
\usepackage{amsthm}
\usepackage{fancyhdr}
\usepackage{units}
\usepackage{times, eurosym}

% FORMATIERUNG
\usepackage[paper=a4paper,left=25mm,right=25mm,top=25mm,bottom=25mm]{geometry}
\setlength{\parindent}{0cm}
\setlength{\parskip}{1mm plus1mm minus1mm}

% PAGESTYLE
\pagestyle{fancy}
\setlength\headheight{60pt}
\lhead{Janina Geiser, Mat Nr. 6420269\\Michael Hufschmidt, Mat.Nr.
6436122\\Farina Ohm, Mat Nr. XXX\\Annika Seidel, Mat Nr. YYY}
\rhead{�bungen zur Vorlesung\\Numerische Mathematik\\WS 2014/15\\Blatt 01 zum 20.10.2014}

%MATH SHORTCUTS
\newcommand{\NN}{\mathbb N}
\newcommand{\RR}{\mathbb R}
\newcommand{\CC}{\mathbb C}
\newcommand{\KK}{\mathbb K}
\newcommand{\eqx}{\overset{!}{=}}
\newcommand{\Det}{\mathrm{Det}}


\begin{document}

\subsection*{Aufgabe 1}
Gegeben sind:
\begin{align*}
  x = \begin{pmatrix} x_1 \\ x_2 \end{pmatrix} \; ; \qquad
  A = \begin{pmatrix} 1 & 1 \\ 1 & 1,01\end{pmatrix} \; ; \qquad
  b = \begin{pmatrix} b_1 \\ b_2 \end{pmatrix} =
    \begin{pmatrix} 1000 \\ 1005 \end{pmatrix}
\end{align*}

\paragraph*{Exakte L�sung:}
Die exakte L�sung $x$ des linearen Gleichungssystems $A \cdot x = b$ ermitteln
wir �ber die Cramersche Regel mit den Matrizen:
\begin{align}
  A =  \begin{pmatrix} 1 & 1 \\ 1 & 1,01\end{pmatrix} \; ; \qquad
  A_1 = \begin{pmatrix} b_1 & 1 \\ b_2 & 1,01\end{pmatrix} \; ; \qquad
  A_2 = \begin{pmatrix} 1 & b_1 \\ 1 & b_2\end{pmatrix}
\end{align}
Diese haben die Determinanten:
\begin{align}
  \Det(A) = 1,01 - 1 = \frac{1}{100} \; ; \qquad
  \Det(A_1) = 1,01 \cdot b_1 - b_2 \; ; \qquad
  \Det(A_2) = b_2 - b_1
\end{align}
Das ergibt:
\begin{align} \label{eq1}
\begin{split}
  x_1 &= \frac{\Det(A_1)}{\Det(A)} = 100 \cdot (1,01 \cdot b_1 - b_2) \\
  x_2 &= \frac{\Det(A_2)}{\Det(A)} = 100 \cdot (b_2 - b_1)
\end{split}
\end{align}
Mit dem Zahlenwerten f�r $b$ ergibt das $x_1 = 500$ und $x_2 = 500$.

\paragraph*{L�sung mit ungenauem b:}
F�r fehlerbehaftete $\tilde{b}_i = (1 + \varepsilon_i)\cdot b_i$ ergeben sich
fehlerbehaftete L�sungen $\tilde{x}_i$. Durch Einestzen der $\tilde{b}_i$ in
Gleichung (\ref{eq1}) ergibt sich:
\begin{align} \label{eq2}
\begin{split}
  \tilde{x}_1 &= 100 \cdot (1,01 \cdot \tilde{b}_1 - \tilde{b}_2)
  = 100 \cdot (1,01 \cdot (1 + \varepsilon_1)\cdot b_1 - (1 + \varepsilon_2)\cdot b_2)) \\
  &= x_1 + 100 \cdot (1,01 \cdot \varepsilon_1 \cdot b_1 - \varepsilon_2 \cdot b_2)\\
  \tilde{x}_2 &= 100 \cdot (\tilde{b}_2 - \tilde{b}_1)
  = 100 \cdot ((1 + \varepsilon_2)\cdot b_2 - (1 + \varepsilon_1)\cdot b_1))\\
  &= x_2 + 100 \cdot (\varepsilon_2 \cdot b_2 - \varepsilon_1 \cdot b_1)
\end{split}
\end{align}
Mit dem Zahlenwerten f�r $b$ ergibt das
\begin{align} \label{eq3}
\begin{split}
  \tilde{x}_1 &= x_1 + 101000 \cdot \varepsilon_1 - 100500 \cdot \varepsilon_2\\
  \tilde{x}_2 &= x_2 + 105000 \cdot \varepsilon_2 - 100000 \cdot \varepsilon_1
\end{split}
\end{align}
Da die Vorzeichen der Fehler $\varepsilon_i$ unbekannt sind, k�nnen sie im
ung�nstigen Fall auch verschieden sein; dann addieren sich die beiden Fehler-Terme.
Mit $\varepsilon := \max(|\varepsilon_1 |, |\varepsilon_2 | )$ kann man dann den
"`worst case"' absoluten Fehler $| \tilde{\tilde{x}}_i - x_i |$ absch�tzen:
\begin{align} \label{eq4}
\begin{split}
  | \tilde{\tilde{x}}_1 - x_1 | &\le 101000 \cdot |\varepsilon_1| +
    100500 \cdot |\varepsilon_2| \le 201500 \cdot \varepsilon\\
  | \tilde{\tilde{x}}_2 - x_2 | &\le 100500 \cdot |\varepsilon_1| +
    100000 \cdot |\varepsilon_2|\le 200500 \cdot \varepsilon
\end{split}
\end{align}
Und f�r die relativen Fehler ergibt sich dann mit $x_1 = x_2 = 500$:
\begin{align} \label{eq5}
  \left| \frac{\tilde{\tilde{x}}_1 - x_1}{x_1} \right| \le
     403 \cdot \varepsilon \; ; \qquad
  \left| \frac{\tilde{\tilde{x}}_2 - x_2}{x_2} \right| \le
     401 \cdot \varepsilon
\end{align}

\newpage
\subsection*{Aufgabe 2}

Hier k�nnt ihr Euch austoben


\end{document}
