%  DOCUMENT CLASS
\documentclass[11pt]{article}

%PACKAGES
\usepackage[latin1]{inputenc}
\usepackage[ngerman]{babel}
\usepackage[reqno,fleqn]{amsmath}
\setlength\mathindent{10mm}
\usepackage{amssymb}
\usepackage{amsthm}
% \usepackage{fancyhdr}
\usepackage{units}
\usepackage{times, eurosym}
\usepackage{verbatim}%F�r Verwendung von multiline Comments mittels \begin{comment}...\end{comment}


% FORMATIERUNG
\usepackage[paper=a4paper,left=25mm,right=25mm,top=25mm,bottom=25mm]{geometry}
\setlength{\parindent}{0cm}
\setlength{\parskip}{1mm plus1mm minus1mm}

% PAGESTYLE


%MATH SHORTCUTS
\newcommand{\NN}{\mathbb N}
\newcommand{\RR}{\mathbb R}
\newcommand{\CC}{\mathbb C}
\newcommand{\KK}{\mathbb K}
\newcommand{\eqx}{\overset{!}{=}}
\newcommand{\Det}{\mathrm{Det}}

\newcommand{\cond}{\mathrm{cond}_{\| \cdot \|}}


\begin{document}

\begin{center}
\textbf{
�bungen zur Vorlesung Numerische Mathematik, WS 2014/15\\
Blatt 01 zum 20.10.2014\\
}

\begin{tabular}{lll}
& \\
von & Janina Geiser & Mat Nr. 6420269\\
& Michael Hufschmidt & Mat.Nr. 6436122\\
& Farina Ohm & Mat Nr. 6314051\\
& Annika Seidel & Mat Nr. 6420536\\
\\
\hline
\end{tabular}
\end{center}

\subsection*{Aufgabe 1}
Gegeben sind:
\begin{align*}
  x = \begin{pmatrix} x_1 \\ x_2 \end{pmatrix} \; ; \qquad
  A = \begin{pmatrix} 1 & 1 \\ 1 & 1,01\end{pmatrix} \; ; \qquad
  b = \begin{pmatrix} b_1 \\ b_2 \end{pmatrix} =
    \begin{pmatrix} 1000 \\ 1005 \end{pmatrix}
\end{align*}

\paragraph*{L�sung mit exaktem b:}
Die exakte L�sung $x$ des linearen Gleichungssystems $A \cdot x = b$ ermitteln
wir �ber die Cramersche Regel mit den Matrizen:
\begin{align}
  A =  \begin{pmatrix} 1 & 1 \\ 1 & 1,01\end{pmatrix} \; ; \qquad
  A_1 = \begin{pmatrix} b_1 & 1 \\ b_2 & 1,01\end{pmatrix} \; ; \qquad
  A_2 = \begin{pmatrix} 1 & b_1 \\ 1 & b_2\end{pmatrix}
\end{align}
Diese haben die Determinanten:
\begin{align}
  \Det(A) = 1,01 - 1 = \frac{1}{100} \; ; \qquad
  \Det(A_1) = 1,01 \cdot b_1 - b_2 \; ; \qquad
  \Det(A_2) = b_2 - b_1
\end{align}
Das ergibt:
\begin{align} \label{eq1}
\begin{split}
  x_1 &= \frac{\Det(A_1)}{\Det(A)} = 100 \cdot (1,01 \cdot b_1 - b_2) \\
  x_2 &= \frac{\Det(A_2)}{\Det(A)} = 100 \cdot (b_2 - b_1)
\end{split}
\end{align}
Mit dem Zahlenwerten f�r $b$ ergibt das $x_1 = 500$ und $x_2 = 500$.

\paragraph*{L�sung mit ungenauem b:}
F�r fehlerbehaftete $\tilde{b}_i = (1 + \varepsilon_i)\cdot b_i$ ergeben sich
fehlerbehaftete L�sungen $\tilde{x}_i$. Durch Einestzen der $\tilde{b}_i$ in
Gleichung (\ref{eq1}) ergibt sich:
\begin{align} \label{eq2}
\begin{split}
  \tilde{x}_1 &= 100 \cdot (1,01 \cdot \tilde{b}_1 - \tilde{b}_2)
  = 100 \cdot (1,01 \cdot (1 + \varepsilon_1)\cdot b_1 - (1 + \varepsilon_2)\cdot b_2)) \\
  &= x_1 + 100 \cdot (1,01 \cdot \varepsilon_1 \cdot b_1 - \varepsilon_2 \cdot b_2)\\
  \tilde{x}_2 &= 100 \cdot (\tilde{b}_2 - \tilde{b}_1)
  = 100 \cdot ((1 + \varepsilon_2)\cdot b_2 - (1 + \varepsilon_1)\cdot b_1))\\
  &= x_2 + 100 \cdot (\varepsilon_2 \cdot b_2 - \varepsilon_1 \cdot b_1)
\end{split}
\end{align}
Mit dem Zahlenwerten f�r $b$ ergibt das
\begin{align} \label{eq3}
\begin{split}
  \tilde{x}_1 &= x_1 + 101000 \cdot \varepsilon_1 - 100500 \cdot \varepsilon_2\\
  \tilde{x}_2 &= x_2 + 105000 \cdot \varepsilon_2 - 100000 \cdot \varepsilon_1
\end{split}
\end{align}
Da die Vorzeichen der Fehler $\varepsilon_i$ unbekannt sind, k�nnen sie im
ung�nstigen Fall auch verschieden sein; dann addieren sich die beiden Fehler-Terme.
Mit $\varepsilon := \max(|\varepsilon_1 |, |\varepsilon_2 | )$ kann man dann den
"`worst case"' absoluten Fehler $| \tilde{\tilde{x}}_i - x_i |$ absch�tzen:
\begin{align} \label{eq4}
\begin{split}
  | \tilde{\tilde{x}}_1 - x_1 | &\le 101000 \cdot |\varepsilon_1| +
    100500 \cdot |\varepsilon_2| \le 201500 \cdot \varepsilon\\
  | \tilde{\tilde{x}}_2 - x_2 | &\le 100500 \cdot |\varepsilon_1| +
    100000 \cdot |\varepsilon_2|\le 200500 \cdot \varepsilon
\end{split}
\end{align}
Und f�r die relativen Fehler ergibt sich dann mit $x_1 = x_2 = 500$:
\begin{align} \label{eq5}
  \left| \frac{\tilde{\tilde{x}}_1 - x_1}{x_1} \right| \le
     403 \cdot \varepsilon \; ; \qquad
  \left| \frac{\tilde{\tilde{x}}_2 - x_2}{x_2} \right| \le
     401 \cdot \varepsilon
\end{align}


\subsection*{Aufgabe 2}

\paragraph*{a)} Zu zeigen: Strikt diagonaldominante Matrizen sind invertierbar.

Beweis-Idee: Wir zeigen dass der Kern von $A$  leer ist, d.h $\ker (A) = \emptyset$.
Sei $A \in \KK^{n \times n}$ diagonaldominant ($\KK = \RR$ oder $\KK = \CC$).
Definitionsgem�� ist $\ker (A) :=\{ x \in \KK^n \mid x \ne 0 \text{ und } A \cdot x = 0 \}$.
F�r ein $x \in \ker (A)$, gilt dann also $0 = A \cdot x$, d.h.
\begin{align} \label{eq8}
\begin{split}
  0 &= a_{11} \cdot x_1 + a_{12} \cdot x_2 + \dots + a_{1n} \cdot x_n \\
  0 &= a_{21} \cdot x_1 + a_{22} \cdot x_2 + \dots + a_{2n} \cdot x_n \\
  & \dots \\
  0 &= a_{n1} \cdot x_1 + a_{n2} \cdot x_2 + \dots + a_{nn} \cdot x_n
\end{split}
\end{align}
Aus diesem Gleichungssystem w�hlen wir nun die $i$-te Zeile, wobei $i$ durch
$|x_i| \ge |x_j| \; \forall \; j=\{1, \cdots , n\}$ gegeben ist. Dann wird:
\begin{align}
  \nonumber
  0 &= a_{i1} \cdot x_1 + a_{i2} \cdot x_2 + \dots + a_{in} \cdot x_n  =
  \sum_{j=1}^n a_{ij} \cdot x_j =
   a_{ii} \cdot x_i +\sum_{\substack{j = 1 \\j \neq i}}^n a_{ij} \cdot x_j \\
  \label{eq9}
  & \Rightarrow a_{ii} \cdot x_i = - \sum_{\substack{j = 1 \\j \neq i}}^n a_{ij} \cdot x_j
\end{align}
Dann gilt f�r die Betr�ge von Gleichung (\ref{eq9}):
\begin{align}
  | a_{ii} \cdot x_i | &=
    \left | \sum_{\substack{j = 1 \\j \neq i}}^n a_{ij} \cdot x_j \right |
    \quad \overset{\text{Dreiecks-Ungleichung}}{\le} \quad
    \sum_{\substack{j = 1 \\j \neq i}}^n | a_{ij} \cdot x_j | =
    \sum_{\substack{j = 1 \\j \neq i}}^n | a_{ij} | \cdot  | x_j |
\end{align}
 Wir k�nnen (\ref{eq9}) nun absch�tzen, da $|x_i|$ die betragsm��ig
gr��te Komponente von $x$ ist:
\begin{align}\label{eq10}
  | a_{ii} \cdot x_i | &\le
    \sum_{\substack{j = 1 \\j \neq i}}^n | a_{ij} | \cdot | x_i | =
    |x_i| \cdot \sum_{\substack{j = 1 \\j \neq i}}^n | a_{ij} |
\end{align}
Gleichung (\ref{eq10}) durch $|x_i| \ne 0$ dividiert ergibt:
\begin{align}
  |a_{ii} | &\le \sum_{\substack{j = 1 \\j \neq i}}^n | a_{ij} | \quad
  \overset{\text{strikte Diagonaldominanz}}{<} \quad
  | a_{ii} |
\end{align}
Wir erhalten also: $|a_{ii}|< |a_{ii}|$ -- Widerspruch! Die Annahme $|x_i| \ge 0$ ist
also falsch. Da aber $|x_i| \ge |x_j|\;\forall j=\{1, \cdots , n\}$ gilt, folgt auch
$x_j=0\;\forall j=\{1, \cdots, n\}$ Der Kern von A ist somit trivial und dann ist A invertierbar.

\paragraph*{b)}
Zu zeigen: A hermitesch, diagonaldominant und $a_{11}, \cdots ,a_{nn}$  nicht negativ $\Rightarrow A \ge 0$

Sei $\lambda \in \RR$ ein Eigenwert und $ x \in \RR^n$ Eigenvektor von A, dann gilt $Ax = \lambda x$.
Nomieren wir weiterhin die Einheitsvektoren mit der Maxinumsnorm, sodass $\|x\|_\infty = 1$
und w�hlen $ i \in \{1, \cdots ,n\}$ s.d. $|x_i|=1$, dann gilt:
\begin{align*}
\sum_{j=1}^n a_{ij}x_j = \lambda x_i \Leftrightarrow a_{ii}x_i + \sum_{\substack{j = 1 \\
  j \neq i}}^n a_{ij}x_j = \lambda x_i
\end{align*}
Wir substrahieren auf beiden Seiten der Gleichung $a_{ii} x_i$:
\begin{align*}
\lambda x_i - a_{ii} x_i &= (\lambda - a_{ii})* x_i = \sum_{j=1, j\neq i}^n a_{ij}x_j \\
|\lambda - a_{ii}|* \underset{=1}{|x_i|} &= \left|\sum_{\substack{j = 1 \\j \neq i}}^n a_{ij}x_j \right|
  \le \sum_{\substack{j = 1 \\j \neq i}}^n |a_{ij}|\underset{\le1}{|x_j|}
  \le \sum_{\substack{j = 1 \\j \neq i}}^n |a_{ij}|
\end{align*}
Also folgt daraus: $|\lambda - a_{ii}| \le \sum_{\substack{j = 1 \\j \neq i}}^n |a_{ij}|$.
Mit der Formel f�r die Diagonaldominanz
\begin{align*}
|a_{ii}| \ge \sum_{\substack{j = 1 \\j \neq i}}^n |a_{ij}|
  \Leftrightarrow |a_{ii}| - \sum_{\substack{j = 1 \\j \neq i}}^n |a_{ij}| \ge 0
\end{align*}
folgt dann:
\begin{align*}
0 \le a_{ii}-\sum_{\substack{j = 1 \\j \neq i}}^n |a_{ij}|
  \le \lambda \le a_{ii} + \sum_{\substack{j = 1 \\j \neq i}}^n |a_{ij}|
\end{align*}
Es folgt also $\lambda \ge 0 \Rightarrow$  alle Eigenwerte sind nicht negativ $\Rightarrow A $ ist positiv semidefinit.

\paragraph*{c)}
Zu zeigen: A hermitesch, strikt diagonaldominant und $a_{11}, \cdots ,a_{nn} \text{ positiv}\ \Rightarrow A > 0$

Der Beweis ist analog zu b). Nur folgt aus der strikten Diagonaldominanz
\begin{align*}
|a_{ii}| &> \sum_{\substack{j = 1 \\j \neq i}}^n |a_{ij}| \Leftrightarrow
  |a_{ii}| - \sum_{\substack{j = 1 \\j \neq i}}^n |a_{ij}| > 0\\
\intertext{folgendes:}
0 &< a_{ii}-\sum_{\substack{j = 1 \\j \neq i}}^n |a_{ij}|
  \le \lambda \le a_{ii} + \sum_{\substack{j = 1 \\j \neq i}}^n |a_{ij}|
\end{align*}
Es folgt also $\lambda > 0 \Rightarrow $ alle Eigenwerte sind positiv $\Rightarrow A $ ist positiv definit.


\end{document}
