%  DOCUMENT CLASS
\documentclass[11pt]{article}

%PACKAGES
\usepackage[latin1]{inputenc}
\usepackage[ngerman]{babel}
\usepackage[reqno,fleqn]{amsmath}
\setlength\mathindent{10mm}
\usepackage{amssymb}
\usepackage{amsthm}
\usepackage{fancyhdr}
\usepackage{units}
\usepackage{times, eurosym}
\usepackage{verbatim}%F�r Verwendung von multiline Comments mittels \begin{comment}...\end{comment}


% FORMATIERUNG
\usepackage[paper=a4paper,left=25mm,right=25mm,top=25mm,bottom=25mm]{geometry}
\setlength{\parindent}{0cm}
\setlength{\parskip}{1mm plus1mm minus1mm}

% PAGESTYLE
\pagestyle{fancy}
\setlength\headheight{60pt}
\lhead{Janina Geiser, Mat Nr. 6420269\\Michael Hufschmidt, Mat.Nr.
6436122\\Farina Ohm, Mat Nr. 6314051\\Annika Seidel, Mat Nr. 6420536}
\rhead{�bungen zur Vorlesung\\Numerische Mathematik\\WS 2014/15\\Blatt 01 zum 20.10.2014}

%MATH SHORTCUTS
\newcommand{\NN}{\mathbb N}
\newcommand{\RR}{\mathbb R}
\newcommand{\CC}{\mathbb C}
\newcommand{\KK}{\mathbb K}
\newcommand{\eqx}{\overset{!}{=}}
\newcommand{\Det}{\mathrm{Det}}


\begin{document}

\subsection*{Aufgabe 1}
Gegeben sind:
\begin{align*}
  x = \begin{pmatrix} x_1 \\ x_2 \end{pmatrix} \; ; \qquad
  A = \begin{pmatrix} 1 & 1 \\ 1 & 1,01\end{pmatrix} \; ; \qquad
  b = \begin{pmatrix} b_1 \\ b_2 \end{pmatrix} =
    \begin{pmatrix} 1000 \\ 1005 \end{pmatrix}
\end{align*}

\paragraph*{L�sung mit exaktem b:}
Die exakte L�sung $x$ des linearen Gleichungssystems $A \cdot x = b$ ermitteln
wir �ber die Cramersche Regel mit den Matrizen:
\begin{align}
  A =  \begin{pmatrix} 1 & 1 \\ 1 & 1,01\end{pmatrix} \; ; \qquad
  A_1 = \begin{pmatrix} b_1 & 1 \\ b_2 & 1,01\end{pmatrix} \; ; \qquad
  A_2 = \begin{pmatrix} 1 & b_1 \\ 1 & b_2\end{pmatrix}
\end{align}
Diese haben die Determinanten:
\begin{align}
  \Det(A) = 1,01 - 1 = \frac{1}{100} \; ; \qquad
  \Det(A_1) = 1,01 \cdot b_1 - b_2 \; ; \qquad
  \Det(A_2) = b_2 - b_1
\end{align}
Das ergibt:
\begin{align} \label{eq1}
\begin{split}
  x_1 &= \frac{\Det(A_1)}{\Det(A)} = 100 \cdot (1,01 \cdot b_1 - b_2) \\
  x_2 &= \frac{\Det(A_2)}{\Det(A)} = 100 \cdot (b_2 - b_1)
\end{split}
\end{align}
Mit dem Zahlenwerten f�r $b$ ergibt das $x_1 = 500$ und $x_2 = 500$.

\paragraph*{L�sung mit ungenauem b:}
F�r fehlerbehaftete $\tilde{b}_i = (1 + \varepsilon_i)\cdot b_i$ ergeben sich
fehlerbehaftete L�sungen $\tilde{x}_i$. Durch Einestzen der $\tilde{b}_i$ in
Gleichung (\ref{eq1}) ergibt sich:
\begin{align} \label{eq2}
\begin{split}
  \tilde{x}_1 &= 100 \cdot (1,01 \cdot \tilde{b}_1 - \tilde{b}_2)
  = 100 \cdot (1,01 \cdot (1 + \varepsilon_1)\cdot b_1 - (1 + \varepsilon_2)\cdot b_2)) \\
  &= x_1 + 100 \cdot (1,01 \cdot \varepsilon_1 \cdot b_1 - \varepsilon_2 \cdot b_2)\\
  \tilde{x}_2 &= 100 \cdot (\tilde{b}_2 - \tilde{b}_1)
  = 100 \cdot ((1 + \varepsilon_2)\cdot b_2 - (1 + \varepsilon_1)\cdot b_1))\\
  &= x_2 + 100 \cdot (\varepsilon_2 \cdot b_2 - \varepsilon_1 \cdot b_1)
\end{split}
\end{align}
Mit dem Zahlenwerten f�r $b$ ergibt das
\begin{align} \label{eq3}
\begin{split}
  \tilde{x}_1 &= x_1 + 101000 \cdot \varepsilon_1 - 100500 \cdot \varepsilon_2\\
  \tilde{x}_2 &= x_2 + 105000 \cdot \varepsilon_2 - 100000 \cdot \varepsilon_1
\end{split}
\end{align}
Da die Vorzeichen der Fehler $\varepsilon_i$ unbekannt sind, k�nnen sie im
ung�nstigen Fall auch verschieden sein; dann addieren sich die beiden Fehler-Terme.
Mit $\varepsilon := \max(|\varepsilon_1 |, |\varepsilon_2 | )$ kann man dann den
"`worst case"' absoluten Fehler $| \tilde{\tilde{x}}_i - x_i |$ absch�tzen:
\begin{align} \label{eq4}
\begin{split}
  | \tilde{\tilde{x}}_1 - x_1 | &\le 101000 \cdot |\varepsilon_1| +
    100500 \cdot |\varepsilon_2| \le 201500 \cdot \varepsilon\\
  | \tilde{\tilde{x}}_2 - x_2 | &\le 100500 \cdot |\varepsilon_1| +
    100000 \cdot |\varepsilon_2|\le 200500 \cdot \varepsilon
\end{split}
\end{align}
Und f�r die relativen Fehler ergibt sich dann mit $x_1 = x_2 = 500$:
\begin{align} \label{eq5}
  \left| \frac{\tilde{\tilde{x}}_1 - x_1}{x_1} \right| \le
     403 \cdot \varepsilon \; ; \qquad
  \left| \frac{\tilde{\tilde{x}}_2 - x_2}{x_2} \right| \le
     401 \cdot \varepsilon
\end{align}

\newpage
\subsection*{Aufgabe 2}
\begin{comment}
%\begin{flushright}
%\small{Namen: Farina Ohm, Janina Geiser, Michael Hufschmidt, Annika Seidel}\normalsize
%\end{flushright}
\renewcommand{\labelenumi}{\alph{enumi})}
\begin{enumerate}
\item zu zeigen: strikt diagonaldominante Matrizen sind invertierbar\\
\end{comment}
\paragraph*{a)} Zu zeigen: Strikt diagonaldominante Matrizen sind invertierbar. Sei
\begin{align*}
A =
\begin{pmatrix}
a_{11}&\dots&a_{1n}\\
\vdots&\ddots&\vdots\\
a_{n1}&\dots&a_{nn}
\end{pmatrix}
\in \KK^{nxn}
\text{ und } x_1,...,x_2 \in \RR
\end{align*}
Dann gilt:
\begin{align}\label{eq8}
0 = Ax =
\begin{pmatrix}
a_{11}*x_1&\dots&a_{1n}*x_n\\
\vdots&\ddots&\vdots\\
a_{n1}*x_1&\dots&a_{nn}*x_n
\end{pmatrix}
\Rightarrow
\begin{pmatrix}
  a_{11}*x_1&\dots&a_{1n}*x_n&=0\\
  \vdots&\ddots&\vdots&\vdots\\
  a_{n1}*x_1&\dots&a_{nn}*x_n&=0
\end{pmatrix}
\end{align}

Sei weiterhin i $\in \{1,...,n\}$, s.d. $|x_i| \ge |x_j|\forall k=\{1,...,n\}$.
Aus (\ref{eq8}) folgt dann f�r die i-te Zeile:
\begin{align*}
\sum_{j=1}^n x_j a_{ij} = 0 \Leftrightarrow x_i a_{ii}+\sum_{\underset{j\neq i}{j=1}}^n x_j a_{ij} = 0
\end{align*}
Wir nehmen nun an das $x_i \neq 0$ ist und f�hren dies zum Widerspruch:
\begin{align*}
|x_i||a_{ii}| &= |\sum_{j=1, j\neq i}^n x_j a_{ij}| \le \sum_{j=1, j\neq i}^n |x_j| |a_{ij}| \\
&= |x_j| \sum_{j=1, j\neq i}^n |a_{ij}| \le |x_i| \sum_{j=1, j\neq i}^n |a_{ij}| < |x_i| |a_{ii}|
\end{align*}
Der letzte Schritt folgt dabei direkt aus der strikten Diagonaldominanz von A.
\newline Wir erhalten also: $|x_i| |a_{ii}|< |x_i| |a_{ii}|$ Widerspruch!
\newline $\Rightarrow x_i = 0$. Da aber $|x_i| \ge |x_j|\;\forall j=\{1,...,n\}$ gilt,
folgt auch $x_j=0\;\forall j=\{1,...,n\}$
Der Kern von A ist trivial $\Leftrightarrow$ A ist invertierbar.

\paragraph*{b)}
Zu zeigen: A hermitisch, diagonaldominant und $a_{11},...,a_{nn}$  nicht negativ $\Rightarrow A \ge 0$

Sei $\lambda \in \RR$ ein Eigenwert und $ x \in \RR^n$ Eigenvektor von A, dann gilt $Ax = \lambda x$.
Nomieren wir weiterhin die Einheitsvektoren mit der Maxinumsnorm, sodass $\|x\|_\infty = 1$
und w�hlen $ i \in \{1,...,n\}$ s.d. $|x_i|=1$, dann gilt:
\begin{align*}
\sum_{j=1}^n a_{ij}x_j = \lambda x_i \Leftrightarrow a_{ii}x_i + \sum_{j=1, j\neq i}^n a_{ij}x_j = \lambda x_i \end{align*}

Wir substrahieren auf beiden Seiten der Gleichung $a_{ii} x_i$:
\begin{align*}
\lambda x_i - a_{ii} x_i &= (\lambda - a_{ii})* x_i = \sum_{j=1, j\neq i}^n a_{ij}x_j \\
|\lambda - a_{ii}|* \underset{=1}{|x_i|} &= \left|\sum_{j=1, j\neq i}^n a_{ij}x_j \right|
  \le \sum_{j=1, j\neq i}^n |a_{ij}|\underset{\le1}{|x_j|} \le \sum_{j=1, j\neq i}^n |a_{ij}|
\end{align*}
Also folgt daraus: $|\lambda - a_{ii}| \le \sum_{j=1, j\neq i}^n |a_{ij}|$.
Mit der Formel f�r die Diagonaldominanz
\begin{align*}
|a_{ii}| \ge \sum_{j=1, j\neq i}^n |a_{ij}| \Leftrightarrow |a_{ii}| - \sum_{j=1, j\neq i}^n |a_{ij}| \ge 0
\end{align*}
folgt dann:
\begin{align*}
0 \le a_{ii}-\sum_{j=1, j\neq i}^n |a_{ij}| \le \lambda \le a_{ii} + \sum_{j=1, j\neq i}^n |a_{ij}|
\end{align*}
Es folgt also $\lambda \ge 0 \Rightarrow$  alle Eigenwerte sind nicht negativ $\Rightarrow A $ ist positiv semidefinit.

\paragraph*{c)}
Zu zeigen: A hermitisch, strikt diagonaldominant und $a_{11},...,a_{nn} \text{ positiv}\ \Rightarrow A > 0$

Der Beweis ist analog zu b). Nur folgt aus der strikten Diagonaldominanz
\begin{align*}
|a_{ii}| &< \sum_{j=1, j\neq i}^n |a_{ij}| \Leftrightarrow |a_{ii}| - \sum_{j=1, j\neq i}^n |a_{ij}| < 0\\
\intertext{folgendes:}
0 &< a_{ii}-\sum_{j=1, j\neq i}^n |a_{ij}| \le \lambda \le a_{ii} + \sum_{j=1, j\neq i}^n |a_{ij}|
\end{align*}
Es folgt also $\lambda > 0 \Rightarrow $ alle Eigenwerte sind positiv $\Rightarrow A $ ist positiv definit.

\end{document}
