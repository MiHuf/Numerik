\subsection*{Aufgabe 42}

Gegeben:
\begin{align*}
  A = \begin{pmatrix}20&10&-29\\ -9&1&0\\ 10&10&-19\end{pmatrix}\; ;  \qquad
  B = \begin{pmatrix}12&2&-13\\ -9&1&0\\ 2&2&-3\end{pmatrix}
\end{align*}
Gesucht: Eine Abschätzung für die Eigenwerte mit Hilfe des Satzes von Gerschgorin.
Wir definieren für $M \in \CC$ und $r \in \RR$:
\begin{align*}
  K(M,r) = \{z\in \CC \mid | z - M | \le r\} \quad \text{offener Kreis mit Radius $r$ um $M$}
\end{align*}
Nach dem Satz von Gerschgorin sind alle Eigenwerte $\lambda_i$ einer Matrix $A = (a_{ij})$
in Kreisen
\begin{align*}
  K(a_{ii}, r_i) \quad \text{mit} \quad r_i = \sum_{\substack{i = 1\\j \ne i}}^n |a_{ij} |
\end{align*}
enthalten. Da $A$ und $A^T$ die gleicehn Eigenwerte haben, sind sie aich in den Kreisen
\begin{align*}
  K(a_{ii}, r'_i) \quad \text{mit} \quad r'_i = \sum_{\substack{i = 1\\j \ne i}}^n |a_{ji} |
\end{align*}
entalten. Somit ergibt sich für $A$:
\begin{align*}
  r_1 &= |10| + |-29| = 39\;; \quad r'_1 = |-9| + |10| = 19\;; \quad\min(r_1, r'_1) = 19\\
  r_2 &= |-9| + |0| = 9\;; \quad r'_2 = |10| + |10| = 20\;; \quad \min(r_2, r'_2) = 9\\
  r_3 &= |10| + |10| = 20\;; \quad r'_3 = |-29| + |0| = 29\;; \quad \min(r_3, r'_3) = 20\\
\end{align*}
Eigenwerte liegen also in den Intervallen [20-19, 20+19], [1-9, 1+9] und [-19-20,-19+20].
Eine Rechnung mit MATLAB liefert $\sigma = \{10, -9, 1\}$.

Für $B$ ergibt sich :
\begin{align*}
  r_1 &= |2| + |-13| = 15\;; \quad r'_1 = |-9| + |2| = 11\;; \quad\min(r_1, r'_1) = 11\\
  r_2 &= |-9| + |0| = 9\;; \quad r'_2 = |12| + |2| = 14\;; \quad \min(r_2, r'_2) = 9\\
  r_3 &= |2| + |2| = 4\;; \quad r'_3 = |0| + |-13| = 13\;; \quad \min(r_3, r'_3) = 4\\
\end{align*}
Eigenwerte liegen also in den Intervallen [12-11, 12+11], [1-9, 1+9] und [-3-4,-3+4].
Eine Rechnung mit MATLAB liefert $\sigma = \{10, 1, 1\}$.
