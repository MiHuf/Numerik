%  DOCUMENT CLASS
\documentclass[11pt]{article}

%PACKAGES
\usepackage[utf8]{inputenc}
\usepackage[ngerman]{babel}
\usepackage[reqno,fleqn]{amsmath}
\setlength\mathindent{10mm}
\usepackage{amssymb}
\usepackage{amsthm}
\usepackage{color}
\usepackage{delarray}
% \usepackage{fancyhdr}
\usepackage{units}
\usepackage{times, eurosym}
\usepackage{verbatim} %Für Verwendung von multiline Comments mittels \begin{comment}...\end{comment}
\usepackage{wasysym} % Für Smileys


% FORMATIERUNG
\usepackage[paper=a4paper,left=25mm,right=25mm,top=25mm,bottom=25mm]{geometry}
\usepackage{array}
\usepackage{fancybox} %zum Einrahmen von Formeln
\setlength{\parindent}{0cm}
\setlength{\parskip}{1mm plus1mm minus1mm}


% PAGESTYLE

%MATH SHORTCUTS
\newcommand{\NN}{\mathbb N}
\newcommand{\ZZ}{\mathbb Z}
\newcommand{\QQ}{\mathbb Q}
\newcommand{\RR}{\mathbb R}
\newcommand{\CC}{\mathbb C}
\newcommand{\KK}{\mathbb K}
\newcommand{\U}{\mathbb O}
\newcommand{\eqx}{\overset{!}{=}}
\newcommand{\Det}{\mathrm{Det}}
\newcommand{\Gl}{\mathrm{Gl}}
\newcommand{\diag}{\mathrm{diag}}
\newcommand{\sign}{\mathrm{sign}}
\newcommand{\rang}{\mathrm{rang}}
\newcommand{\cond}{\mathrm{cond}_{\| \cdot \|}}
\newcommand{\conda}{\mathrm{cond}_{\| \cdot \|_1}}
\newcommand{\condb}{\mathrm{cond}_{\| \cdot \|_2}}
\newcommand{\condi}{\mathrm{cond}_{\| \cdot \|_\infty}}
\newcommand{\eps}{\epsilon}

\setlength{\extrarowheight}{1ex}

\begin{document}

\begin{center}
\textbf{
Übungen zur Vorlesung Numerische Mathematik, WS 2014/15\\
Zusatz-Übungsblatt zum 14.01.2015\\
}

\begin{tabular}{lll}
& \\
von & Janina Geiser & Mat Nr. 6420269\\
& Michael Hufschmidt & Mat.Nr. 6436122\\
& Farina Ohm & Mat Nr. 6314051\\
& Annika Seidel & Mat Nr. 6420536\\
\\
\hline
\end{tabular}
\end{center}

\subsection*{Aufgabe Z1}
Partielle Laplace-DGLen $-\Delta u(x,y) = f(x,y)$ mit Dirichlet-Randbedingungen
$u(x,y) = 0$ lassen sich, wie in Aufgabe 20a gezeigt, auf dem Intervall $(0,1)$
mit einem diskretisierten Laplace-Operator lösen: In einem Gitter mit
$n \times n$ Punkten im Abstand $h = \frac{1}{n+1}$, also $x_i=i\cdot h$ und
$y_j = j \cdot h$ für $i, j = 1, \cdots, n$ für das Intervall
$x, y \in [h, \frac{n}{n+1}]$
ist das die blocktridiagonale $n^2 \times n^2$ Matrix
\begin{align}
\label{eq-A}
A_n&=\frac{1}{h^2} \begin{pmatrix}
D & -I_n & 0 & \cdots & 0 & 0\\
-I_n & D & -I_n & 0 & \cdots & 0\\
0 & -I_n& \ddots & \ddots & \ddots &\vdots\\
\vdots & 0 & \ddots& \ddots & -I_n & 0\\
0 & \vdots & \ddots & -I_n & D & -I_n\\
0 & 0 & \cdots & 0 & -I_n & D
\end{pmatrix} \\
\intertext{ mit der $n^2 \times n^2$ Matrix }
\label{eq-D}
D &=\begin{pmatrix}
4 & -1 & 0 & \cdots & 0 & -1\\
-1 & 4 & -1 & 0 &\cdots & 0\\
0 & -1 & \ddots & \ddots & \ddots & \vdots\\
\vdots & 0 & \ddots & \ddots &-1 & 0\\
0 & \vdots & \ddots & -1 & 4 & -1\\
-1 & 0 & \cdots & 0 & -1 & 4
\end{pmatrix}
\end{align}
Die $u_{i,j}$ ergeben sich nach Lösung des LGS $A \cdot \vec u = \vec f$
mit den $n^2$-dimensionalen Vektoren $\vec u $ und $\vec f$, welche zeilenweise
alle Matrix-Elemente $u_{i,j}$ und $f_{i, j} = f_{x_i, y_j} \; i, j = 1, \cdots, n$
enthalten.
\paragraph{}
Im vorliegenden Fall liegen periodische Randbedingungen für $x = 0, 1$ und  $y = 0, 1$ vor.
Wenn wir wieder $n \times n$ Punkte  $i, j = 1, \cdots, n$ berücksichtigen,
sind die diskretisierten Koordinaten $x_i=(i-1)\cdot h$ und $y_j = (j-1)\cdot h$
für das Intervall $x, y \in [0, \frac{n-1}{n}]$ mit einer Gitterweite $h = \frac{1}{n}$.

Mit periodischen Randbedingungen lautet die Matrix$A'_n$ aber nun:
\begin{align}
\label{eq-A2}
A'_n&=\frac{1}{h^2} \begin{pmatrix}
D & -I_n & 0 & \cdots & 0 & -I_n\\
-I_n & D & -I_n & 0 & \cdots & 0\\
0 & -I_n& \ddots & \ddots & \ddots &\vdots\\
\vdots & 0 & \ddots& \ddots & -I_n & 0\\
0 & \vdots & \ddots & -I_n & D & -I_n\\
-I_n & 0 & \cdots & 0 & -I_n & D
\end{pmatrix}
\end{align}
mit $D$ wie in \eqref{eq-D}.
Wiederum enthalten die $n^2$-dimensionalen Vektoren $\vec u $ und $\vec f$
zeilenweise die Elemente von $u_{i,j} = u(x_i, y_j)$ und $f_{i, j} = f(x_i, y_j)$;
die DGL $- \Delta u(x,y) + u(x,y) = f(x, y)$ lautet dann diskretisiert:
\begin{align}
\label{eq-B}
& \left(A'_n + I_{n^2} \right) \vec u = \vec f \quad \text{mit} \\
\nonumber
& \vec u = \left(u_{1,1}, \cdots , u_{1,n}, u_{2,1}, \cdots , u_{2,n}, \cdots , \cdots u_{n,1}, \cdots , u_{n,n} \right)^T \\
\nonumber
& \vec f = \left(f_{1,1}, \cdots , f_{1,n}, f_{2,1}, \cdots , f_{2,n}, \cdots , \cdots f_{n,1}, \cdots , f_{n,n} \right)^T
\end{align}
mit der Matrix $A'_n$ wie \eqref{eq-A2}. Nach Lösen von \eqref{eq-B} mit einem
geeigneten Verfahren erhält man dann die gesuchten  $u_{i,j}$ ähnlich wie
in Aufgabe 20a.


\end{document}
