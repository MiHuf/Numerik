%  DOCUMENT CLASS
\documentclass[11pt]{article}

%PACKAGES
\usepackage[latin1]{inputenc}
\usepackage[ngerman]{babel}
\usepackage[reqno,fleqn]{amsmath}
\setlength\mathindent{10mm}
\usepackage{amssymb}
\usepackage{amsthm}
\usepackage{delarray}
% \usepackage{fancyhdr}
\usepackage{units}
\usepackage{times, eurosym}
\usepackage{verbatim}%F�r Verwendung von multiline Comments mittels \begin{comment}...\end{comment}


% FORMATIERUNG
\usepackage[paper=a4paper,left=25mm,right=25mm,top=25mm,bottom=25mm]{geometry}
\usepackage{array}
\usepackage{fancybox} %zum Einrahmen von Formeln
\setlength{\parindent}{0cm}
\setlength{\parskip}{1mm plus1mm minus1mm}


% PAGESTYLE


%MATH SHORTCUTS
\newcommand{\NN}{\mathbb N}
\newcommand{\RR}{\mathbb R}
\newcommand{\CC}{\mathbb C}
\newcommand{\KK}{\mathbb K}
\newcommand{\eqx}{\overset{!}{=}}
\newcommand{\Det}{\mathrm{Det}}
\newcommand{\cond}{\mathrm{cond}_{\| \cdot \|}}
\newcommand{\conda}{\mathrm{cond}_{\| \cdot \|_1}}
\newcommand{\condb}{\mathrm{cond}_{\| \cdot \|_2}}
\newcommand{\condi}{\mathrm{cond}_{\| \cdot \|_\infty}}

\setlength{\extrarowheight}{1ex}
\begin{document}

\begin{center}
\textbf{
�bungen zur Vorlesung Numerische Mathematik, WS 2014/15\\
Blatt 02 zum 27.10.2014\\
}

\begin{tabular}{lll}
& \\
von & Janina Geiser & Mat Nr. 6420269\\
& Michael Hufschmidt & Mat.Nr. 6436122\\
& Farina Ohm & Mat Nr. 6314051\\
& Annika Seidel & Mat Nr. 6420536\\
\\
\hline
\end{tabular}
\end{center}


	Inhalt...

\subsection*{Aufgabe 3}
\paragraph*{a)}


\paragraph*{b)}


\paragraph*{c)}


\newpage
\subsection*{Aufgabe 4}
Gegeben:
\begin{align*} \label{eq8}
A=
\begin{split}
\begin{pmatrix}
\frac{1}{\sqrt2} & \frac{1}{\sqrt2} \\
-\frac{1}{\sqrt2} &\frac{1}{\sqrt2}
\end{pmatrix}
\end{split}
\end{align*}
Nach Vorlesung gilt: \fbox{$\mathrm{\cond}(A)=\|A\|*\|A^{-1}\|$}\\

Wir berechnen also $A^{-1}$ mithilfe des Gauss-Jordan-Verfahren und erhalten:
%hier weis ich nicht, wie ich den senkrechten Strich in der Matrix zwischen A und Einheitsmatrix hinbekomme
\begin{align*}
(A|I)=
\begin{split}
\begin{array}({cc|cc})
\frac{1}{\sqrt2} & \frac{1}{\sqrt2} & 1 & 0 \\
-\frac{1}{\sqrt2} &\frac{1}{\sqrt2} & 0 & 1
\end{array}
\Rightarrow(I|A^{-1})=
\begin{array}({cc|cc})
 1 & 0 & \frac{\sqrt2}{2} & -\frac{\sqrt2}{2} \\
 0 & 1 & \frac{\sqrt2}{2} & \frac{\sqrt2}{2}
\end{array}
\end{split}
\end{align*}
Daraus folgt:\\
Spaltensummennorm:
\begin{align*}
\begin{split}
&\|A\|_1=\mathrm{max}\left\lbrace \left|\frac{1}{\sqrt2}\right|+\left|-\frac{1}{\sqrt2}\right|, \left|\frac{1}{\sqrt2}\right|+\left|\frac{1}{\sqrt2}\right|\right\rbrace=\frac{2}{\sqrt2}\\
&\|A^{-1}\|_1=\mathrm{max}\left\lbrace\left|\frac{\sqrt2}{2}\right|+\left|\frac{\sqrt2}{2}\right|, \left|-\frac{\sqrt2}{2}\right|+\left|\frac{\sqrt2}{2}\right|\right\rbrace={\sqrt2}\\
& \conda(A)=\|A|_1*\|A^{-1}\|_1=\frac{2}{\sqrt2} * {\sqrt2} = 2\\
\end{split}
\end{align*}
euklidische Norm:
\begin{align*}
\begin{split}
&\|A\|_2=\sqrt{ \left(\frac{1}{\sqrt2}\right)^2+\left(-\frac{1}{\sqrt2}\right)^2+\left(\frac{1}{\sqrt2}\right)^2+\left(\frac{1}{\sqrt2}\right)^2}=\sqrt{2}\\
&\|A^{-1}\|_2=\sqrt{\left(\frac{\sqrt2}{2}\right)^2+\left(\frac{\sqrt2}{2}\right)^2 + \left(-\frac{\sqrt2}{2}\right)^2 + \left(\frac{\sqrt2}{2}\right)^2}=\sqrt{2}\\
& \condb(A)=\|A|_2*\|A^{-1}\|_2=\sqrt{2}*\sqrt{2}=2\\
\end{split}
\end{align*}
Zeilensummennorm:
\begin{align*}
\begin{split}
&\|A\|_\infty =\mathrm{max}\left\lbrace\left|\frac{1}{\sqrt2}\right|+\left|\frac{1}{\sqrt2}\right|, \left|-\frac{1}{\sqrt2}\right|+\left|\frac{1}{\sqrt2}\right|\right\rbrace=\frac{2}{\sqrt2}\\
&\|A^{-1}\|_\infty=\mathrm{max}\left\lbrace\left|\frac{\sqrt2}{2}\right|+\left|-\frac{\sqrt2}{2}\right|, \left|\frac{\sqrt2}{2}\right|+\left|\frac{\sqrt2}{2}\right|\right\rbrace={\sqrt2}\\
& \condi(A)=\|A|_\infty*\|A^{-1}\|_\infty=\frac{2}{\sqrt2} * {\sqrt2} = 2\\
\end{split}
\end{align*}
\newline


Gegeben:\\
\begin{align*} 
B=
\begin{split}
\begin{pmatrix}
1 & 2 & 0 \\
0 & 1 & 0 \\
0 & 1 & 1
\end{pmatrix}
\end{split}
\end{align*}
Analog zu oben berechnen wir $B^{-1}$:\\
\begin{align*}
(B|I)=
\begin{split}
\begin{array}({ccc|ccc})
1 & 2 & 0 & 1 & 0 & 0\\
0 & 1 & 0 & 0 & 1 & 0\\
0 & 1 & 1 & 0 & 0 & 1
\end{array}
\Rightarrow(I|B^{-1})=
\begin{array}({ccc|ccc})
1 & 0 & 0 &1 & -2 & 0 \\
0 & 1 & 0 &0 & 1 & 0 \\
0 & 0 & 1 &0 & -1 & 1  
\end{array}
\end{split}
\end{align*}
Daraus folgt:\\
Spaltensummennorm:
\begin{align*}
\begin{split}
&\|B\|_1=\mathrm{max} \left\lbrace|1|+|0|+|0|, |2|+|1|+|1|, |0|+|0|+|1|\right\rbrace = 4\\
&\|B^{-1}\|_1=\mathrm{max} \left\lbrace|1|+|0|+|0|, |-2|+|1|+|-1|, |0|+|0|+|1|\right\rbrace = 4\\
&\conda(B)=\|B|_1*\|B^{-1}\|_1= 4 * 4 = 16\\
\end{split}
\end{align*}
euklidische Norm:
\begin{align*}
\begin{split}
&\|B\|_2=\sqrt{1^2+0^2+0^2+2^2+1^2+1^2+0^2+0^2+1^2}=\sqrt{8}\\
&\|B^{-1}\|_2=\sqrt{1^2+0^2+0^2+(-2)^2+1^2+(-1)^2+0^2+0^2+1^2}=\sqrt{8}\\
&\condb(B)=\|B|_2*\|B^{-1}\|_2=\sqrt{8}*\sqrt{8}=8\\
\end{split}
\end{align*}
Zeilensummennorm:
\begin{align*}
\begin{split}
&\|B\|_\infty = \mathrm{max} \left\lbrace|1|+|2|+|0|, |0|+|1|+|0|,|0|+|1|+|1|\right\rbrace = 3\\
&\|B^{-1}\|_\infty = \mathrm{max} \left\lbrace|1|+|-2|+|0|, |0|+|1|+|0|,|0|+|-1|+|1|\right\rbrace = 3\\
&\condi(B)=\|B|_\infty*\|B^{-1}\|_\infty = 3 * 3 = 9\\
\end{split}
\end{align*}

\newpage
\subsection*{Aufgabe 5}
\paragraph*{a)}
zu zeigen: die Zeilensummennorm wird von der Maximumnorm induziert\\
\newline
$\mathrm{Zeilensummennorm: }\|A\|_\infty=\underset{1 \le i \le m}{\mathrm{max}} \sum_{j=1}^{n}|a_{ij}|$\\
\newline
$\mathrm{Maximumnorm: }\|x\|_\infty=\underset{i}{\mathrm{max}} |x_{i}|$\\
\newline
Die der Maximumnorm $\|x\|_\infty$ zugeordnete Matrixnorm $\|A\|_\infty$ ist gegeben durch:
\begin{align*}
&\|A\|_\infty := \underset{x\neq 0}{\mathrm{max}} \frac{\|Ax\|_\infty}{\|x\|} = 
\underset{\|x\|=1}{\mathrm{max}} \|Ax\|_\infty =
\underset{\|x\|=1}{\mathrm{max}}\left\lbrace \underset{i}{\mathrm{max}}\left|\sum_{j=1}^{n}a_{ij}x_j\right|\right\rbrace\\&=
\underset{i}{\mathrm{max}}\left\lbrace\underset{\|x\|=1}{\mathrm{max}}\left|\sum_{j=1}^{n} a_{ij}x_j\right|\right\rbrace
\overset{x_j=sign(a_{ij})}{=}\underset{i}{\mathrm{max}}\sum_{j=1}^{n}|a_{ij}|=: \mathrm{Zeilensummennorm} \|A\|_\infty
\end{align*}
\begin{flushright}Q.e.d.\end{flushright}

\paragraph*{b)}
zu zeigen: die Spaltensummennorm wird von der Betragssummennorm induziert\\
\newline
$\mathrm{Spaltensummennorm: }\|A\|_1=\underset{1 \le j \le n}{\mathrm{max}} \sum_{i=1}^{m}|a_{ij}|$\\
\newline
$\mathrm{Betragssummenmnorm: }\|x\|_1=\sum_{i=1}^{m}|x_i|$\\
\newline
Die der Betragssummennorm $\|x\|_1$ zugeordnete Matrixnorm $\|A\|_1$ ist gegeben durch:
\begin{align*}
\|A\|_1 := \underset{x\neq 0}{\mathrm{max}} \frac{\|Ax\|_1}{\|x\|} = \underset{\|x\|=1}{\mathrm{max}} \|Ax\|_1 = \underset{\|x\|=1}{\mathrm{max}}\left\lbrace\sum_{i=1}^{m}\left|\sum_{j=1}^{n}a_{ij}x_j\right|\right\rbrace = ...
\end{align*}
%hier fehlt noch was
TODO!!!

\subsection*{Aufgabe 6}
Zu zeigen: F�r $ A \in \Gl_n (\KK) $ gilt
\begin{align*}
\condb(A)=\sqrt{\frac{\lambda_{\max} (A^* \cdot A)}{\lambda_{\min} (A^* \cdot A)}}
\end{align*}
wobei $\lambda_{\max}(A^* \cdot A)$ und $\lambda_{\min}(A^* \cdot A)$ jeweils der
gr��te und kleinste Eigenwert von $A^* \cdot A$ ist.

Die Matrix $A^* \cdot A$ ist hermitesch, denn $(A^* \cdot A)^* = A^* \cdot A$.
Sie hat somit $n$ reelle, nicht-negative Eigenwerte. Da $ A \in \Gl_n (\KK) $,
kann es auch keinen Eigenwert $0$ geben, die Eigenwerte von  $A^* \cdot A$ sind
also positive reelle Zahlen $\lambda_i > 0$. Diese ordne ich nun an, so dass
$\lambda_1 \ge \lambda_2 \ge \cdots \lambda_n > 0$ gilt. Damit ist
$\lambda_{\max} (A^* \cdot A) = \lambda_1$ und $\lambda_{\min} (A^* \cdot A) = \lambda_n$

Da  $A^* \cdot A$ hermitesch ist, ist $A^* \cdot A$ auch diagonalisierbar, es
gibt also eine unit�re Matrix $U$, so dass\\
$A^* \cdot A = U^* \cdot \diag(\lambda_1 \cdots \lambda_n) \cdot U $ gilt.

Seien $A, B, C$ beliebige invertierbare Matrizen, dann gilt:
$(A \cdot B \cdot C)^{-1} = C^{-1} \cdot B^{-1} \cdot A^{-1}$, denn\\
$(A \cdot B \cdot C)^{-1} \cdot (A \cdot B \cdot C) =
  (C^{-1} \cdot B^{-1} \cdot A^{-1}) \cdot (A \cdot B \cdot C) = I $.
Damit berechnen wir nun
$(A^* \cdot A)^{-1} = (U^* \cdot \diag(\lambda_1 \cdots \lambda_n) \cdot U)^{-1}
= U^{-1} \cdot \diag(\lambda_1 \cdots \lambda_n)^{-1}  \cdot (U^*)^{-1}$, bzw.
wegen der Unitari�t $U = (U^*)^{-1} \; \text{:}
\quad (A^* \cdot A)^{-1} = U^* \cdot \diag(\lambda_1 \cdots \lambda_n)^{-1}  \cdot U$

\begin{align*}
  \diag(\lambda_1 \cdots \lambda_n) =
  \begin{pmatrix}
    \lambda_1 & &\\
    & \ddots & \\
    & & \lambda_n
  \end{pmatrix} \quad \text{und} \quad
  \diag(\lambda_1 \cdots \lambda_n)^{-1} =
  \begin{pmatrix}
    1 / \lambda_1 & &\\
    & \ddots & \\
    & & 1/ \lambda_n
  \end{pmatrix}
\end{align*}
Denn man sieht leicht, dass
\begin{align*}
  \begin{pmatrix}
    \lambda_1 & &\\
    & \ddots & \\
    & & \lambda_n
  \end{pmatrix} \cdot
  \begin{pmatrix}
    1 / \lambda_1 & &\\
    & \ddots & \\
    & & 1 / \lambda_n
  \end{pmatrix}
  = I
\end{align*}

Die Diagonalmatrizen enthalten die Eigenwerte. Der gr��te Eigenwert von
$(A^* \cdot A)^{-1}$ ist also \\
$\lambda_{\max} (A^* \cdot A) ^{-1} = 1 / \lambda_n$, damit wird
\begin{align*}
\condb(A) & = \|A^{-1}\|_2  \cdot \| A \|_2 =
\sqrt{\lambda_{\max} (A^* \cdot A)^{-1}} \cdot \sqrt{\lambda_{\max} (A^* \cdot A)}
= \sqrt{\frac{1}{\lambda_n}} \cdot \sqrt{\lambda_1} \\
&= \sqrt{\frac{\lambda_1}{\lambda_n}} = \sqrt{\frac{\lambda_{_max}}{\lambda_{\min}}}
\end{align*}
\begin{flushright}Q.e.d.\end{flushright}

Noch zu zeigen Sei $U$ unit�re Matrix. $A = U^n \; \Leftrightarrow \;  \condb(A) = 1$

"`$\Rightarrow$"': Wenn $U$ unit�re Matrix ist, gilt $\| U^* U \|_2 = $
TODO ...

"`$\Leftarrow$"': Sei $ \condb(A) = 1$, dann gilt nach dem Obigen
$\lambda_{\max} (A^* \cdot A) = \lambda_{\min} (A^* \cdot A) = 1$
TODO ...


TODO: hier fehlt noch etwas



\end{document}
