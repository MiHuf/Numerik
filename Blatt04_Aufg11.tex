\subsection*{Aufgabe 11}
Sei $A \in \KK^{n \times n}$ hermitesch.\\
\newline
zu zeigen: Singulärwerte von $A$ bestehen aus der Menge der Beträge der Eigenwerte von $A$.\\
\newline
Beweis:\\
Für die Matrix $A$ mit Eigenvektor $x \neq 0$ und Eigenwert $\lambda \in \KK$ gilt folgende Eigenwertgleichung:
\begin{align}
\label{EW}
Ax=\lambda x
\end{align}
Für eine hermitesche Matrix A gilt weiterhin:\begin{align}\label{Hermitesch}A=A^*\end{align}
Wir betrachten die Eigenwertgleichung der Matrix $A^*A$:
\begin{align*}
A^*Ax\overset{\eqref{Hermitesch}}{=}A\cdot \underbrace{Ax}_{\substack{\overset{\eqref{EW}}{=}\lambda x}}=A\lambda x=\lambda \underbrace{Ax}_{\substack{\overset{\eqref{EW}}{=}\lambda x}}=\lambda \lambda x =\lambda^2x
\end{align*}\\
Daraus folgt, dass $\lambda^2$ ein Eigenwert von $(A^*A)$ ist $(x \neq 0)$.\\
Aus der Vorlesung folgt, dass die Singulärwerte von $A$ gleich den Quadratwurzeln der Eigenwerten von $(A^*A)$ sind (\textit{Kapitel 2.8.3 Singulärwertzerlegung}).\\\\
Also gilt nun für die Singulärwerte $\sigma_i$ von $A$:\\
\begin{align}\label{betrag}\sigma_i= \sqrt{\lambda_i^{2}}=|\lambda_i|\quad \forall i \in\{1,\cdots,r\} \text{ mit } r=\mathrm{rang}(A)\end{align}
\newline
Da $\lambda$ genau so gewählt war, dass es die Eigenwertgleichung \eqref{EW} von $A$ erfüllt, also ein Eigenwert von $A$ ist, ist damit bewiesen, dass die Singulärwerte von $A$ aus der Menge der Beträge der Eigenwerte von $A$ bestehen \eqref{betrag}.
\begin{flushright}Q.e.d.\end{flushright}
Beziehung zwischen der Eigenwert- und Singulärwertzerlegung:\\\\
Es sei $A=U\Sigma V^*$ eine Singulärwertzerlegung mit $U$ unitäre $n\times n$ Matrix, $V^*$ die Adjungierte einer unitären $n\times n$-Matrix V und $\Sigma$ eine reelle $n\times n$-Diagonalmatrix.\\\\
Um die Beziehung darzustellen, nutzen wir folgende Eigenschaften der unitären Matrizen U und V:
\begin{align}
U\cdot U^* =I \text{ bzw. } U^* \cdot U=I\\
U^*=U^{-1}
\end{align}
Für die hermitesche Matrix A gilt dann:
\begin{align}
A^*A\overset{\text{Singulärwertzerlegung}}{=} V\Sigma^T\underbrace{U^*U}_{\overset{(4)}{=}I}\Sigma
V^* = V\underbrace{\Sigma^T\Sigma}_{=\Delta}V^*=V\Delta V^*\overset{(5)}{=}V\Delta V^{-1}
\end{align}
mit $\Delta=\Sigma^T \Sigma=diag(\lambda_1,\ldots,\lambda_n)$\\
\newline
Da die Singulärwerte der Matrix A gleich den Quadratwurzeln aus den positiven Eigenwerten von $A^*A$ sind, ergibt sich für eine hermitesche Matrix:
\begin{align*}
	\sigma_i(A)=\sqrt{\lambda(A^*A)} \overset{A^*=A}{=} \sqrt{\lambda_i(AA)}=\sqrt{\lambda_i(A)\lambda_i(A)}=\sqrt{\lambda_i(A)^2}=\lambda_i(A) \quad \forall i\in\{1,\ldots,n\}
\end{align*}
Für hermitesche Matrizen ergibt sich also, die Besonderheit, dass die Eigenwerte mit den Singulärwerten übereinstimmen.\\\\
\newline
Beispiel einer nicht-Hermiteschen Matrix um zu zeigen, dass dann obige Aussagen nicht mehr gelten:
\begin{align*}
&B=
\begin{pmatrix}
	1 & 1000\\
	0 & 1
\end{pmatrix}, 
B^*=
\begin{pmatrix}
1 & 0\\
1000 & 1
\end{pmatrix}
\Rightarrow B\neq B^*\end{align*}
Eigenwerte von B berechnen:
\begin{align*}
&(B-\lambda I)=(1-\lambda)^2\\
&(1-\lambda)^2=0 \Rightarrow \lambda_{1,2}(B)=1
\end{align*}
Eigenwerte von ($B^*B$) berechnen:
\begin{align*}
&(B^*B)=
\begin{pmatrix}
1 & 1000\\
1000 & 10^6+1
\end{pmatrix}\\
&((B^*B)-\lambda I)=(1-\lambda)(1000001-\lambda)-1000^2=1000001-\lambda-1000001\lambda+\lambda^2-1000^2\\ &=\lambda^2-1000002\lambda+1\\
&\lambda^2-1000002\lambda+1=0 \Rightarrow \lambda_{1}(B^*B)\approx 10^6, \lambda_{2}(B^*B) \approx 1,00001\cdot 10^{-6}\\
&\sigma_1(B)=\sqrt{\lambda_1(B^*B)}=1000 \neq |1|\\
&\sigma_2(B)=\sqrt{\lambda_2(B^*B)}=0,001000005 \neq|1|
\end{align*}
Die Singulärwerte von B (einer nicht-hermiteschen Matrix) sind also nicht aus der Menge der Beträge der Eigenwerte von B.