\subsection*{Aufgabe 11}
Sei $A \in \KK^{n \times n}$ hermitesch.\\
\newline
zu zeigen: Singulärwerte von $A$ bestehen aus der Menge der Beträge der Eigenwerte von $A$.\\
\newline
Beweis:\\
Für die Matrix $A$ mit Eigenvektor $x \neq 0$ und Eigenwert $\lambda \in \KK$ gilt folgende Eigenwertgleichung:
\begin{align}
\label{EW}
Ax=\lambda x
\end{align}
Für eine hermitesche Matrix A gilt weiterhin:\begin{align}\label{Hermitesch}A=A^*\end{align}
Wir betrachten die Eigenwertgleichung der Matrix $A^*A$:
\begin{align*}
A^*Ax\overset{\eqref{Hermitesch}}{=}A\cdot \underbrace{Ax}_{\substack{\overset{\eqref{EW}}{=}\lambda x}}=A\lambda x=\lambda \underbrace{Ax}_{\substack{\overset{\eqref{EW}}{=}\lambda x}}=\lambda \lambda x =\lambda^2x
\end{align*}\\
Daraus folgt, dass $\lambda^2$ ein Eigenwert von $(A^*A)$ ist $(x \neq 0)$.\\
Aus der Vorlesung folgt, dass die Singulärwerte von $A$ gleich den Quadratwurzeln der Eigenwerten von $(A^*A)$ sind (\textit{Kapitel 2.8.3 Singulärwertzerlegung}).\\\\
Also gilt nun für die Singulärwerte $\sigma_i$ von $A$:\\
\begin{align}\label{betrag}\sigma_i= \sqrt{\lambda_i^{2}}=|\lambda_i|\quad \forall i \in\{1,\cdots,r\} \text{ mit } r=\mathrm{rang}(A)\end{align}
\newline
Da $\lambda$ genau so gewählt war, dass es die Eigenwertgleichung \eqref{EW} von $A$ erfüllt, also ein Eigenwert von $A$ ist, ist damit bewiesen, dass die Singulärwerte von $A$ aus der Menge der Beträge der Eigenwerte von $A$ bestehen \eqref{betrag}.
\begin{flushright}Q.e.d.\end{flushright}
Beziehung zwischen der Eigenwert- und Singulärwertzerlegung:\\\\
Es sei $A=U\Sigma V^*$ mit $U$ eine unitäre $n\times n$ Matrix, $V$ die Adjungierte einer unitären $n\times n$-Matrix V und $\Sigma$ eine reelle $n\times n$-Diagonalmatrix.\\
Für die hermitesche Matrix A gilt dann:
\begin{align*}
AA^*=U\Sigma \underbrace{V^*V}_{=I}\Sigma^TU^*=U\underbrace{\Sigma\Sigma^T}_{=\Delta}U^*=U\Delta U^* = U\Delta U^{-1}
\end{align*}
Es gilt also $\Delta=\Sigma^T \Sigma=diag(\lambda_1,\ldots,\lambda_n)$\\
\newline
IST DAS HIER ÜBERHAUPT SINNVOLL WAS ICH TUE???\\
\newline
Beispiel einer nicht-Hermiteschen Matrix:\\
TODO\\