%  DOCUMENT CLASS
\documentclass[11pt]{article}

%PACKAGES
\usepackage[utf8]{inputenc}
\usepackage[ngerman]{babel}
\usepackage[reqno,fleqn]{amsmath}
\setlength\mathindent{10mm}
\usepackage{amssymb}
\usepackage{amsthm}
\usepackage{color}
\usepackage{delarray}
% \usepackage{fancyhdr}
\usepackage{units}
\usepackage{times, eurosym}
\usepackage{verbatim} %Für Verwendung von multiline Comments mittels \begin{comment}...\end{comment}
\usepackage{wasysym} % Für Smileys


% FORMATIERUNG
\usepackage[paper=a4paper,left=25mm,right=25mm,top=25mm,bottom=25mm]{geometry}
\usepackage{array}
\usepackage{fancybox} %zum Einrahmen von Formeln
\setlength{\parindent}{0cm}
\setlength{\parskip}{1mm plus1mm minus1mm}


% PAGESTYLE

%MATH SHORTCUTS
\newcommand{\NN}{\mathbb N}
\newcommand{\ZZ}{\mathbb Z}
\newcommand{\QQ}{\mathbb Q}
\newcommand{\RR}{\mathbb R}
\newcommand{\CC}{\mathbb C}
\newcommand{\KK}{\mathbb K}
\newcommand{\U}{\mathbb O}
\newcommand{\eqx}{\overset{!}{=}}
\newcommand{\Det}{\mathrm{Det}}
\newcommand{\Gl}{\mathrm{Gl}}
\newcommand{\diag}{\mathrm{diag}}
\newcommand{\sign}{\mathrm{sign}}
\newcommand{\rang}{\mathrm{rang}}
\newcommand{\cond}{\mathrm{cond}_{\| \cdot \|}}
\newcommand{\conda}{\mathrm{cond}_{\| \cdot \|_1}}
\newcommand{\condb}{\mathrm{cond}_{\| \cdot \|_2}}
\newcommand{\condi}{\mathrm{cond}_{\| \cdot \|_\infty}}
\newcommand{\eps}{\epsilon}

\setlength{\extrarowheight}{1ex}

\begin{document}

\begin{center}
\textbf{
Übungen zur Vorlesung Numerische Mathematik, WS 2014/15\\
Blatt 06 zum 03.12.2014\\
}

\begin{tabular}{lll}
& \\
von & Janina Geiser & Mat Nr. 6420269\\
& Michael Hufschmidt & Mat.Nr. 6436122\\
& Farina Ohm & Mat Nr. 6314051\\
& Annika Seidel & Mat Nr. 6420536\\
\\
\hline
\end{tabular}
\end{center}

\subsection*{Aufgabe 18}
Betrachtet wird das eindimensionale Newtonverfahren für Funktionen, bei denen die Bedingung $f'(x_*) \neq 0$ verletzt ist:\\
Sei also $p \ge 2$ und eine $p+1$-fach stetig differenzierbare Funktion $f:[a,b] \rightarrow \RR$ mit eine $p$-fachen Nullstelle in $x_* \in (a,b)$, d.h.
\begin{align*}
f(x_*) = f'(x_*) = ... = f^{(p-1)}(x_*) = 0, \qquad f^{(p)}(x_*) \neq 0
\end{align*}

\paragraph*{a)}
Zu zeigen: Das Newton Verfahren $x^{(k+1)}=x^{(k)} - \frac{f(x^{(k)})}{f'(x^{(k)})}$ ist linear konvergent (solange man nicht sofort in der Lösung startet).\\

Beweis: Im Folgenden sei $g(x)$ die Iterationsvorschrift $g(x^k):= x^{(k)} - \frac{f(x^{(k)})}{f'(x^{(k)})} = x^{k+1}$. Es gilt unter anderem $f(x_*) = 0$ und $g(x_*) = x_*$. Wir untersuchen
$g'(x_*)$ in Abhängigkeit der Vielfachheit von p:
\begin{align*}
g'(x_*) = \lim\limits_{d \rightarrow 0} \frac{g(x_* +d) - g(x_*)}{d}
\end{align*}
wobei $g(x_*+d) = x_*+d-\frac{f(x_*+d)}{f'(x_*+d)})$.
\begin{align*}
f(x_*+d) = f(x_*) + d \cdot f'(x_*) +\frac{d^2}{2!}\cdot f''(x_*) + ... + \frac{d^p}{p!}\cdot f^{(p)}(x_*) + O(d^{p+1})
\end{align*}
Durch die Annahme das $x_*$ die $p$-fache Nullestelle, also $f(x_*) = f'(x_*) = ... = f^{(p-1)}(x_*) = 0$, folgt:
\begin{align*}
f(x_*+d) = \frac{d^p}{p!}\cdot f^{(p)}(x_*) + O(d^{p+1})
\end{align*}
Analog gilt
\begin{align*}
f'(x_*+d) = f'(x_*) + d \cdot f''(x_*) +\frac{d^2}{2!}\cdot f'''(x_*) + ... + \frac{d^{p-1}}{(p-1)!}\cdot f^{(p)}(x_*) + O(d^{p})
\end{align*}
Woraus durch die gleiche Annahme der Nullstelle $x_*$ folgt:
\begin{align*}
f'(x_*+d) = \frac{d^{p-1}}{(p-1)!}\cdot f^{(p)}(x_*) + O(d^{p})
\end{align*}
Es folgt also nun:
\begin{align*}
g(x_*+d) &= x_*+d-\frac{f(x_*+d)}{f'(x_*+d)}\\
&= x_*+d-\frac{\frac{d^p}{p!}\cdot f^{(p)}(x_*) + O(d^{p+1})}{\frac{d^{p-1}}{(p-1)!}\cdot f^{(p)}(x_*) + O(d^{p})}\\
&= x_*+d-\frac{f^{(p)}(x_*) + \frac{p!}{d^p} O(d^{p+1})}{f^{(p)}(x_*) + \frac{(p-1)!}{d^{p-1}}O(d^{p})} \cdot \frac{d}{p}\\
\Rightarrow g'(x_*) &= \lim\limits_{d \rightarrow 0} \frac{1}{d}\left(d-\frac{f^{(p)}(x_*) + \frac{p!}{d^p} O(d^{p+1})}{f^{(p)}(x_*) + \frac{(p-1)!}{d^{p-1}}O(d^{p})} \cdot \frac{d}{p} \right)\\
&= \lim\limits_{d \rightarrow 0}\left(1- \frac{1}{p} \cdot \frac{f^{(p)}(x_*) + \frac{d!}{h^p} O(d^{p+1})}{f^{(p)}(x_*) + \frac{(p-1)!}{d^{p-1}}O(d^{p})} \cdot \frac{d}{p} \right)
\end{align*}
Da $p\ge2$, gilt $0<g'(x_*) < 1$. Die Iteration konvergiert also linear.

\paragraph*{b)}
Zu zeigen: Das Verfahren $x^{(k+1)}=x^{(k)} - p \cdot \frac{f(x^{(k)})}{f'(x^{(k)})}$ ist quadratisch konvergent (solange man nicht sofort in der Lösung startet).\\
Beweis: Mit analogen Vorgehen für $g(x_*)$ und $'g(x_*)$ wie in Teil a) folgt:
\begin{align*}
g'(x_*) &= \lim\limits_{d \rightarrow 0} \frac{g(x_*+d)-g(x_*)}{d}\\
g(x_*+d) &= x_* + d - p \cdot \frac{f(x^{(k)})}{f'(x^{(k)})} = x_*+d-p\cdot \frac{d}{p}\frac{f'(x_*) + \frac{p!}{d^p}O(d^{p+1})}{f'(x_*) + \frac{(p-1)!}{d^{p-1}}O(d^{p})}\\
g'(x_*) &= \lim\limits_{d \rightarrow 0} d-d \cdot \frac{d}{p}\left(\frac{f'(x_*) + \frac{p!}{d^p}O(d^{p+1})}{f'(x_*) + \frac{(p-1)!}{d^{p-1}}O(d^{p})} \right) = 1-1 = 0
\end{align*}
Da $g'(x_*) = 0$ liegt eine Konvergenzordnung größer Eins vor. Welches also mindestens eine quadratische Konvergenz ist.


\subsection*{Aufgabe 19}

Gegeben $y, q \in \QQ \cap (0, \infty)$. Gesucht ein iteratives Verfahren $x_{k + 1} = \Phi (x_k)$,
das lokal und quadratisch gegen $y^q$ konvergiert und $x_k \in \QQ \; \forall k \ge 1$
gewährleistet falls $x_0 \in \QQ$.

Beobachtung: Möglicherweise ist $y^q \notin \QQ$, aber man kann jede Zahl
$q \in \QQ$ schreiben als $q =: m /n$ mit $m \in \ZZ$ und $n \in \NN$,
also $y^q = y^{m/n}$. Dann gilt $(x_k) \rightarrow y^{m/n}$ genau dann wenn
$\left(x_k^n\right)\rightarrow y^m$ und es ist $y^m \in \QQ$ als rationale Zahl
mit ganzzahligem Exponenten.

Wir definieren nun die Funktion $f(x) = x^n - y^m$, diese hat eine Nullstelle $x_*$
für $x_*^n - y^m = 0$, also  $x_*^n = y^m$, d.h. diese Nullstelle ist
$x_* = y^{m/n}= y^q$. Offensichtlich ist $f(x)$ beliebig oft
stetig differenzierbar. Es ist: $f'(x) = n x^{n - 1}$ und die Ableitung
am Fixpunkt ist $f'(x_*) =  n (y^q)^{n-1} \ne 0 \; \text{ da } \; y \ne 0$.
Somit sind die Voraussetzungen für das Newton-Iterationsverfahren erfüllt:
Die Iterationsvorschrift lautet also:
\begin{align}
  \label{eq-19-iter}
  x_{k+1} =  \Phi (x_k)
  = x_k - \frac{f(x_k)}{f'(x_k)} = x_k - \frac{x_k^n - y^m}{n x_k^{n-1}}
  = \frac{1}{n} \left((n -1)x_k + \frac{y^m}{x_k^{n-1}} \right)
\end{align}
Aus der Vorlesung ist bekannt, dass das Newton-Verfahren lokal quadratisch
konvergiert, außerdem erkennt man durch scharfes Hinsehen, dass die rechte Seite
von \eqref{eq-19-iter} unter den gegebenen Bedingungen rational ist, somit
genügt diese Iterationsvorschrift den Anforderungen.


\subsection*{Aufgabe 20}


\paragraph*{a)}

\paragraph*{b)}


\end{document}
